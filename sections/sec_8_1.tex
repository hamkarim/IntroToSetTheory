\subsection{The Axiom of Choice and its Equivalents}

\exercise{1}{
  Prove: If a set $A$ can be linearly ordered, then every system of finite subsets of $A$ has a choice function.
  (It does not follow from the Zermelo-Fraenkel axioms that every set can be linearly ordered.)
}
\sol{
  \begin{lem}\label{lem:aoc:finitelo}
    Any linear ordering of a finite set is a well-ordering.
  \end{lem}
  \qproof{
    We show this by strong induction on the cardinality of the set.
    So consider natural number $n$ and suppose that all linear ordered sets of cardinality $k  < n$ are well-orderings.
    Also suppose that $(A, \prece)$ is any linearly ordered set with $\abs{A} = n$.
    Consider any nonempty $B \ss A$ so that there is a $b \in B$.
    Then clearly $C = B - \braces{b}$ is also a finite set with $C \pss B \ss A$ so that $\abs{C} < n$.
    Clearly also $C$ is linearly ordered by $\prece$ so that, by the induction hypothesis, $C$ is well-ordered by $\prece$.
    Now, if $C = \es$, then it follows that $B = \braces{b}$, which clearly has least element $b$.
    On the other hand, if $C \neq \es$, then it has a least element $c$ since it is well-ordered.
    Since $\prece$ is a \emph{linear} ordering, it has to be that either $c \prece b$ or $b \prece c$.

    Case: $c \prece b$.
    Then consider any $x \in B$.
    If $x = b$ then obviously $c \prece b = x$.
    If $x \neq b$ then $x \in C = B - \braces{b}$ so that again $c \prece x$ since $c$ is the least element of $C$.
    This shows that $c$ is the least element of $B$ since $x$ was arbitrary.

    Case: $b \prece c$.
    Then consider any $x \in B$.
    If $x = b$ then obviously $b \prece b = x$.
    If $x \neq b$ then $x \in C = B - \braces{b}$ so that $b \prece c \prece x$ since $c$ is the least element of $C$.
    This shows that $b$ is the least element of $B$ since $x$ was arbitrary.

    Hence in all cases we have that $B$ has a $\prece$-least element.
    This shows that $\prece$ is a well-ordering of $A$ since $B \ss A$ was arbitrary.
    This completes the inductive proof.
  }

  \mainprob
  \qproof{
    Suppose that $A$ is a set that can be linearly ordered and suppose that $\prece$ is a such a linear ordering.
    Suppose also that $S$ is a system of finite subsets of $A$.
    Then clearly any $B \in S$ is finite and linearly ordered by $\prece$.
    We then have that $\prece$ is a well-ordering of $B$ by Lemma~\ref{lem:aoc:finitelo}.
    So we then set
    \gath{
      f(B) = \begin{cases}
        \text{the least element of $B$ according to $\prece$} & B \neq \es \\
        \es & B = \es
      \end{cases}
    }
    for any $B \in S$.
    Clearly then $f$ is a choice function for $S$.
  }
}

\exercise{2}{
  If $A$ can be well-ordered, then $\pset{A}$ can be linearly ordered.
  [Hint: Let $<$ be a well-ordering of $A$; for $X,Y \ss A$ define $X \prec Y$ if any only if the $<$-least element of $X \sd Y$ belongs to $X$.]
}
\sol{
  \qproof{
    Suppose that $<$ is a well-ordering of $A$.
    Then, following the hint, defined the relation $X \prec Y$ if and only if the $<$-least element of $X \sd Y$ is in $X$ for any $X$ and $Y$ in $\pset{A}$.
    Note that, for any $x \in X \sd Y = (X - Y) \cup (Y - X)$, we clearly have that $x \in X$ or $x \in Y$ so that $x \in A$ since $X \ss A$ and $Y \ss A$.
    Hence $X \sd Y \ss A$ so that it is also well-ordered by $<$.

    First we show that $\prec$ is a (strict) order on $\pset{A}$.
    Hence we must show that it is asymmetric and transitive.
    So consider any $X$ and $Y$ in $\pset{A}$ where $X \prec Y$.
    Then by definition the $<$-least element $x$ in $X \sd Y$ is in $X$.
    Suppose also that $Y \prec X$ so that, since $Y \sd X = X \sd Y$, $x$ is also in $Y$.
    Then clearly $x$ can be neither in $X - Y$ nor $Y - X$, but then it cannot be that $x \in (X - Y) \cup (Y - X) = X \sd Y$.
    This is a contradiction since $x$ was defined to be in $X \sd Y$.
    Hence it cannot be that $Y \prec X$ as well, which shows that $\prec$ is asymmetric since $X$ and $Y$ were arbitrary.

    To see that $\prec$ is transitive, consider any $X, Y, Z \in \pset{A}$ where $X \prec Y$ and $Y \prec Z$.
    Then the least element $x$ of $X \sd Y$ is in $X$ and the least element $y$ of $Y \sd Z$ is in $Y$.
    Thus it has to be that $x \in X - Y$ and $y \in Y - Z$ so that $x \in X$, $x \notin Y$, $y \in Y$, and $y \notin Z$.
    Note that, in particular, this means that $x \neq y$.
    Since clearly $\braces{x, y} \ss A$, it follows that it has a $<$-least element $a$.
    Thus either $a = x$ or $a = y$.
    For each case we show that
    \begin{enumerate}
    \item $a \in X$
    \item $a \in X \sd Z$
    \item $a$ is a lower bound of $X \sd Z$
    \end{enumerate}

    Case: $a = x$.
    Then clearly $a = x \leq y$ so that $a < y$ since $a = x \neq y$.
    \begin{enumerate}
    \item Clearly $a \in X$ since $a = x$.
    \item Suppose that $a \in Z$.
      Then, since $a = x \notin Y$, we have that $a \in Z - Y$ so that $a \in Y \sd Z$.
      Hence $y \leq a$ since $y$ is the least element of $Y \sd Z$, but this contradicts the fact that $y > a$.
      So it must be that in fact $a \notin Z$ so that $a \in X - Z$.
      Thus $a \in X \sd Z$.
    \item Now consider any $z \in X \sd Z$.

      Case: $z \in X - Z$.
      Then, if $z \in Y$, we have that $z \in Y - Z$ so that $z \in Y \sd Z$.
      It then follows that $a = x \leq y \leq z$ since $y$ is the least element of $Y \sd Z$.
      On the other hand, if $z \notin Y$, then we have $z \in X - Y$ so that $z \in X \sd Y$.
      Hence $a = x \leq z$ since $x$ is the least element of $X \sd Y$.

      Case: $x \in Z - X$.
      Then, if $z \in Y$, we have $x \in Y - X$ so $z \in X \sd Y$.
      Then $a = x \leq z$ since $x$ is the least element of $X \sd Y$.
      On the other hand, if $z \notin Y$, then we have $z \in Z - Y$ so that $z \in Y \sd Z$.
      Then, as before, $a = x \leq y \leq z$ since $y$ is the least element of $Y \sd Z$.

      Hence in all cases $a \leq z$ so that $a$ is a lower bound since $z$ was arbitrary.
    \end{enumerate}

    Case: $a = y$.
    Then clearly $a = y \leq x$ so that $a < x$ since $a = y \neq x$.
    \begin{enumerate}
    \item Suppose that $a \notin X$.
      Then since $a = y \in Y$ we have that $a \in Y - X$ so that also $a \in X \sd Y$.
      But then $x \leq a$ since $x$ is the least element of $X \sd Y$, which contradicts the fact that $x > a$.
      Hence it has to be that $a \in X$.
    \item We already know that $a = y \notin Z$ so that $a \in X - Z$ since we just showed that $a \in X$.
      Hence $a \in X \sd Z$.
    \item Consider any $z \in X \sd Z$.

      Case: $z \in X - Z$.
      If also $z \in Y$ then clearly $z \in Y - Z$ so that $z \in Y \sd Z$.
      It then follows that $a = y \leq z$ since $y$ is the least element of $Y \sd Z$.
      On the other hand, if $z \notin Y$, then clearly $z \in X - Y$ so that $z \in X \sd Y$.
      Hence $a = y \leq x \leq z$ since $x$ is the least element of $X \sd Y$.

      Case: $z \in Z - X$.
      Then, if $z \in Y$, clearly $z \in Y - X$ so that $z \in X \sd Y$.
      We then have that $a = y \leq x \leq z$ since $x$ is the least element of $X \sd Y$.
      On the other hand, if $z \notin Y$, then $z \in Z - Y$ so that $z \in Y \sd Z$.
      Then clearly $a = y \leq z$ since $y$ is the least element of $Y \sd Z$.

      Hence in all cases $a \leq z$ so that $a$ is a lower bound since $z$ was arbitrary.
    \end{enumerate}
    
    Thus in all cases we have that $a$ is the least element of $X \sd Z$ (since it is in $X \sd Z$ and also is a lower bound) and $a \in X$.
    By definition, this shows that $X \prec Z$ so that $\prec$ is transitive.
    This also shows that $\prec$ is a (strict) order.

    Lastly, we show that $\prec$ is a linear ordering.
    So consider any $X, Y \in \pset{A}$.
    Assume that $X \neq Y$ so that $X - Y \neq \es$ or $Y - X \neq \es$ (or both).
    From this it follows that $X \sd Y \neq \es$.
    Since also clearly $X \sd Y \ss A$, it has a least element $a$.
    If $a \in X - Y$ then $a \in X$ so that $X \prec Y$.
    Similarly, if $a \in Y - X$ then $a \in Y$ so that $Y \prec X$.
    Hence we have shown that either $X = Y$, $X \prec Y$, or $Y \prec X$ so that $\prec$ is in fact linear since $X$ and $Y$ were arbitrary.

    This completes the proof since we have shown that $\prec$ is a linear ordering of $\pset{A}$.
  }
}

\exerciseapp{3}{*}{
  Let $(A, \leq)$ be an ordered set in which every chain has an upper bound.
  Then for every $a \in A$, there is a $\leq$-maximal element of $x$ of $A$ such that $a \leq x$.
}
\sol{
  \begin{lem}\label{lem:aoc:notin}
  For any set $A$, there is a $b \notin A$.
\end{lem}
\qproof{
  Let $X = \braces{\a \in A \where \a \text{ is an ordinal number}}$.
  Then by Theorem~6.2.6e there is an ordinal $\a$ such that $\a \notin X$.
  It also has to be that $\a \notin A$ since, if it were, then $\a$ would be in $X$ since it is an ordinal number, which would be a contradiction.
}


  \mainprob
  
  The proof of this is similar to the proof of Zorn's Lemma from the Axiom of Choice (part of Theorem~8.1.13 in the text).
  \qproof{
    First, by Lemma~\ref{lem:aoc:notin}, there is a $b \notin A$.
    Also, by the Axiom of Choice, there is a choice function $g$ on $\pset{A}$.
    Now consider any $a \in A$.
    We then define a transfinite sequence $\angles{a_\a \where \a < h(A)}$ by transfinite recursion as follows.
    Set $a_0 = a$.
    Then, having constructed the sequence $\angles{a_\x \where \x < \a}$ for $0 < \a < h(A)$, we define the set $A_\a = \braces{x \in A \where a_\x < x \text{ for all } \x < \a}$.
    We then set
    \gath{
      a_\a = \begin{cases}
        g(A_\a) & \text{if $a_\x \neq b$ for all $\x < \a$ and $A_\a \neq \es$} \\
        b & \text{otherwise} \,.
      \end{cases}
    }

    We claim that there is an $\a < h(A)$ such that $a_\a = b$.
    To see this, suppose to the contrary that $a_\a \neq b$ for all $\a < h(A)$ so that it has to be that each $a_\a \in A$.
    Consider now any $\a < h(A)$ and $\b < h(A)$ where $\a \neq \b$.
    Without loss of generality we can assume that $\a < \b$.
    Clearly then, by definition, we have that $a_\b \in A_\b$ so that $a_\x < a_\b$ for all $\x < \b$.
    But since $\a < \b$, we have that $a_\a < a_\b$ so that $a_\a \neq a_\b$.
    Since $\a$ and $\b$ were arbitrary, this shows that the sequence is an injective function from $h(A)$ to $A$.
    However, this would mean that $h(A)$ is equipotent to some subset of $A$, which contradicts the definition of the Hartogs number.
    Hence it has to be that $a_\a = b$ for some $\a < h(A)$.

    So let $\l < h(A)$ be the least ordinal such that $a_\l = b$ and let $C = \braces{a_\x \where \x < \l}$.
    We claim that $C$ is a chain in $(A, \leq)$.
    So consider any $a_\a$ and $a_\b$ in $C$ so that $\a < \l$ and $\b < \l$
    Without loss of generality we can assume that $\a \leq \b$.
    If $\a = \b$ then obviously $a_\a = a_\b$ so that $a_a \leq a_\b$ clearly holds.
    If $\a < \b$ then, by what was shown above, we have that $a_\a < a_\b$ so that $a_\a \leq a_\b$ again holds.
    Hence, in every case, $a_\a$ and $a_\b$ are comparable in $\leq$, which shows that $C$ is a chain since $a_\a$ and $a_\b$ were arbitrary.

    Thus, since $C$ is a chain of $A$, it has an upper bound $c \in A$.
    We claim that $c$ is also a maximal element of $A$.
    To show this, suppose that there is an $x \in A$ such that $c < x$.
    Now consider any $\x < \l$.
    Then, since $c$ is an upper bound of $C$, we have that $a_\x \leq c < x$ so that $a_\x < x$ since orders are transitive.
    It then follows from the definition of $A_\l$ that $x \in A_\l$ so that $A_\l \neq \es$.
    Also note that, by the definition of $\l$, we have that $a_\x \neq b$ for any $\x < \l$.
    Thus, by the recursive definition of the sequence, it follows that $a_\l = g(A_\l) \neq b$, which contradicts the definition of $\l$ (as the least ordinal such that $a_\l = b$).
    So it has to be that there is no such element $x$, which shows that $c$ is in fact a maximal element of $A$.

    Now, it has to be that $0 \neq \l$ since $a_0 = a \neq b = a_\l$.
    It then follows that $0 < \l$ since $\l$ is an ordinal.
    Hence $a = a_0 \in C$ by the definition of $C$.
    Then, since $c$ is an upper bound of $C$, we have that $c$ is a maximal element of $A$ where $a \leq c$.
    Since $a$ was arbitrary this shows the desired result.
  }
}

\exercise{4}{
  Prove that Zorn's Lemma is equivalent to the statement: For all $(A, \leq)$, the set of all chains of $(A, \leq)$ has an $\ss$-maximal element.
}
\sol{
  \qproof{
    $(\to)$ First, suppose that Zorn's Lemma is true, and let $C$ be the set of all chains of $(A, \leq)$.
    First, it is trivial to show that $\ss$ is a partial order on $C$, i.e. that it is reflexive, antisymmetric, and transitive.
    Let $B \ss C$ be any $\ss$-chain, and let $U = \bigcup B$.

    First we claim that that $U \in C$, which requires that we show that $U$ is a chain of $(A, \leq)$.
    So consider any $x,y \in U = \bigcup B$ so that there are sets $X,Y \in B$ such that $x \in X$ and $y \in Y$.
    Since $B$ is a $\ss$-chain it follows that either $X \ss Y$ or $Y \ss X$.
    In the case of $X \ss Y$ then clearly both $x$ and $y$ are in $Y$ (since $x \in X$ and $X \ss Y$).
    Then, since $Y \in C$ (since $Y \in B$ and $B \ss C$), we have that $Y$ is a $\leq$-chain.
    Hence $x$ and $y$ are comparable in $\leq$.
    The case in which $Y \ss X$ is analogous.
    Since $x,y \in U$ were arbitrary, this shows that $U$ is a $\leq$-chain so that $U \in C$.

    We also claim that $U$ is an upper bound (with respect to $\ss$) of $B$.
    To show this, consider any $X \in B$ and any $x \in X$.
    Then clearly $x \in \bigcup B = U$.
    Hence $X \ss U$ since $x$ was arbitrary.
    Since also $X \in B$ was arbitrary, this shows that $U$ is an upper bound of $B$.

    Thus, since $B$ was an arbitrary $\ss$-chain, this shows that every chain of $(C, \ss)$ has an upper bound.
    It then follows from Zorn's Lemma that $C$ has a $\ss$-maximal element as desired.

    $(\leftarrow)$ Suppose that the set of all chains of $(A, \leq)$ has a $\ss$-maximal element for any $(A, \leq)$.
    So consider any such ordered set $(A, \leq)$ where every chain has a upper bound.
    Let $C$ be the set of all chains of $(A, \leq)$ so that $C$ has a $\ss$-maximal element $M$ by our initial supposition.
    Then, since $M \in C$, it is a chain so it has an upper bound $a \in A$.
    We claim that $a$ is a maximal element of $(A, \leq)$.

    To show this, assume to the contrary that there is a $b \in A$ such that $a < b$, and let $M' = M \cup \braces{b}$.
    Consider any $x,y \in M'$.
    If $x,y \in M$ then clearly $x$ and $y$ are comparable in $\leq$ since $M$ is a chain.
    On the other hand, if $x \in M$ but $y = b$, then $x \leq a$ since $a$ is an upper bound of $M$.
    We also have that $a < b = y$ so that $x < y$ since orders are transitive.
    The case in which $y \in M$ but $x = b$ similarly leads to $y < x$.
    Lastly, if $x = y = b$ then clearly $x \leq y$ is true.
    Hence in all cases $x$ and $y$ are comparable in $\leq$, which shows that $M'$ is a chain since $x$ and $y$ were arbitrary.
    Therefore $M' \in C$.

    Now, it has to be that $b \notin M$ since, if it were, $a$ could not be an upper bound of $M$ since $a < b$ (and therefore it cannot be that $b \leq a$ since the strict ordering is asymmetric).
    So, since $b \notin M$ it follows that $M \pss M \cup \braces{b} = M'$, which contradicts the fact that $M$ is a $\ss$-maximal element of $C$ since also $M' \in C$.
    So it has to be that there is no such $b$ where $a < b$, which shows that $a$ is in fact a maximal element of $(A, \leq)$.
    This proves Zorn's Lemma since $(A, \leq)$ was arbitrary.
  }
}

\exercise{5}{
  Prove that Zorn's Lemma is equivalent to the statement: If $A$ is a system of sets such that, for each $B \ss A$ which is linearly ordered by $\ss$, $\bigcup B  \in A$, then $A$ has an $\ss$-maximal element.
}
\sol{
  \qproof{
    $(\to)$ Suppose Zorn's Lemma and let $A$ be a system of sets where $\bigcup B \in A$ for any $B$ that is linearly ordered by $\ss$.
    We know that $\ss$ is a partial order on $A$.
    So let $B$ be any $\ss$-chain of $A$.
    Then we know that $\bigcup B \in A$, and we also claim that $\bigcup B$ is an upper bound of $B$.
    To see this, consider any $X \in B$ and any $x \in X$, so that clearly $x \in \bigcup B$.
    Hence $X \ss B$ since $x$ was arbitrary.
    This shows that $\bigcup B$ is an upper bound of $B$ since $X$ was arbitrary.
    Since $B$ was an arbitrary chain, this shows that $(A, \ss)$ is an ordered set where every chain has an upper bound.
    Thus by Zorn's Lemma there is a $\ss$-maximal element of $A$ as desired.

    $(\leftarrow)$ Now suppose that $A$ has a $\ss$-maximal element for any system of sets $A$ such that $\bigcup B \in A$ for any $B \ss A$ where $B$ is linearly ordered by  $\ss$.
    Consider any ordered set $(A, \leq)$ and let $C$ be the set of all chains of $A$.
    Let $B$ be any subset of $C$ that is linearly ordered by $\ss$, and consider any $x$ and $y$ in $\bigcup B$.
    Then there are $X$ and $Y$ in $B$ such that $x \in X$ and $y \in Y$.
    Since $B$ is linearly ordered by $\ss$ we have that either $X \ss Y$ or $Y \ss X$.
    In the former case we have $x \in X \ss Y$ so that both $x$ and $y$ are in $Y$.
    Hence $x$ and $y$ are comparable in $\leq$ since $Y \in B \ss C$ so that $Y$ is a $\leq$-chain.
    A similar argument shows that $x$ and $y$ are comparable if $Y \ss X$.
    Since $x$ and $y$ were arbitrary this shows that $\bigcup B$ is a $\leq$-chain so that $\bigcup B \in C$.

    Thus $C$ is a system of sets that meet the criteria of the initial supposition since $B$ was arbitrary.
    Hence $C$ has a $\ss$-maximal element.
    Since $(A, \leq)$ was an arbitrary ordered set and we have shown that the set of all chains of $(A, \leq)$ has a $\ss$-maximal element, Zorn's Lemma follows from Exercise~8.1.4.
  }
}

\exercise{6}{
  A system of sets $A$ has \emph{finite character} if $X \in A$ if and only if every finite subset of $X$ belongs to $A$.
  Prove that Zorn's Lemma is equivalent to the following (Tukey's Lemma): Every system of sets of finite character has an $\ss$-maximal element.
  [Hint: Use Exercise~8.1.5.]
}
\sol{
  \qproof{
    $(\to)$ Suppose Zorn's Lemma and let $A$ be an arbitrary system of sets of finite character.
    Suppose that $B$ is any subset of $A$ that is linearly ordered by $\ss$ and let $C$ be any finite subset of $\bigcup B$.
    Now, for each $x \in C$ there is a set $X_x \in B$ such that $x \in X_x$, since $x \in C \ss \bigcup B$.
    Clearly the set $D = \braces{X_x \where x \in C}$ is a subset of $B$ so that $D$ is also linearly ordered by $\ss$.
    Also clearly $D$ is finite since $C$ is.
    Hence $D$ has a $\ss$-greatest element $X$.\
    Note that the Axiom of Choice is not needed in selecting the set $X_x$ for each $x \in C$ since we are only making a finite number of choices.
    So consider any $x \in C$ so that $x \in X_x \ss X$.
    Hence $x \in X$ so that $C \ss X$ since $x$ was arbitrary.
    We also have that $X \in B \ss A$ so $X \in A$.
    Therefore $C$ is a finite subset of $X$, which is an element of $A$, so that $C$ is also in $A$ since $A$ has finite character.
    Since $C$ was an arbitrary finite subset of $\bigcup B$ and $C \in A$ it follows that $\bigcup B \in A$.
    Hence, since $B$ was an arbitrary linearly ordered (by $\ss$) subset of $A$, we have by Exercise~8.1.5 and Zorn's Lemma that $A$ has a $\ss$-maximal element as desired.

    $(\leftarrow)$ Now suppose that every system of sets of finite character has a $\ss$-maximal element.
    Let $(A, \leq)$ be any ordered set and let $C$ be the set of all chains of $(A, \leq)$.
    Now suppose that $X \in C$ and let $Y$ be any finite subset of $X$.
    Clearly since $X \in C$, it is linearly ordered by $\leq$ so that $Y$ is as well since $Y \ss X$.
    Hence $Y \in C$.
    Now let $X'$ be any set such that every finite subset of $X'$ is in $C$.
    Consider any $x,y \in X'$.
    Then $\braces{x, y}$ is clearly a finite subset of $X'$ so that it is in $C$ and therefore a $\leq$-chain.
    Hence $x$ and $y$ are comparable in $\leq$, which shows that $X'$ itself is a $\leq$-chain since $x$ and $y$ were arbitrary.
    Hence $X' \in C$.
    Thus we have just shown that $X \in C$ if and only if every finite subset of $X$ is in $C$ so that $C$ has finite character by definition.
    Therefore, by the initial supposition, $C$ has a $\ss$ maximal element.
    Since again $C$ is the set of all chains of the arbitrary $(A, \leq)$, Zorn's Lemma follows from Exercise~8.1.4.
  }
}

\exerciseapp{7}{*}{
  Let $E$ be a binary relation on a set $A$.
  Show that there exists a function $f: A \to A$ such that for all $x \in A$, $(x, f(x)) \in E$ if and only if there is some $y \in A$ such that $(x, y) \in E$.
}
\sol{
  \qproof{
    If $A = \es$ then clearly it must be that $E = \es$ since $E \ss A \times A = \es \times \es = \es$.
    Hence $f = \es$ is vacuously such a function.
    So assume that $A \neq \es$ so that there is an $a \in A$.
    For any $x \in A$ define the set $Y_x = \braces{y \in A \where (x,y) \in E}$, noting that this could certainly be empty if $x$ is not in the domain of $E$.
    Clearly $S = \braces{Y_x \where x \in A}$ is a system of sets, and so has a choice function $g$ by the Axiom of Choice.
    We then define a function $f : A \to A$ by
    \gath{
      f(x) = \begin{cases}
        g(Y_x) & Y_x \neq \es \\
        a & Y_x = \es
      \end{cases}
    }
    for all $x \in A$.
    We claim that $f$ meets the required criteria, so let $x$ be some element of $A$.

    $(\to)$ Suppose that $(x, f(x)) \in E$.
    Then clearly for $y = f(x)$ we have that $(x, y) = (x, f(x)) \in E$.
    We note that, if $Y = \es$, then $y = f(x) = a \in A$, and if $Y_x \neq \es$ then $y = f(x) = g(Y_x) \in Y_x$ since $g$ is a choice function so that again $y \in A$ since clearly $Y_x \ss A$.

    $(\leftarrow)$ Now suppose that there is a $y \in A$ such that $(x,y) \in E$.
    Then clearly by definition we have $y \in Y_x$ so that $Y_x \neq \es$.
    Thus $f(x) = g(Y_x) \in Y_x$ since $g$ is a choice function.
    We therefore have $(x, f(x)) \in E$ as desired, again by the definition of $Y_x$.
  }
}
