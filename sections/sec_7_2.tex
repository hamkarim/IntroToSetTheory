\subsection{Addition and Multiplication of Alephs}

\def\ala{\al_\a}
\exercise{1}{
  Give a direct proof of $\ala + \ala = \ala$ by expressing $\w_\a$ as a disjoint union of two sets of cardinality $\ala$.
}
\sol {
  \begin{lem}\label{lem:aleph:alephadd:ordgt}
    If $\a$ and $\b$ are ordinals and $\a > \b$ then there is a $\g  <\a$ such that $\abs{\b} \leq \abs{\g}$.
  \end{lem}
  \qproof{
    Clearly for $\g = \b$ we have $\g = \b < \a$ and $\abs{\b} = \abs{\g}$ so that $\abs{\b} \leq \abs{\g}$ is true.
  }

  \begin{lem}\label{lem:aleph:alephadd:setgt}
    If a set $(A, \prec)$ is isomorphic to ordinal $\a$ and $\a > \b$ for another ordinal $\b$ then there is an $a \in A$ such that $\abs{\b} \leq \abs{X}$ for the set
    $$
    X = \braces{x \in A \where x \prec a} \,.
    $$
  \end{lem}
  \qproof{
    First let $f$ be the isomorphism from $\a$ to $A$.
    Clearly by Lemma~\ref{lem:aleph:alephadd:ordgt} there is an ordinal $\g < \a$ such that $\abs{\b} \leq \abs{\g}$.
    Now let $a = f(\g)$ so that clearly $a \in A$, and let
    $$
    X = \braces{x \in A \where x \prec a} \,.
    $$
    Now, we claim that $X = f[\g]$.
    So consider any $x \in X$ so that $x \prec a$.
    It then follows that $\inv{f}(x) < \inv{f}(a) = \g$ since $\inv{f}$ is an isomorphism since $f$ is.
    Hence $\inv{f}(x) \in \g$ so that clearly $x = f(\inv{f}(x)) \in f[\g]$.
    Since $x$ was arbitrary it follows that $X \ss f[\g]$.
    Now consider any $x \in f[\g]$ so that there is a $\d \in \g$ such that $x = f(\d)$.
    Then $\d < \g$ so that $x = f(\d) \prec f(\g) = a$ since $f$ is an isomorphism so that by definition $x \in X$.
    Hence $f[\g] \ss X$.
    This shows that $X = f[\g]$.
    Then, since $f$ is bijective, it follows that $\abs{\b} \leq \abs{\g} = \abs{f[\g]} = \abs{X}$.
  }

  \begin{lem}\label{lem:aleph:initle}
  For initial ordinals $\a$ and $\b$, if $\abs{\a} \leq \abs{\b}$, then $\a \leq \b$.
\end{lem}
\qproof{
  Suppose that $\abs{\a} \leq \abs{\b}$ but that $\a > \b$.
  Then clearly $\b$ is isomorphic (and therefore equipotent) to an initial segment of $\a$ so that $\abs{\b} \leq \abs{\a}$.
  Then by the \cbthrm{} we have $\abs{\a} = \abs{\b}$.
  However since $\a$ is an initial ordinal and $\b < \a$ it cannot be that $\abs{\a} = \abs{\b}$.
  Thus we have a contradiction so that it must be that $\a \leq \b$ as desired.
}

  \begin{lem}\label{lem:aleph:initltwa}
  For any ordinal $\a$ and any infinite initial ordinal $\W$ where $\W < \w_\a$, there is a $\g < \a$ such that $\W = \w_\g$.
\end{lem}
\qproof{
  We show this by induction on $\a$.
  For $\a = 0$ we have $\w_\a = \w_0 = \w$ so that there is no infinite initial ordinal $\W$ such that $\W < \w_\a = \w$.
  Hence the hypothesis is vacuously true.
  Now suppose that, for every infinite initial ordinal $\W < \w_\a$, there is a $\g < \a$ such that $\W = \w_\g$.
  Consider any infinite initial ordinal $\W < \w_{\a+1}$.
  Then $\W < \w_{\a+1} = h(\w_\a)$ so that $\W$ is equipotent to some subset of $\w_\a$ by the definition of the Hartogs number.
  From this it clearly follows that $\abs{\W} \leq \abs{\w_\a}$ and hence $\W \leq \w_\a$ by Lemma~\ref{lem:aleph:initle} since both $\W$ and $\w_\a$ are initial ordinals.
  If $\W = \w_\a$ then we are finished but if $\W < \w_\a$ then by the induction hypothesis there is a $\g < \a$ such that $\W = \w_\g$ so that we are also finished.

  Now suppose that $\a$ is a nonzero limit ordinal and that for every $\b < \a$ and infinite initial ordinal $\W < \w_\b$ there is a $\g < \b$ such that $\W = \w_\g$.
  Consider then any infinite initial ordinal $\W < \w_\a$.
  Then since $\w_\a = \sup\braces{\w_\b \where \b < \a}$ it follows that $\W$ is not an upper bound of $\braces{\w_\b \where \b < \a}$ so that there is a $\b < \a$ such that $\W < \w_\b$.
  But then by the induction hypothesis there is a $\g < \b$ such that $\W = \w_\g$.
  This completes the transfinite induction.
}


  \begin{lem}\label{lem:aleph:alephadd:alephlt}
    For ordinal $\a > 0$ and an ordinal $\b < \w_\a$ there is an ordinal $\g < \a$ such that $\abs{\b} \leq \al_\g$.
  \end{lem}
  \qproof{
    First, if $\b$ is finite then clearly $\b < \w$ so that $\b \in \w$.
    Then $\b \ss \w$ since $\w$ is transitive (since it is an ordinal number).
    Hence $\abs{\b} \leq \abs{\w} = \al_0$ (i.e. $\g = 0$ so that $\g < \a$).
    On the other hand if $\b$ is infinite then by Theorem~7.1.3 $\b$ is equipotent to some initial ordinal $\W$.
    Clearly $\W$ is infinite since $\b$ is and clearly $\W < \w_\a$ since $\b < \w_\a$ and $\w_\a$ is an initial ordinal.
    It then follows from Lemma~\ref{lem:aleph:initltwa} that there is a $\g < \a$ such that $\W = \w_\g$.
    Then we have $\abs{\b} = \abs{\W} = \abs{\w_\g} = \al_\g$ so that $\abs{\b} \leq \al_\g$ is true.
  }

  \begin{lem}\label{lem:aleph:initlimit}
  Every infinite initial ordinal is a limit ordinal.
\end{lem}
\qproof{
  Suppose that $\a$ is an infinite initial ordinal and that it a successor so that $\a = \b+1$.
  It was shown in Lemma~\ref{lem:aleph:infwo:succ} that $\abs{\b} = \abs{\b+1} = \abs{\a}$, but since clearly $\b < \a$ this contradicts the fact that $\a$ is an initial ordinal.
  Hence $\a$ must be a limit ordinal.
}

  \def\aA{\abs{A}}
\def\aB{\abs{B}}
\begin{lem}\label{lem:aleph:wolege}
  For well ordered sets $A$ and $B$ either $\aA \leq \aB$ or $\aB \leq \aA$ (or both in which case $\aA = \aB$).
\end{lem}
\qproof{
  By Theorem~6.1.3 we have:

  Case: $A$ and $B$ are isomorphic.
  Let $f$ be the isomorphism from $A$ to $B$.
  Then clearly $f$ is a bijection so that $\aA = \aB$.
  Also since $f$ is injective $\aA \leq \aB$.
  Clearly also $\inv{f}$ is bijective from $B$ to $A$ so that $\aB \leq \aA$ as well.

  Case: $A$ is isomorphic to an initial segment of $B$.
  Then if $f$ is the isomorphism clearly $f$ is an injective function from $A$ to $B$ so that $\aA \leq \aB$.

  Case: $B$ is isomorphic to an initial segment of $A$.
  Then if $f$ is the isomorphism clearly $f$ is an injective function from $B$ to $A$ so that $\aB \leq \aA$.

  Since these cases are exhaustive by Theorem~6.1.3 clearly the result has been shown.

  Note that this did not require the Axiom of Choice.
}

  \def\aA{\abs{A}}
\def\aB{\abs{B}}
\begin{cor}\label{cor:aleph:wonlegt}
  If $A$ and $B$ are well ordered sets then $\aA \nleq \aB$ if and only if $\aB < \aA$.
\end{cor}
\qproof{
  ($\to$) Suppose that $\aA \nleq \aB$.
  Then it follows from Lemma~\ref{lem:aleph:wolege} above that $\aB \leq \aA$.
  Suppose that $\aB = \aA$.
  Then there is a bijection $f$ from $B$ to $A$.
  But then clearly $\inv{f}$ is also a bijection and therefore injective.
  Hence by definition $\aA \leq \aB$, a contradiction.
  So it cannot be that $\aB = \aA$.
  Hence $\aB < \aA$ by definition as desired.

  ($\leftarrow$) We show this by proving the contrapositive.
  So suppose that $\aA \leq \aB$.
  Also suppose that $\aB \leq \aA$ so that by Lemma~\ref{lem:aleph:wolege} above $\aA = \aB$.
  Thus we have shown that
  \gath{
    \aB \leq \aA \to \aA = \aB \\
    \aB \nleq \aA \lor \aA = \aB \\
    \lnot \parens{\aB \leq \aA \land \aA \neq \aB} \\
    \lnot (\aB < \aA) \,,
  }
  thereby showing the contrapositive.
}


  \mainprob

  The following proof is similar to the proof of Theorem~7.2.1.
  \qproof{
    Suppose that $A_1$ and $A_2$ are disjoint sets that are both equipotent to $\w_\a$ for some ordinal $\a$.
    Then there are bijections $f_1$ and $f_2$ from $A_1$ and $A_2$, respectively, to $\w_\a$.
    We define a well-ordering $\prec$ of $A = A_1 \cup A_2$ as follows: for $a$ and $b$ in $A$ we let $a \prec b$ if and only if
    \begin{itemize}
    \item $a$ and $b$ are in $A_1$ and $f_1(a) < f_1(b)$, or
    \item $a$ and $b$ are in $A_2$ and $f_2(a) < f_2(b)$, or
    \item $a \in A_1$ and $b \in A_2$ and $f_1(a) \leq f_2(b)$, or
    \item $a \in A_2$ and $b \in A_1$ and $f_2(a) < f_1(b)$.
    \end{itemize}

    First we show that $\prec$ is transitive.
    So consider $a$, $b$, and $c$ in $A$ such that $a \prec b$ and $b \prec c$.

    Case: $a \in A_1$
    \begin{indpar}
      Case: $b \in A_1$
      \begin{indpar}
        Case: $c \in A_1$.
        Then $f_1(a) < f_1(b) < f_1(c)$ so that $f_1(a) < f_1(c)$ and hence $a \prec c$.

        Case: $c \in A_2$.
        Then $f_1(a) < f_1(b) \leq f_2(c)$ so that $f_1(a) \leq f_2(c)$ is true and hence $a \prec c$.
      \end{indpar}

      Case: $b \in A_2$
      \begin{indpar}
        Case: $c \in A_1$.
        Then $f_1(a) \leq f_2(b) < f_1(c)$ so that $f_1(a) < f_1(c)$ and hence $a \prec c$.

        Case: $c \in A_2$.
        Then $f_1(a) \leq f_2(b) < f_2(c)$ so that $f_1(a) \leq f_2(c)$ is true and hence $a \prec c$.
      \end{indpar}
    \end{indpar}

    Case: $a \in A_2$
    \begin{indpar}
      Case: $b \in A_1$
      \begin{indpar}
        Case: $c \in A_1$.
        Then $f_2(a) < f_1(b) < f_1(c)$ so that $f_2(a) < f_1(c)$ and hence $a \prec c$.

        Case: $c \in A_2$.
        Then $f_2(a) < f_1(b) \leq f_2(c)$ so that $f_2(a) < f_2(c)$ and hence $a \prec c$.
      \end{indpar}

      Case: $b \in A_2$
      \begin{indpar}
        Case: $c \in A_1$.
        Then $f_2(a) < f_2(b) < f_1(c)$ so that $f_2(a) < f_1(c)$ and hence $a \prec c$.

        Case: $c \in A_2$.
        Then $f_2(a) < f_2(b) < f_2(c)$ so that $f_2(a) < f_2(c)$ and hence $a \prec c$.
      \end{indpar}
    \end{indpar}
    Since all cases imply that $a \prec c$ and $a$, $b$, and $c$ were arbitrary this shows that $\prec$ is transitive.

    Now we show that for any $a$ and $b$ in $A$, that either $a \prec b$, $a = b$, or $b \prec a$ and that only one of these is true.
    So consider any $a$ and $b$ in $A$.
    Then we have

    Case: $a \in A_1$
    \begin{indpar}
      Case: $b \in A_1$.
      Then clearly exactly one of the following is true: $a \prec b$ if $f_1(a) < f_1(b)$, $a = b$ if $f_1(a) = f_1(b)$ since $f_1$ is a bijection, and $b \prec a$ if $f_1(b) < f_1(a)$.

      Case: $b \in A_2$.
      Clearly $a = b$ is not possible since $A_1$ and $A_2$ are disjoint.
      Then if $f_1(a) \leq f_2(b)$ then $a \prec b$ and if $f_1(a) > f_2(b)$ then $b \prec a$, noting that these are mutually exclusive.
    \end{indpar}

    Case: $a \in A_2$
    \begin{indpar}
      Case: $b \in A_1$.
      Then again clearly $a = b$ is not possible since $A_1$ and $A_2$ are disjoint.
      Then if $f_2(a) < f_1(b)$ then $a \prec b$ and if $f_2(a) \geq f_1(b)$ then $b \prec a$, noting that these are mutually exclusive.

      Case: $b \in A_2$.
      Then clearly exactly one of the following is true: $a \prec b$ if $f_2(a) < f_2(b)$, $a = b$ if $f_2(a) = f_2(b)$ since $f_2$ is a bijection, and $b \prec a$ if $f_2(b) < f_2(a)$.
    \end{indpar}
    Thus we have shown that $\prec$ is a total (strict) order on $A$.

    Now we show that $\prec$ is also a well-ordering.
    So let $X$ be a nonempty subset of $A$.
    Let $B = f_1[X \cap A_1] \cup f_2[X \cap A_2]$, noting that this is a nonempty set of ordinals.
    Then let $B$ has a least element $\a$.

    Case: $\a \in f_1[X \cap A_1]$.
    Then we claim that $x = \inv{f_1}(\a)$ is the $\prec$-least element of $X$, noting that $x \in A_1$.
    Note also that clearly then $f_1(x) = f_1(\inv{f_1}(\a)) = \a$.
    So consider any $y \in X$.
    \begin{indpar}
      Case: $y \in A_1$.
      Then $y \in X \cap A_1$ so that $f_1(y) \in f_1[X \cap A_1]$ so $f_1(y) \in B$.
      Thus $f_1(x) = \a \leq f_1(y)$ since $\a$ is the least element of $B$.
      Clearly if $f_1(x) = f_1(y)$ then $x = y$ since $f_1$ is bijective.
      On the other hand if $f_1(x) < f_1(y)$ then by definition $x \prec y$.
      Hence in either case we have $x \prece y$.

      Case: $y \in A_2$.
      Then $y \in X \cap A_2$ so that $f_2(y) \in f_2[X \cap A_2]$ so that $f_2(y) \in B$.
      Thus $f_1(x) = \a \leq f_2(y)$ so that by definition $x \prec y$.
      Hence again $x \prece y$ is true.
    \end{indpar}

    Case: $\a \notin f_1[X \cap A_1]$.
    Then it has to be that $\a \in f_2[X \cap A_2]$.
    Then we claim that $x = \inv{f_2}(\a)$ is the $\prec$-least element of $X$, noting that $x \in A_2$.
    Note also that clearly then $f_2(x) = f_2(\inv{f_2}(\a)) = \a$.
    So consider any $y \in X$.
    \begin{indpar}
      Case: $y \in A_1$.
      Then $y \in X \cap A_1$ so that $f_1(y) \in f_1[X \cap A_1]$ so $f_1(y) \in B$.
      Thus $f_2(x) = \a \leq f_1(y)$ since $\a$ is the least element of $B$.
      Now, it cannot be that $f_2(x) = \a = f_1(y)$ for then $\a$ would be in $f_1[X \cap A_1]$.
      So it must be that $f_2(x) = \a < f_1(y)$ so that by definition $x \prec y$ so that $x \prece y$ is true.

      Case: $y \in A_2$.
      Then $y \in X \cap A_2$ so that $f_2(y) \in f_2[X \cap A_2]$ so that $f_2(y) \in B$.
      Thus $f_2(x) = \a \leq f_2(y)$.
      Then if $f_2(x) = \a = f_2(y)$ then $x = y$ since $f_2$ is bijective.
      On the other hand if $f_2(x) = \a < f_2(y)$ then $x \prec y$ by definition.
      In either case we have $x \prece y$.
    \end{indpar}
    Hence in all cases we have shown that $X$ has a $\prec$-least element so that $\prec$ is a well-ordering of $A$.

    Now we show by transfinite induction that $\ala + \ala = \ala$ for all ordinals $\a$.
    First it was already shown in a previous chapter that $\al_0 + \al_0 = \al_0$.
    So now consider any $\a > 0$ and suppose $\al_\g + \al_\g = \al_\g$ for all $\g < \a$.

    Then consider two disjoint sets $A_1$ and $A_2$ that are both equipotent to $\w_\a$ and the well-ordering $\prec$ on $A = A_1 \cup A_2$ as defined above, also again letting $f_1$ and $f_2$ be the isomorphisms from $A_1$ and $A_2$, respectively, to $\w_\a$.
    Now let $a$ be any element of $A$ and define
    $$
    X = \braces{x \in A \where x \prec a} \,.
    $$
    Let $X_1 = X \cap A_1$ and $X_2 = X \cap A_2$ so that clearly $X_1$ and $X_2$ are disjoint and $X = X_1 \cup X_2$.
    From this it follows from the definition of cardinal addition that $\abs{X} = \abs{X_1} + \abs{X_2}$.

    If $a \in A_1$ then define $\b = f_1(a) \in \w_\a$ so that $\b < \w_\a$.
    It follows from this and Lemma~\ref{lem:aleph:alephadd:alephlt} that there is a $\g < \a$ such that $\abs{\b} \leq \al_\g$ since $\a > 0$, noting also that $\al_\g < \ala$ by the remarks following Definition~7.1.8.
    Now consider any $x_1 \in X_1$ so that also $x_1 \in X$.
    Then by definition $x_1 \prec a$ so that by the definition of $\prec$ we have $f_1(x_1) < f_1(a) = \b$ since $x_1 \in A_1$ and $a \in A_1$.
    Hence $f_1(x_1) \in \b$ so that $f_1[X_1] \ss \b$ since $x_1$ was arbitrary.
    Hence, since $f_1$ is bijective, we have $\abs{X_1} = \abs{f_1[X_1]} \leq \abs{\b} \leq \al_\g$.
    Next, consider any $x_2 \in X_2$ so that $x_2 \in X$ and hence $x_2 \prec a$.
    Then, again by the definition of $\prec$, we have that $f_2(x_2) < f_1(a) = \b$ since $x_2 \in A_2$ and $a \in A_1$.
    Hence $f_2(x_2) \in \b$ so that $f_2[X_2] \ss \b$ since $x_2$ was arbitrary.
    Thus we have $\abs{X_2} = \abs{f_2[X_2]} \leq \abs{\b} \leq \al_\g$ since $f_2$ is bijective.

    A similar argument shows that $\abs{X_1} \leq \al_\g$ and $\abs{X_2} \leq \al_\g$ for some $\g < \a$ in the case when $a \in A_2$.
    However, in this case we must set $\b = f_2(a) + 1$, noting that $\b \in \w_\a$ since $f_2(a) \in \w_\a$ and $\w_\a$ is a limit ordinal by Lemma~\ref{lem:aleph:initlimit}.

    Thus in all cases we have
    \ali{
      \abs{X} &= \abs{X_1} + \abs{X_2} \\
      &\leq \al_\g + \al_\g & \text{(by property (d) of in section~5.1)} \\
      &= \al_\g & \text{(by the induction hypothesis since $\g < \a$)} \\
      &< \ala \,.
    }
    Thus we have shown that $\abs{X} < \ala = \abs{\w_\a}$ for any $a \in A$, and hence $\abs{\w_\a} \not\leq \abs{X}$ by Corollary~\ref{cor:aleph:wonlegt} since $\w_\a$ and $X$ are both well-ordered.
    If $\d$ is the ordinal isomorphic to $(A, \prec)$ (which exists by Theorem~6.3.1 since we have shown that $\prec$ is a well-ordering), then it follows from the contrapositive of Lemma~\ref{lem:aleph:alephadd:setgt} that $\d \leq \w_\a$ and hence $\abs{A} = \abs{\d} \leq \abs{\w_\a} = \ala$.
    Thus we have
    $$
    \ala + \ala = \abs{A_1} + \abs{A_2} = \abs{A} \leq \ala \,.
    $$
    Since obviously $0 \leq \ala$ it follows again from property (d) in section~5.1 that
    $$
    \ala = \ala + 0 \leq \ala + \ala \,.
    $$
    Hence, by the Cantor-Bernstein Theorem we have that $\ala = \ala + \ala$, which completes the inductive step.
  }
}

\exercise{2}{
  Give a direct proof of $n \cdot \ala = \ala$ by constructing a one-to-one mapping of $\w_\a$ onto $n \times \w_\a$ (where $n$ is a positive natural number).
}
\sol{
  \begin{lem}\label{lem:aleph:natal:natlim}
    If $\a$ is a limit ordinal and $n$ is a natural number then $n \cdot \a = \a$.
  \end{lem}
  \qproof{
    First, clearly we have by definition that $n \cdot \w = \sup \braces{n \cdot k \where k < \w} = \w$.
    Then, since $\a$ is a limit ordinal, we have from Exercise~6.5.10 that $\a = \w \cdot \b$ for some ordinal $\beta$.
    Thus we have
    $$
    n \cdot \a = n \cdot (\w \cdot \b) = (n \cdot \w) \cdot \b = \w \cdot \b = \a \,.
    $$
  }

  \mainprob
  \qproof{
    Consider any natural number $n > 0$ and any ordinal number $\a$.
    We then show that $n \cdot \ala = \ala$ by constructing a bijective $f: \w_\a \to n \times \w_\a$, which clearly shows the result by the definition of cardinal multiplication.

    So consider any $\b \in \w_\a$.
    Then, since $n > 0$, we have by Theorem~6.6.3 that there is a unique ordinal $\g$ and unique natural number $k < n$ such that
    $$
    \b = n \cdot \g + k \,,
    $$
    where $k < n$ and clearly
    $$
    \g = 1 \cdot \g \leq n \cdot \g = n \cdot \g + 0 \leq n \cdot \g + k = \b
    $$
    since $1 \leq n$ and $0 \leq k$.
    Hence we have $k \in n$ and $\g \leq \b < \w_\a$ so that $\g \in \w_\a$.
    We then set $f(\b) = (k, \g) \in n \times \w_\a$.

    First we show that $f$ is injective.
    So consider $\b_1$ and $\b_2$ in $\w_\a$ where $f(\b_1) = f(\b_2)$.
    If we have $f(\b_1) = (k_1, \g_1)$ and $f(\b_2) = (k_2, \g_2)$ then clearly this means that $k_1 = k_2$ and $\g_1 = \g_2$.
    It then clearly follows that
    $$
    \b_1 = n \cdot \g_1 + k_1 = n \cdot \g_2 + k_2 = \b_2 \,,
    $$
    which shows that $f$ is injective.

    Now we show that $f$ is also surjective.
    So consider any $(k, \g) \in n \times \w_\a$ so that $k \in n$ and $\g \in \w_\a$.
    Then let $\b = n \cdot \g + k$ so that clearly $f(\b) = (k, \g)$.
    However, we must show that $\b$ is actually in $\w_\a$.
    To see this, first we note that since $\g < \w_\a$ is an ordinal we have $\g = \d + m$ for some limit ordinal $\d$ and natural number $m$ by Exercise~6.5.4, where clearly $\d \leq \g$.
    Hence we have
    $$
    \b = n \cdot \g + k = n \cdot (\d + m) + k = (n \cdot \d + n \cdot m) + k = n \cdot \d + (n \cdot m + k) = \d + (n \cdot m + k) \,,
    $$
    where $n \cdot \d = \d$ by Lemma~\ref{lem:aleph:natal:natlim} since $\d$ is a limit ordinal.
    Then since also $\d \leq \g < \w_\a$ and $n \cdot m + k$ is a natural number we clearly have that $\b = \d + (n \cdot m + k)< \w_\a$ as well since $\w_\a$ is a limit ordinal (by Theorem~7.1.9b).
    Hence $\b \in \w_\a$.

    Thus we have shown that $f$ is bijective so that by definition $n \cdot \ala = \abs{n \times \w_\a} = \abs{\w_\a} = \ala$ as desired.
  }
}

\newcommand\nss[1]{\squares{\ala}^{#1}}
\def\wss{\nss{<\w}}
\exercise{3}{
  Show that

  (a) $\ala^n = \ala$ for all positive natural numbers $n$.

  (b) $\abs{\nss{n}} = \ala$, where $\nss{n}$ is the set of all $n$-element subsets of $\ala$ for all $n > 0$.

  (c) $\abs{\wss} = \ala$, where $\wss$ is the set of all finite subsets of $\ala$.

  [Hint: Use Theorem~7.2.1 and induction; for (c), proceed as in the proof of Theorem~3.10 in Chapter~4, and use $\al_0 \cdot \ala = \ala$.]
}
\sol{

  (a)
  \qproof{
    For any ordinal $\a$, we show this by induction on $n$, noting that we only need to show this for positive $n$ so that $n \geq 1$ (in fact it is untrue for $n=0$).
    First, for $n=1$ we clearly have $\ala^n = \ala^1 = \ala$ by what was shown in Exercise~5.1.2.
    Now assume that $\ala^n = \ala$.
    We then have
    \ali{
      \ala^{n+1} &= \ala^n \cdot \ala^1 & \text{(by Theorem~5.1.7a)} \\
      &= \ala^n \cdot \ala & \text{(again by Exercise~5.1.2)} \\
      &= \ala \cdot \ala & \text{(by the induction hypothesis)} \\
      &= \ala \,. & \text{(by Theorem~7.2.1)}
    }
    This completes the induction step.
  }

  (b)
  \qproof{
    For ordinal $\a$ and natural number $n$, first we show that $\abs{\nss{n}} \leq \ala$ by constructing an injective $f : \nss{n} \to \w_\a^n$.
    For a set $X \in \nss{n}$ we have that $\abs{X} = n$.
    Thus there is a bijective $g$ from $n$ to $X$, and since clearly $X \ss \ala = \w_\a$ it follows that $g$ is a function from $n$ to $\w_\a$.
    Hence we simply set $f(X) = g$.
    Now consider any $X_1$ and $X_2$ in $\nss{n}$ where $X_1 \neq X_2$.
    Then let $g_1$ and $g_2$ be the corresponding bijections from $n$ to $X_1$ and $X_2$, respectively.
    Thus $f(X_1) = g_1$ and $f(X_2) = g_2$.
    Then, since clearly the range of $g_1$ is $X_1$, the range of $g_2$ is $X_2$, and $X_1 \neq X_2$ it follows that $f(X_1) = g_1 \neq g_2 = f(X_2)$, which shows that $f$ is injective.
    Hence it follows that $\abs{\nss{n}} \leq \abs{\w_\a^n} = \abs{\w_\a}^{\abs{n}} = \ala^n = \ala$ by what was shown in part (a).

    Now we show that also $\ala \leq \abs{\nss{n}}$ by constructing an injective $f : \w_\a \to \nss{n}$.
    So for any $\b \in \w_\a$ let $X = \braces{\b + k \where k \in n}$, noting that clearly $X \ss \w_\a = \ala$ since $\w_\a$ is a limit ordinal (since then $\b + k \in \w_\a$ for any natural number $k$).
    Also clearly $\abs{X} = n$ so that $X \in \nss{n}$.
    We then set $f(\b) = X$.
    Now consider any $\b_1$ and $\b_2$ in $\w_\a$ where $\b_1 \neq \b_2$ and let $X_1 = \braces{\b_1 + k \where k \in n}$ and $X_2 = \braces{\b_2 + k \where k \in n}$ so that $f(\b_1) = X_1$ and $f(\b_2) = X_2$.
    Since $\b_1 \neq \b_2$ we can assume that $\b_1 < \b_2$ without loss of generality.
    Now, clearly $\b_1 = \b_1 + 0$ is the least element of $X_1$ and $\b_2 = \b_2 + 0$ the least element of $X_2$.
    Since $\b_1 < \b_2$ it then follows that $\b_1 \notin X_2$, but since $\b_1 \in X_1$ this clearly implies that $f(\b_1) = X_1 \neq X_2 = f(\b_2)$.
    This shows that $f$ is injective so that $\ala = \abs{\w_\a} \leq \abs{\nss{n}}$.

    Since we have shown that both $\abs{\nss{n}} \leq \ala$ and $\ala \leq \abs{\nss{n}}$, it follows from the Cantor-Bernstein Theorem that $\abs{\nss{n}} = \ala$, which is what we intended to show.
  }

  (c)
  \qproof{
    For any ordinal $\a$ we show first note that clearly $\wss = \bigcup_{n < \w} \nss{n}$.
    We show that $\abs{\wss} = \ala$ by constructing a bijective $f : \w \times \w_\a \to \bigcup_{n < \w} \nss{n}$.
    So consider any $n \in \w$ and $\b \in \w_\a$.
    Now, by what was shown in part (b), we have $\abs{\nss{n}} = \ala = \abs{\w_\a}$ so that there is a bijective $g_n : \w_\a \to \nss{n}$, i.e. $g_n$ is a transfinite enumeration of $\nss{n}$.
    We then set $f(n, \b) = g_n(\b)$, from which it should be clear that $f(n,\b) \in \bigcup_{k < \w} \nss{k}$ since $f(n, \b) = g_n(\b) \in \nss{n}$.

    First we show that $f$ is injective.
    To this end consider any $(n_1, \b_1)$ and $(n_2, \b_2)$ in $\w \times \w_\a$ where $(n_1, \b_1) \neq (n_2, \b_2)$.
    Then either $n_1 \neq n_2$ or $\b_1 \neq \b_2$ (or both).
    Let $g_{n_1}$ and $g_{n_2}$ be the corresponding bijections from $\w_\a$ to $\nss{n_1}$ and $\nss{n_2}$, respectively, as described above.
    Clearly if $n_1 \neq n_2$ then $f(n_1, \b_1) = g_{n_1}(\b_1) \in \nss{n_1}$ whereas $f(n_2, \b_2) = g_{n_2}(\b_2) \in \nss{n_2}$ so that $f(n_1, \b_1) \neq f(n_2, \b_2)$ since $\nss{n_1}$ and $\nss{n_2}$ are clearly disjoint (since $\nss{n_1}$ contains only sets with $n_1$ elements and $\nss{n_2}$ contains only sets with $n_2$ elements and $n_1 \neq n_2$).
    On the other hand, if $n_1 = n_2$ then it must be the case that $\b_1 \neq \b_2$.
    It also follows that $g_{n_1} = g_{n_2}$ since $n_1 = n_2$.
    Hence we have $f(n_1, \b_1) = g_{n_1}(\b_1) \neq g_{n_1}(\b_2) = g_{n_2}(\b_2) = f(n_2, \b_2)$ since $g_{n_1} = g_{n_2}$ is injective and $\b_1 \neq \b_2$.
    Thus in any case $f(n_1, \b_1) \neq f(n_2, \b_2)$ so that $f$ is injective.

    Next we show that $f$ is surjective.
    So consider any $X \in \bigcup_{n < \w} \nss{n}$ so that there is an $n < \w$ such that $X \in \nss{n}$.
    Then let $\b = \inv{g_n}(X)$ (where $g_n$ is the bijection from $\w_\a$ to $\nss{n}$ as described above).
    Then clearly $(n, \b) \in \w \times \w_\a$ and we have $f(n, \b) = g_n(\b) = g_n(\inv{g_n}(X)) = X$.
    Since $X$ was arbitrary this shows that $f$ is surjective.

    Hence $f$ is a bijection so that $\abs{\wss} = \abs{\bigcup_{n < \w} \nss{n}} = \abs{\w \times \w_\a} = \abs{\w} \cdot \abs{\w_\a} = \al_0 \cdot \ala = \ala$ by Corollary~7.2.2 since clearly $0 \leq \a$.
    This shows the desired result.
  }
}

\def\alg{\al_\g}
\exercise{4}{
  If $\a$ and $\b$ are ordinals and $\abs{\a} \leq \alg$ and $\abs{\b} \leq \alg$, then $\abs{\a + \b} \leq \alg$, $\abs{\a \cdot \b} \leq \alg$, $\abs{\a^\b} \leq \alg$ (where $\a + \b$, $\a \cdot \b$, and $\a^\b$ are \emph{ordinal} operations).
}
\sol{
  \def\maxab{\max\parens{\abs{\a}, \abs{\b}}}
  \begin{lem}\label{lem:aleph:orcarar:add}
    If $\a$ and $\b$ are ordinals then $\abs{\a + \b} = \abs{\a} + \abs{\b}$.
  \end{lem}
  \qproof{
    Suppose that $A$ and $B$ are disjoint sets where $\abs{A} = \abs{\a}$ and $\abs{B} = \abs{\b}$ so that, by the definition of cardinal addition, $\abs{\a} + \abs{\b} = \abs{A \cup B}$.
    We then show the result by constructing a bijection $f$ from $A \cup B$ to the ordinal $\a + \b$.
    First, since $\abs{A} = \abs{\a}$, there is a bijective $f_A : A \to \a$.
    Similarly there is a bijective $f_B : B \to \b$ since $\abs{B} = \abs{\b}$.
    Now consider any $x \in A \cup B$.
    We then set
    $$
    f(x) = \begin{cases}
      f_A(x) & x \in A \\
      \a + f_B(x) & x \in B
    \end{cases} \,,
    $$
    noting that this is unambiguous since $A$ and $B$ are disjoint.
    If $x \in A$ then we clearly have $f(x) = f_A(x) < \a = \a + 0 \leq \a + \b$ by Lemma~6.5.4 so that $f(x) \in \a + \b$.
    On the other hand if $x \in B$ then $f(x) = \a + f_B(x) < \a + \b$ again by Lemma~6.5.4 since $f_B(x) < \b$, and hence again $f(x) \in \a + \b$.
    This shows that $f$ really is a function into $\a + \b$.

    Next we show that $f$ is injective.
    So consider any $x$ and $y$ in $A \cup B$ where $x \neq y$.

    Case: $x$ and $y$ are both in $A$.
    Then $f(x) = f_A(x) \neq f_A(y) = f(y)$ since $f_A$ is injective and $x \neq y$.

    Case: $x \in A$ and $y \in B$.
    Then $f(x) = f_A(x) < \a = \a + 0 \leq \a + f_B(y) = f(y)$ by Lemma~6.5.4 so that clearly $f(x) \neq f(y)$.

    Case: $x \in B$ and $y \in A$.
    This is analogous to the previous case.

    Case: $x$ and $y$ are both in $A$.
    Then $f(x) = \a + f_B(x) \neq \a + f_B(y) = f(y)$ by Lemma~6.5.4b since $f_B$ is injective and $x \neq y$ so that $f_B(x) \neq f_B(y)$.

    Hence in all cases we have $f(x) \neq f(y)$, which shows that $f$ is injective.

    Lastly, we show that $f$ is surjective.
    So consider any $y$ in $\a + \b$.
    If $y < \a$ then $y \in \a$.
    Since $f_A$ is surjective there is an $x \in A$ such that $f_A(x) = y$.
    Then, since $x \in A$, clearly $f(x) = f_A(x) = y$.
    If $y \geq a$ then by Lemma~6.5.5 there is an ordinal $\x$ such that $\a + \x = y$.
    It then follows that $\a + \x = y < \a + \b$ so that $\x < \b$ by Lemma~6.5.4a.
    Hence $\x \in \b$ so that there is an $x \in B$ such that $f_B(x) = \x$ since $f_B$ is surjective.
    Then, since $x \in B$,  clearly we have $f(x) = \a + f_B(x) = \a + \x = y$.
    Hence in both cases there is an $x \in A \cup B$ where $f(x) = y$.
    Since $y$ was arbitrary, this shows that $f$ is surjective.

    Thus we have shown that $f$ is bijective so that $\abs{\a} + \abs{\b} = \abs{A \cup B} = \abs{\a + \b}$.
  }

  \begin{lem}\label{lem:aleph:orcarar:mult}
    If $\a$ and $\b$ are ordinals then $\abs{\a \cdot \b} = \abs{\a} \cdot \abs{\b}$.
  \end{lem}
  \qproof{
    First, if $\a = 0$ then
    $$
    \abs{\a \cdot \b} = \abs{0 \cdot \b} = \abs{0} = 0 = 0 \cdot \abs{\b} = \abs{0} \cdot \abs{\b} = \abs{\a} \cdot \abs{\b} \,.
    $$
    The result also holds when $\b = 0$.
    Hence going forward we can assume that $\a$ and $\b$ are nonzero and therefore nonempty.

    We then show the result by constructing a bijective $f : \a \times \b \to \a \cdot \b$.
    So consider any $(\g, \d) \in \a \times \b$.
    It then follows that $\g < \a$ and $\d < \b$ so that $\d+1 \leq \b$.
    We then set $f(\g, \d) = \a \cdot \d + \g$.
    We note that
    $$
    f(\g, \d) = \a \cdot \d + \g < \a \cdot \d + \a = \a \cdot \d + \a \cdot 1 = \a \cdot (\d + 1) \leq \a \cdot \b
    $$
    by Lemma~6.5.4, Exercise~6.5.2, and Exercise~6.5.7.
    Hence $f(\g, \d) \in \a \cdot \b$ so that $f$ is into $\a \cdot \b$.

    Now we show that $f$ is injective.
    So consider any $(\g_1, \d_1)$ and $(\g_2, \d_2)$ in $\a \times \b$ where $(\g_1, \d_1) \neq (\g_2, \d_2)$.
    Then clearly we have $\g_1 < \a$, $\g_2 < \a$, $\d_1 < \b$, and $\d_2 < \b$.
    We then have

    Case: $\d_1 = \d_2$.
    Then it must be that $\g_1 \neq \g_2$.
    We then have
    $$
    f(\g_1, \d_1) = \a \cdot \d_1 + \g_1 \neq \a \cdot \d_1 + \g_2 = \a \cdot \d_2 + \g_2 = f(\g_2, \d_2)\,,
    $$
    where we have used Lemma~6.5.4b and Exercise~6.5.7b since $\a \neq 0$.
    
    Case: $\d_1 \neq \d_2$.
    Without loss of generality we can assume that $\d_1 < \d_2$ so that clearly $\d_1 + 1 \leq \d_2$.
    Then we have
    $$
    f(\g_1, \g_2) = \a \cdot \d_1 + \g_1 < \a \cdot \d_1 + \a = \a \cdot (\d_1 + 1) \leq \a \cdot \d_2 \leq \a \cdot \d_2 + \g_2 = f(\g_2, \d_2) \,,
    $$
    where we have again used Lemma~6.5.4a, Exercise~6.5.2, and Exercise~6.5.7.

    Hence in all cases we have $f(\g_1, \d_1) \neq f(\g_2, \d_2)$, which shows that $f$ is injective.

    Lastly, we show that $f$ is surjective.
    So consider any ordinal $\x \in \a \cdot \b$ so that $\x < \a \cdot \b$.
    Since $\a \neq 0$, it follows from Theorem~6.6.3 that there is a unique $\d$ and unique $\g < \a$ such that $\x = \a \cdot \d + \g$.
    Note that we have $\d < \b$ since otherwise $\d \geq \b$ would imply that
    $$
    \x = \a \cdot \d + \g \geq \a \cdot \d \geq \a \cdot \b
    $$
    by Lemma~6.5.4 and Exercise~6.5.7, which is impossible since $\x < \a \cdot \b$.
    Hence we have that $(\g, \d) \in \a \times \b$, and clearly $f(\g, \d) = \x$.
    Since $\x$ was arbitrary this shows that $f$ is surjective.

    Thus we have shown that $f$ is bijective so that by the definition of cardinal multiplication we have $\abs{\a \cdot \b} = \abs{\a \times \b} = \abs{\a} \cdot \abs{\b}$ as desired.
  }

  The following two lemmas are straightforward generalizations of Theorems~4.3.9 and 4.3.10, respectively.

  \def\tfun{\bigcup_{\g < \a} A_\g}
  \begin{lem}\label{lem:aleph:orcarar:union}
    Consider a nonzero ordinals $\a$ and $\b$.
    Let $\angles{A_\g \where \g < \a}$ be a (potentially transfinite) system of at most $\abs{\b}$ sets, and let $\angles{a_\g \where \g < \a}$ be a system of enumerations for $\angles{A_\g \where \g <\a}$, i.e., for each $\g < \a$, $a_\g = \angles{a_\g(\d) \where \d < \b}$ is a (potentially transfinite) sequence, and $A_\g = \braces{a_\g(\d) \where \d < \b}$.
    Then $\tfun$ is at most $\abs{\a} \cdot \abs{\b}$.
  \end{lem}
  \qproof{
    We define a function $f : \a \times \b \to \tfun$ by simply setting $f(\g, \d) = a_\g(\d)$ for any $\g \in \a$ and $\d \in \b$.
    We show that $f$ is onto by considering any $x \in \tfun$ so that there is an $\g < \a$ such that $x \in A_\g$.
    Then, since $A_\g$ is the range of $a_\g$, there is a $\d < \b$ such that $a_\g(\d) = x$.
    Hence $f(\g, \d) = a_\g(\d) = x$ so that $f$ is onto since $x$ was arbitrary.

    We also have that $\a \times \b$ is well-orderable by Theorem~6.5.8 (for example the lexicographic ordering has order type $\b \cdot \a$).
    It then follows from Lemma~\ref{lem:aleph:hs:onto} that $\abs{\tfun} \leq \abs{\a \times b} = \abs{\a} \cdot \abs{\b}$ since $f$ is onto.
    Hence $\tfun$ is at most $\abs{\a} \cdot \abs{\b}$ as desired.
  }

  \begin{lem}\label{lem:aleph:orcarar:seqinf}
    If $A$ is a set with cardinality $\al_\g$ for some ordinal $\g$, then the set $\Seq(A)$ of all finite sequences of elements of $A$ also has cardinality $\al_\g$.
  \end{lem}
  \qproof{
    Let $f$ be a bijection from $\w_\g$ to $A$.
    We also know from Theorem~7.2.1 that $\abs{\w_\g \times \w_\g} = \al_\g \cdot \al_\g = \al_\g$, so let $g$ be a bijection from $\w_\g$ to $\w_\g \times \w_\g$.
    For each $n < \w_0 = \w$ (i.e. $n \in \nats$) we define a transfinite enumeration $\angles{a_n(\a) \where \a < \w_\g}$ of $A^n$, where $A^n$ of course denotes the set of all sequences of elements of $A$ of length $n$.
    We define these enumerations recursively.
    Clearly for $n=0$ we have $A^n = A^0 = \braces{\es}$ so that we can set
    \ali{
      a_0(\a) &= \es \text{ for any $\a < \w_\g$} \\
      a_1(\a) &= \angles{f(\a)} \text{ for any $\a < \w_\g$}
    }
    Then, having defined $a_n$, for any $\a < \w_\g$, we let $g(\a) = (\a_1, \a_2)$ and define the sequence $a_{n+1}(\a)$ of length $n$ as follows:
    $$
    a_{n+1}(\a)(k) = \begin{cases}
      a_n(\a_1)(k) & k < n \\
      f(\a_2) & k = n \,.
    \end{cases}
    $$
    for any $k < n+1$.

    Clearly each $a_n(\a)$ is a sequence of length $n$ for any $n < \w_0$ and $\a < \w_\g$, but we must show that $a_n$ is in fact an enumeration by showing that it is onto $A^n$ for all $n < \w_0$.
    We show this by induction on $n$.
    Obviously this is true for the trivial $n=0$ case and for the $n=1$ case as well since, for any sequence $\angles{a}$ of length 1 where $a \in A$, we have that there is a an $\a < \w_\g$ such that $f(\a) = a$ since $f$ is onto.
    Hence by definition $a_1(\a) = \angles{f(\a)} = \angles{a}$ so that $a_1$ is onto.

    Now suppose that $a_n$ is onto $A^n$ and consider any sequence $h \in A^{n+1}$.
    Now, since $h \rest n$ is a sequence of length $n$ there is an $\a_1 < \w_\g$ such that $a_n(\a_1) = h \rest n$ by the induction hypothesis.
    We also have $h(n) \in A$ so that there is an $\a_2 < \w_\g$ such that $f(\a_2) = h(n)$ since again $f$ is onto.
    Now, clearly $(\a_1, \a_2) \in \w_\g \times \w_\g$ so that there is an $\a < \w_\g$ such that $g(\a) = (\a_1, \a_2)$ since $g$ is onto.
    Now consider any $k < n+1$.
    If $k < n$ then clearly we have $a_{n+1}(\a)(k) = a_n(\a_1)(k) = h(k)$ since $a_n(\a_1) = h \rest n$.
    On the other hand, if $k = n$, then by definition $a_{n+1}(\a)(k) = f(\a_2) = h(n) = h(k)$.
    Thus $a_{n+1}(\a) = h$ since $k$ was arbitrary and the cases are exhaustive.
    This shows that $a_{n+1}$ is onto since $h$ was arbitrary, hence it is an enumeration.

    Clearly we have that $\Seq(A) = \bigcup_{n < \w_0} A^n$ and, since we also have an transfinite enumeration (indexed by $\w_\g$) of each $A^n$ it follows from Lemma~\ref{lem:aleph:orcarar:union} that
    $$
    \abs{\Seq(A)} = \abs{\bigcup_{n < \w_0} A^n} \leq \abs{\w_0} \cdot \abs{\w_\g} = \al_0 \cdot \al_\g = \al_\g \,,
    $$
    where we have utilized Corollary~7.2.2 since $0 \leq \g$.
    Also clearly $\al_\g \leq \abs{\Seq(A)}$ since, for example, the function $f: \w_\g \to \Seq(A)$ defined by $f(\a) = \angles{\a}$ (for any $\a < \w_\g$) is injective.
    Thus by the Cantor-Bernstein Theorem we have $\abs{\Seq(A)} = \al_\g$ as desired.
  }

  \begin{cor}\label{cor:aleph:orcarar:seqfin}
    If $A$ is a nonempty finite set then the set $\Seq(A)$ of all finite sequences of elements of $A$ is countable.
  \end{cor}
  \qproof{
    First we note that clearly the set $B \cup \w$ is countable (since $B$ is finite and $\w$ is countable) so that $\Seq(A \cup \w)$ is also countable by Lemma~\ref{lem:aleph:orcarar:seqinf}.
    Now let $f$ be a function from $\Seq(A)$ to $\Seq(A \cup \w)$ defined by the identity $f(g) = g$ for any sequence $g \in \Seq(A)$.
    Note that $A \ss A \cup \w$ so that any sequence of elements of $A$ is a also a sequence with elements in $A \cup \w$.
    Clearly $f$ is injective so that $\abs{\Seq(A)} \leq \abs{\Seq(A \cup \w)} = \al_0$.

    Since $A$ is nonempty there is an $a \in A$.
    For any $n < \w$ consider the finite sequence $g_n(k) = a$ for any $k < n$, which is clearly a sequence of length $n$ with elements in $A$.
    Consider then the function $f$ from $\w$ to $\Seq(A)$ defined $f(n) = g_n$ for $n < \w$.
    Clearly this is a injective function since, for any $n_1$ and $n_2$ in $\w$ where $n_1 \neq n_2$, we have that $f(n_1) = g_{n_1}$ and $f(n_2) = g_{n_2}$ are sequences of different lengths so cannot be equal.
    Thus we have $\al_0 = \abs{\w} \leq \abs{\Seq(a)}$ as well so that $\abs{\Seq(A)} = \al_0$ by the Cantor-Bernstein Theorem.
  }

  \def\afs{\squares{A}^{< \w}}
  \def\nfs{\squares{\al_\a}^{< \w}}
  The following is a corollary to Exercise~7.2.3c.
  \begin{cor}\label{cor:aleph:orcarar:fss}
    Suppose that $A$ is a set such that $\abs{A} = \al_\a$ for some ordinal $\a$.
    Let $\afs$ denote the set of all finite subsets of $A$.
    Then $\abs{\afs} = \al_\a$.
  \end{cor}
  \qproof{
    First let $\nfs$ again denote the set of finite subsets of $\al_\a$.
    Since $\abs{A} = \al_\a$ there is a bijective $f: A \to \al_\a$.
    We then define $g : \afs \to \nfs$ by letting $g(B) = f[B]$ for any $B \in \afs$, noting that clearly $g(B) \ss \al_\a$ and $g(B)$ is finite so that $g(B) \in \nfs$.
    We now show that $g$ is bijective.

    First consider any $B_1$ and $B_2$ in $\afs$ where $g(B_1) = g(B_2)$.
    Hence $f[B_1] = g(B_1) = g(B_2) = f[B_2]$.
    So consider any $x \in B_1$ so that $f(x) \in f[B_1]$.
    Then also $f(x) \in f[B_2]$ so that there is a $y \in B_2$ such that $f(y) = f(x)$.
    But since $f$ is injective it has to be that $y = x$ so that $x = y \in B_2$.
    Hence $B_1 \ss B_2$ since $x$ was arbitrary.
    A similar argument shows that $B_2 \ss B_1$ as well so that $B_1 = B_2$.
    This shows that $g$ is injective.

    Now consider any $C \in \nfs$ and let $B = \inv{f}[C]$, noting that $\inv{f}$ is a bijective function since $f$ is.
    We claim then that $g(B) = C$.
    So consider any $y \in g(B) = f[B]$ so that there is a $x \in B$ such that $f(x) = y$.
    Then we have $x \in \inv{f}[C]$ by the definition of $B$ so that there is a $z \in C$ such that $\inv{f}(z) = x$.
    Hence $z = f(\inv{f}(z)) = f(x) = y$ so that $y = z \in C$.
    Thus $g(B) \ss C$ since $y$ was arbitrary.
    Now consider any $y \in C$ so that clearly $x =\inv{f}(y) \in \inv{f}[C] = B$ and also $f(x) = y$.
    Hence clearly $y = f(x) \in f[B] = g(B)$ so that $C \ss g(B)$ since $y$ was arbitrary.
    This shows that $g(B) = C$ so that $g$ is surjective since $C$ was arbitrary.

    We have just shown that $g$ is bijective so that $\abs{\afs} = \abs{\nfs} = \al_\a$ by Exercise~7.2.3c.
  }

  \def\bfs{\squares{\b}^{< \w}}
  \def\sba{S(\b, \a)}
  \begin{lem}\label{lem:aleph:orcarar:exp}
    If $\a$ and $\b$ are ordinals where at least one is infinite then $\abs{\a^\b} \leq \maxab$.
  \end{lem}
  \qproof{
    To show this we reference the representation of ordinal exponentiation discussed in Exercise~6.5.16.
    In that exercise we showed that the set $S(\b,\a) = \braces{f \where f: \b \to \a \text{ and } s(f) \text{ is finite}}$, where $s(f) = \braces{\x < \b \where f(\x) \neq 0}$ for any $f: \b \to \a$, can be ordered to be isomorphic to $\a^\b$.
    From this it clearly follows that $\abs{S(\b,\a)} = \abs{\a^\b}$.

    We now construct an injective function $f$ from $\sba$ to $\bfs \times \Seq(\a)$, where $\bfs$ is the set of all finite subsets of $\b$ and $\Seq(\a)$ is the set of all finite sequences of elements of $\a$.
    So consider any $g \in \sba$ so that $g: \b \to \a$ and $s(g)$ is a finite subset of $\b$.
    Also clearly $s(g)$ is a finite set of ordinals so that there is a unique isomorphism $h$ from some natural number $n$ to $s(g)$.
    Clearly then $g \circ h$ is a finite sequence from $n$ to $\a$.
    Thus we have that $s(g) \in \bfs$ and $g \circ h \in \Seq(\a)$, so we set $f(g) = (s(g), g \circ h)$.

    To see that this mapping is injective consider $g_1$ and $g_2$ in $\sba$ where $g_1 \neq g_2$.
    Then there is some $\x < \b$ where $g_1(\x) \neq g_2(\x)$.
    Let $h_1$ and $h_2$ be the isomorphisms from natural numbers $n_1$ and $n_2$ to $s(g_1)$ and $s(g_2)$, respectively, as described above.
    If $s(g_1) \neq s(g_2)$ then clearly $f(g_1) = (s(g_1), g_1 \circ h_1) \neq (s(g_2), g_2 \circ h_2) = f(g_2)$.
    So assume that $s(g_1) = s(g_2)$, from which it follows that $n_1 = n_2$ and $h_1 = h_2$.
    Since $h_1 = h_2$ are bijections there is a $k \in n_1 = n_2$ such that $h_1(k) = h_2(k) = \x$, noting that it has to be that $\x \in s(g_1) = s(g_2)$ since otherwise we would have $g_1(\x) = 0 = g_2(\x)$.
    We then have $(g_1 \circ h_1)(k) = g_1(h_1(k)) = g_1(\x) \neq g_2(\x) = g_2(h_2(k)) = (g_2 \circ h_2)(k)$ so that $g_1 \circ h_1 \neq g_2 \circ h_2$.
    Thus once again $f(g_1) = (s(g_1), g_1 \circ h_1) \neq (s(g_2), g_2 \circ h_2) = f(g_2)$, which shows that $f$ is injective.

    So since $f$ is injective it follows that $\abs{\a^\b} = \abs{\sba} \leq \abs{\bfs \times \Seq(\a)} = \abs{\bfs} \cdot \abs{\Seq(\a)}$.
    Suppose first that $\abs{\a} \leq \abs{\b}$ so it has to be that $\b$ is infinite so that $\maxab = \abs{\b} = \al_\g$ for some ordinal $\g$.
    Thus by Lemma~\ref{cor:aleph:orcarar:fss} we have $\abs{\bfs} = \al_\g$.
    If $\a = 0 = \es$ then clearly $\Seq(\a) = \braces{\es}$ so that $\abs{\Seq(\a)} = 1$.
    If $\a$ is finite but nonzero then it is nonempty so that $\abs{\Seq(\a)} = \al_0$ by Corollary~\ref{cor:aleph:orcarar:seqfin}.       Lastly, if $\a$ is infinite then $\abs{\a} = \al_\d$ for some $\d \leq \g$ since $\al_\d = \abs{\a} \leq \abs{\b} = \al_\g$.
    Hence $\abs{\Seq(\a)} = \al_\g$ by Lemma~\ref{lem:aleph:orcarar:seqinf}.
    Thus in all three cases $\k = \abs{\Seq(\a)}$ is either a natural number or $\al_\d$ for some $\d \leq \g$.
    It then follows from Corollary~7.2.2 that
    $$
    \abs{\a^\b} \leq \abs{\bfs} \cdot \abs{\Seq(\a)} = \al_\g \cdot \k = \k \cdot \al_\g = \al_\g = \maxab \,.
    $$
    On the other hand if $\abs{\b} \leq \abs{\a}$ then it must be that $\a$ is infinite so that $\maxab = \abs{\a} = \al_\g$ for some ordinal $\g$.
    Thus by Lemma~\ref{lem:aleph:orcarar:seqinf} we have $\abs{\Seq(\a)} = \al_\g$ as well.
    If $\b$ is finite then clearly every subset of $\b$ is finite so that $\bfs = \pset{\b}$ is finite by Theorem~4.2.8.
    Hence $\abs{\bfs} = n$ for some natural number $n$.
    If $\b$ is infinite then $\abs{\b} = \al_\d$ for some $\d \leq \g$ since $\al_\d = \abs{\b} \leq \abs{\a} = \al_\g$.
    We then have that $\abs{\bfs} = \al_\d$ as well by Lemma~\ref{cor:aleph:orcarar:fss}.
    Hence in either case $\k = \abs{\bfs}$ is a natural number or $\al_\d$ for some $\d \leq \g$.
    Hence we have
    $$
    \abs{\a^\b} \leq \abs{\bfs} \cdot \abs{\Seq(\a)} = \k \cdot \al_\g = \al_\g = \maxab
    $$
    again by Corollary~7.2.2.
    Thus in all cases we have shown that $\abs{\a^\b} \leq \maxab$ as desired.
  }

  \mainprob
  \qproof{
    That $\abs{\a + \b} \leq \al_\g$ follows almost immediately from Lemma~\ref{lem:aleph:orcarar:add}.
    We have that $\abs{\a + \b} = \abs{\a} + \abs{\b} \leq \al_\g + \al_\g = \al_\g$, where we have also used property (c) of cardinal numbers after Lemma~5.1.2, and Corollary~7.2.3.

    Similarly, $\abs{\a \cdot \b} \leq \al_\g$ follows from Lemma~\ref{lem:aleph:orcarar:mult}.
    We have that $\abs{\a \cdot \b} = \abs{\a} \cdot \abs{\b} \leq \al_\g \cdot \al_\g = \al_\g$, where we have used property (i) of cardinal numbers following Lemma~5.1.4, and Theorem~7.2.1.

    The analogous lemma for ordinal exponentiation (i.e. that $\abs{\a^\b} = \abs{\a}^{\abs{\b}}$ for ordinals $\a$ and $\b$) is evidently not true.
    As a counterexample consider $\a = 2$ and $\b = \w$.
    We then have that $\abs{\a^\b} = \abs{2^\w} = \abs{\w} = \al_0$ is countable whereas we know that $\abs{\a}^{\abs{\b}} = \abs{2}^{\abs{\w}} = 2^{\al_0}$ is uncountable.

    However, the somewhat analogous Lemma~\ref{lem:aleph:orcarar:exp} will help us show the desired result.
    First, if both $\a$ and $\b$ are finite then clearly $\a^\b$ is also finite so that clearly $\abs{\a^\b} \leq \al_\g$.
    On the other hand, if at least one of $\a$ or $\b$ is infinite, then have we have that $\abs{\a^\b} \leq \maxab \leq \al_\g$ by Lemma~\ref{lem:aleph:orcarar:exp} as desired, noting that clearly $\maxab \leq \al_\g$ since both $\abs{\a} \leq \al_\g$ and $\abs{\b} \leq \al_\g$.
  }
}

\exercise{5}{
  If $X$ is the image of $\w_\a$ by some function $f$, then $\abs{X} \leq \ala$.
  [Hint: Construct a one-to-one mapping $g$ of $X$ into $\w_\a$ by letting $g(x) =$ the least element of the inverse image of $\braces{x}$ by $f$.]
}
\sol{
  \qproof{
    Clearly $f$ is a function from $\w_\a$ onto its image $X$ so that it follows from Lemma~\ref{lem:aleph:hs:onto} that $\abs{X} \leq \abs{\w_\a} = \al_\a$ as desired.
  }
  
  Note that the proof of Lemma~\ref{lem:aleph:hs:onto} uses exactly the technique given in the hint to argue its conclusion.
}

\exercise{6}{
  If $X$ is a subset of $\w_\a$ such that $\abs{X} < \ala$, then $\abs{\w_\a - X} = \ala$.
}
\sol{
  \begin{lem}\label{lem:aleph:minus:add}
    If $\k$ and $\l$ are cardinal numbers and $\k \leq \l < \al_\a$ for some ordinal $\a$, then $\k + \l < \al_\a$.
  \end{lem}
  \qproof{
    We have
    \ali{
      \k + \l &= \l + \k & \text{(by the commutativity of cardinal addition)} \\
      &\leq \l + \l & \text{(by property~(d) in section~5.1)} \\
      &= \l \cdot 1 + \l \cdot 1 \\
      &= \l \cdot (1 + 1) & \text{(by property~(g) in section~5.1)} \\
      &= \l \cdot 2 \,.
    }
    If $\l$ is finite then clearly $\l \cdot 2$ is finite so that $\k + \l \leq \l \cdot 2 < \al_\a$.
    On the other hand if $\l$ is infinite then $\l = \al_\b$ where $\b < \a$ since $\al_\b = \l < \al_\g$.
    Thus we have $\k + \l \leq \l \cdot 2 = 2 \cdot \l = 2 \cdot \al_\b = \al_\b < \al_\a$ by Corollary~7.2.2.
    Hence in either case $\k + \l < \al_\a$ as desired.
  }

  \mainprob
  \qproof{
    First, we clearly have that $\w_\a - X$ and $X$ are disjoint and $(\w_\a - X) \cup X = \w_\a$ so that, by the definition of cardinal addition, we have
    $$
    \abs{\w_\a - X} + \abs{X} = \abs{(\w_\a - X) \cup X} = \abs{\w_\a} = \al_\a \,.
    $$
    Now suppose that $\abs{\w_\a - X} < \al_\a$
    Since $\w_\a - X \ss \w_\a$ and $X \ss \w_\a$ are both sets of ordinals, they are clearly both well ordered by $<$.
    Thus $\abs{\w_\a - X} \leq \abs{X}$ or $\abs{X} \leq \abs{\w_\a - X}$ by Lemma~\ref{lem:aleph:wolege}.
    Since we also have $\abs{\w_\a - X} < \al_\a$ and $\abs{X} < \al_\a$, in either case it follows from Lemma~\ref{lem:aleph:minus:add} that
    $$
    \al_\a = \abs{\w_\a - X} + \abs{X} < \al_\a \,,
    $$
    which is clearly a contradiction.
    Thus it must be that $\abs{\w_\a - X} \nless \al_\a$ so that $\al_\a \leq \abs{\w_\a - X}$ by Corollary~\ref{cor:aleph:wonlegt}.
    Since $\w_\a - X \ss \w_\a$ we clearly also have that $\abs{\w_\a - X} \leq \abs{\w_\a} = \al_\a$.
    Hence $\abs{\w_\a - X} = \al_\a$ by the Cantor-Bernstein Theorem.
  }
}
