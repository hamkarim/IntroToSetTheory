\subsection{Cardinal Arithmetic}

\exercise{1}{
  Prove properties (a)-(n) of cardinal arithmetic stated in the text of this section.
  These are

  (a) $\k + \l = \l + \k$

  (b) $\k + (\l + \mu) = (\k + \l) + \mu$

  (c) $\k \leq \k + \l$

  (d) If $\k_1 \leq \k_2$ and $\l_1 \leq \l_2$, then $\k_1 + \l_1 \leq \k_2 + \l_2$

  (e) $\k \cdot \l = \l \cdot \k$

  (f) $\k \cdot (\l \cdot \mu) = (\k \cdot \l) \cdot \mu$

  (g) $\k \cdot (\l + \mu) = \k \cdot \l + \k \cdot \mu$

  (h) $\k \leq \k \cdot \l$ if $\l > 0$

  (i) If $\k_1 \leq \k_2$ and $\l_1 \leq \l_2$, then $\k_1 \cdot \l_1 \leq \k_2 \cdot \l_2$

  (j) $\k + \k = 2 \cdot \k$

  (k) $\k + \k \leq \k \cdot \k$, whenever $\k \geq 2$

  (l) $\k \leq \k^\l$ if $\l > 0$

  (m) $\l \leq \k^\l$ if $\k > 1$

  (n) If $\k_1 \leq \k_2$ and $\l_1 \leq \l_2$, then $\k_1^{\l_1} \leq \k_2^{\l_1}$
}
\sol{
  For solutions (a) through (c) suppose that
  \ali{
    \k &= |K|  &
    \l &= |L| &
    \mu &= |M| \,,
  }
  where $K$, $L$, and $M$ are mutually disjoint sets.

  (a)
  \qproof{
    It is obvious that
    $$
    \k + \l = |K \cup L| = |L \cup K| = \l + \k
    $$
    since $K \cup L = L \cup K$ and $K$ and $L$ are disjoint.
  }

  (b)
  \qproof{
    First we note that clearly
    $$
    K \cup (L \cup M) = K \cup L \cup M = (K \cup L) \cup M \,.
    $$
    Now suppose that there is an $x \in K \cap (L \cup M)$ so that $x \in K$ and $x \in L \cup M$
    If $x \in L$ then $x \in K \cap L$ and if $x \in M$ then $x \in K \cap M$, either of which is a contradiction since all three sets are mutually disjoint.
    Hence $K$ and $L \cup M$ are disjoint.
    A similar argument show that $K \cup L$ and $M$ are disjoint.
    Thus we have the following:
    $$
    \k + (\l + \mu) = |K| + |L \cup M| = |K \cup (L \cup M)| = |(K \cup L) \cup M| = |K \cup L| + |M| = (\k + \l) + \mu
    $$
    as desired.
  }

  (c)
  \qproof{
    Define the function $f: K \to K \cup L$ by simply the identity $f(k) = k$ for any $k \in K$.
    Obviously this is an injective function so that $\k = |K| \leq |K \cup L| = \k + \l$.
  }

  (d)
  \qproof{
    Suppose that
    \ali{
      \k_1 &= |K_1| &
      \k_2 &= |K_2| &
      \l_1 &= |L_1| &
      \l_2 &= |L_2|
    }
    for sets $K_1$, $K_2$, $L_1$, and $L_2$ where $K_1 \cap L_1 = \es$ and $K_2 \cap L_2 = \es$.
    Also suppose that $\k_1 \leq \k_2$ and $\l_1 \leq \l_2$.
    Thus $|K_1| = \k_1 \leq \k_2 = |K_2|$ so that there is an injective function $f$ from $K_1$ to $K_2$.
    Similarly there is an injective function $g : L_1 \to L_2$ since $|L_1| = \l_1 \leq \l_2 = |L_2|$.
    Now define $h : K_1 \cup L_1 \to K_2 \cup L_2$ by
    $$
    h(x) = \begin{cases}
      f(x) & x \in K_1 \\
      g(x) & x \in L_1 \,.
    \end{cases}
    $$
    We show that $h$ is injective so consider $x$ and $y$ in $K_1 \cup L_1$ where $x \neq y$.

    Case: $x \in K_1$, $y \in K_1$.
    Then
    $$
    h(x) = f(x) \neq f(y) = h(y)
    $$
    since $f$ is injective and $x \neq y$.

    Case: $x \in L_1$, $y \in L_1$.
    Then
    $$
    h(x) = g(x) \neq g(y) = h(y)
    $$
    since $g$ is injective and $x \neq y$.

    Case: $x \in K_1$, $y \in L_1$.
    Then we have $h(x) = f(x) \in K_2$ and $h(y) = g(y) \in L_2$ so that $h(x) \neq h(y)$ since $K_2$ and $L_2$ are disjoint.
    Note that this is the same as the case in which $x \in L_1$ and $y \in K_1$ since we simply switch $x$ and $y$.

    Since these cases are exhaustive and $h(x) \neq h(y)$ in each this shows that $h$ is injective.
    Hence we have demonstrated that
    $$
    \k_1 + \l_1 = |K_1 \cup L_1| \leq |K_2 \cup L_2| = \k_2 + \l_2
    $$
    as desired.
  }

  For solutions (e) through (h) suppose that
  \ali{
    \k &= |A| & \l &= |B| & \mu &= |C|
  }
  for sets $A$, $B$, and $C$.

  (e)
  \qproof{
    First we show that $|A \times B| = |B \times A|$ by constructing a bijection $f : A \times B \to B \times A$.
    For $(a,b) \in A \times B$ define
    $$
    f(a,b) = (b,a) \in B \times A \,,
    $$
    which is clearly a function.
    Then for $(a,b) \in A \times B$ and $(c,d) \in A \times B$ where $f(a,b) = f(c,d)$ we have
    $$
    f(a,b) = (b,a) = f(c,d) = (d,c)
    $$
    so that $b=d$ and $a=c$.
    Hence $(a,b) = (c,d)$ so that $f$ is injective.
    Now consider any $(b,a) \in B \times A$ so that clearly $f(a,b) = (b,a)$, noting that $(a,b) \in A \times B$.
    Clearly this shows that $f$ is surjective.

    Hence $f$ is bijective so that
    $$
    \k \cdot \l = |A \times B| = |B \times A| = \l \cdot \k
    $$
    as required.
  }

  (f)
  \qproof{
    Similar part (e) above, it is trivial to find a bijection from $A \times (B \times C)$ to $(A \times B) \times C$ so that
    $$
    \k \cdot (\l \cdot \mu) = |A \times (B \times C)| = |(A \times B) \times C| = (\k \cdot \l) \cdot \mu
    $$
    as desired.
  }

  (g)
  \qproof{
    Here suppose additionally that $B \cap C = \es$.
    First we note that since $B$ and $C$ are disjoint that $A \times B$ and $A \times C$ are also disjoint.
    Suppose that this is not the case so that there is an $(a,b) \in A \times B$ where $(a,b) \in A \times C$ also.
    Then clearly $b \in B$ and $b \in C$, which is a contradiction since they are disjoint.
    Now, it is also trivial to show the equality
    $$
    A \times (B \cup C) = (A \times B) \cup (A \times C) \,.
    $$
    Hence we have that
    $$
    \k \cdot (\l + \mu) = \k \cdot |B \cup C| = |A \times (B \cup C)| = |(A \times B) \cup (A \times C)|
    = |A \times B| + |A \times C| = \k \cdot \l + \k \cdot \mu
    $$
    as desired.
  }

  (h)
  \qproof{
    Here suppose that $\l > 0$  so that $B \neq \es$.
    Here we construct a bijection $f: A \to A \times B$, from which it follows that
    $$
    \k = |A| \leq |A \times B| = \k \cdot \l \,.
    $$
    Since $B \neq \es$ there exists a $b \in B$.
    So for any $a \in A$ define
    $$
    f(a) = (a, b) \,,
    $$
    which is clearly a function.
    So for $a_1, a_2 \in A$ where $a_1 \neq a_2$ we have that
    $$
    f(a_1) = (a_1, b) \neq (a_2, b) = f(a_2)
    $$
    so that $f$ is injective.
  }

  (i)
  \qproof{
    Suppose that
    \ali{
      \k_1 &= |A_1| & \k_2 &= |A_2| & \l_1 &= |B_1| & \l_2 &= |B_2|
    }
    for sets $A_1$, $A_2$, $B_1$, and $B_2$ where $\k_1 = |A_1| \leq |A_2| = \k_2$ and $\l_1 = |B_1| \leq |B_2| = \l_2$.
    Hence there is an injective function $f: A_1 \to A_2$ and injective function $g: B_1 \to B_2$.
    We shall construct an injective function $h: A_1 \times B_1 \to A_2 \times B_2$ so that it immediately follows that
    $$
    \k_1 \cdot \l_1 = |A_1 \times B_1| \leq |A_2 \times B_2| = \k_2 \cdot \l_2
    $$
    as required.
    So for $(a, b) \in A_1 \times B_1$ define
    $$
    h(a, b) = (f(a), g(b))
    $$
    Suppose then $(a, b) \in A_1 \times B_1$ and $(c, d) \in A_1 \times B_1$ where $(a,b) \neq (c,d)$.
    If $a \neq c$ then $f(a) \neq f(c)$ since $f$ is injective so that
    $$
    h(a,b) = (f(a), g(b)) \neq (f(c), g(d)) = h(c,d)
    $$
    Similarly if $b \neq d$ then $g(b) \neq g(d)$ since $g$ is injective.
    Hence again
    $$
    h(a,b) = (f(a), g(b)) \neq (f(c), g(d)) = h(c,d)
    $$
    Thus in all cases $h(a,b) \neq h(c,d)$ so that $h$ is injective.
  }

  (j) This is adequately proven in the text.

  For solutions (k) through (m) suppose that
  \ali{
    \k &= |A| & \l &= |B|
  }
  for sets $A$ and $B$.
  
  (k)
  \qproof{
    Suppose here that $\k \geq 2$.
    Then $2 \leq \k$ and $\k \leq \k$ so that by property (i) we have
    $$
    2 \cdot \k \leq \k \cdot \k \,.
    $$
    Then by property (j) we have
    $$
    \k + \k = 2 \cdot \k \leq \k \cdot \k
    $$
    as desired.
  }

  (l)
  \qproof{
    Here suppose that $\l = |B| > 0$ so that $B \neq \es$.
    Hence there exists a $b \in B$.
    We shall construct an injective $f: A \to A^B$, from which it follows that
    $$
    \k = |A| \leq |A^B| = \k^\l \,.
    $$
    So for any $a \in A$ define $f(a) = g$ where $g : B \to A$ is a function defined by $g(b) = a$ for all $b \in B$, noting that $g \neq \es$ since $B \neq \es$.

    Now consider any $a_1, a_2 \in A$ where $a_1 \neq a_2$ so that for any $b \in B$ we have
    $$
    f(a_1)(b) = a_1 \neq a_2 = f(a_2)(b) \,.
    $$
    From this it follows that $f(a_1) \neq f(a_2)$ so that $f$ is injective.
  }

  (m)
  \qproof{
    Here suppose that $\k = |A| > 1$ so that there are $a_1, a_2 \in A$ where $a_1 \neq a_2$.
    We shall construct an injective function $f: B  \to A^B$ so that
    $$
    \l = |B| \leq |A^B| = \k^\l \,.
    $$
    So for any $b \in B$ define $f(b) = g$ where $g: B \to A$ is a function defined by
    $$
    g(c) =
    \begin{cases}
      a_1 & c = b \\
      a_2 & c \neq b
    \end{cases}
    $$
    for $c \in B$.
    Now suppose that $b_1, b_2 \in B$ where $b_1 \neq b_2$.
    We then have
    $$
    f(b_1)(b_1) = a_1 \neq a_2 = f(b_2)(b_1)
    $$
    since $b_1 \neq b_2$.
    From this it follows that $f(b_1) \neq f(b_2)$ so that $f$ is injective.
  }

  (n)
  \qproof{
    Suppose that
    \ali{
      \k_1 &= |A_1| & \k_2 &= |A_2| & \l_1 &= |B_1| & \l_2 &= |B_2|
    }
    for sets $A_1$, $A_2$, $B_1$, and $B_2$ where $\k_1 = |A_1| \leq |A_2| = \k_2$ and $\l_1 = |B_1| \leq |B_2| = \l_2$.

    The theorem as presented in the text is actually not true in full generality.
    As a counterexample suppose that $A_1 = A_2 = B_1 = \es$ so that $\k_1 = \k_2 = \l_1 = 0$ and $B_2 = 1$ so that $\l_2 = 1$.
    Then certainly the hypotheses above are true but we also have
    $$
    \k_1^{\l_1} = 0^0 = 1 > 0 = 0^1 = \k_2^{\l_2} \,
    $$
    where we have used the results of Exercises~5.1.2 and 5.1.3.

    However, if we add the restriction that $\k_2 > 0$ then it becomes true.
    To prove this first note that this implies that $A_2 \neq \es$ so that there is an $a_2 \in A_2$.
    Also there is an injective function $f: A_1 \to A_2$ and an injective function $g: B_1 \to B_2$.
    We shall construct an injective $F: A_1^{B_1} \to A_2^{B_2}$, from which it follows that
    $$
    \k_1^{\l_1} = |A_1^{B_1}| \leq |A_2^{B_2}| = \k_2^{\l_2} \,.
    $$
    So for any $h_1 \in A_1^{B_1}$ define $F(h_1) = h_2$ where $h_2 \in A_2^{B_2}$ is defined by
    $$
    h_2(b) = \begin{cases}
      f(h_1(g^{-1}(b))) & b \in \ran(h_1) \\
      a_2 & b \notin \ran(h_1)
    \end{cases}
    $$
    for any $b \in B_2$, noting that $g^{-1}$ is a function on $\ran(h_1)$ since $g$ is injective.
    Clearly $F$ is a function but now we show that it is injective.

    So consider any $h_1, h_2 \in A_1^{B_1}$ where $h_1 \neq h_2$.
    Then there is a $b_1 \in B_1$ such that $h_1(b_1) \neq h_2(b_1)$.
    So let $b_2 = g(b_1)$ so that clearly $b_2 \in \ran(g)$ and $b_1 = g^{-1}(b_2)$.
    Hence we have
    $$
    F(h_1)(b_2) = f(h_1(g^{-1}(b_2))) = f(h_1(b_1)) \neq f(h_2(b_1)) = f(h_2(g^{-1}(b_2))) = F(h_2)(b_2)
    $$
    since $h_1(b_1) \neq h_2(b_1)$ and $f$ is injective.
    It thus follows that $F(h_1) \neq F(h_2)$ so that we have shown that $F$ is injective.
  }
}

\exercise{2}{
  Show that $\k^0 = 1$ and $\k^1 = \k$ for all $\k$.
}
\sol{
  \qproof{
    Suppose that $\k = |A|$ for a set $A$.

    We claim that $A^\es = \braces{\es} = 1$ so that clearly
    $$
    \k^0 = |A^\es| = |1| = 1 \,.
    $$
    First consider any $f \in A^\es$.
    Suppose that $f \neq \es$ so that there is a $(b,a) \in f \subseteq \es \times A$.
    But then $b \in \es$, which is a contradiction.
    Hence $f = \es$.
    So if there are any $f \in A^\es$ then $f = \es$ but are there any $f \in A^\es$?
    Clearly the empty set is a function from $\es$ to $A$ since it is vacuously true that for every $b \in \es$ there is a unique $a \in A$ such that $(b,a) \in \es$.
    Hence $\es \in A^\es$ so that $A^\es = \braces{\es} = 1$.

    We also claim that $|A^1| = |A|$ so that
    $$
    \k^1 = |A^1| = |A| = \k \,.
    $$
    To this end for any $f \in A^1$ define $F(f) = f(\es)$, noting that $1 = \braces{\es}$, so that clearly $F: A^1 \to A$.
    Now consider any $f,g \in A^1$ where $f \neq g$.
    Then it has to be that
    $$
    F(f) = f(\es) \neq g(\es) = F(g)
    $$
    so that $F$ is injective.
    Now consider any $a \in A$ and define $f \in A^1$ by $f(\es) = a$.
    Then clearly
    $$
    F(f) = f(\es) = a
    $$
    so that $F$ is surjective.
    Hence we've shown that $F$ is bijective.
  }
}

\exercise{3}{
  Show that $1^\k = 1$ for all $\k$ and $0^\k = 0$ for all $\k > 0$.
}
\sol{
  \qproof{
    Suppose that $\k = |A|$ for a set $A$.

    Note first that if $\k = 0$ then by Exercise~5.1.2 it follows that
    $$
    1^\k = 1^0 = 1 \,.
    $$
    In the case where $\k > 0$ we claim that there is a unique $f \in 1^A$ so that clearly then
    $$
    1^\k = |1^A| = 1 \,.
    $$
    For existence define $f:A \to 1$ by $f(x) = \es$ for all $x \in A$ so that clearly $f \in 1^A$.
    For uniqueness consider any $f_1, f_2 \in 1^A$.
    Since $1 = \braces{\es}$ it has to be that $f_1(x) = f_2(x) = \es$ for all $x \in A$.
    Hence $f_1 = f_2$.

    Now suppose also that $\k > 0$ so that $A \neq \es$.
    Hence there is an $a \in A$.
    We claim that in this case that $\es^A = \es$ so that
    $$
    0^\k = |\es^A| = |\es| = 0 \,.
    $$
    So suppose that $\es^A \neq \es$ so that there \emph{is} an $f \in \es^A$.
    Then it's true that for every $a \in A$ there is a unique $b \in \es$ such that $f(a) = b$.
    But since there \emph{is} an $a \in A$ this implies that there is also a $b \in \es$, which is a contradiction.
    Hence there can be no $f \in \es^A$ so that $\es^A = \es$.
  }
}

\exercise{4}{
  Prove that $\k^\k \leq 2^{\k \cdot \k}$.
}
\sol{
  \qproof{
    Suppose that $\k = |A|$ for a set $A$.
    Then we construct an injective $F : A^A \to 2^{A \times A}$ so that
    $$
    \k^\k = |A^A| \leq |2^{A \times A}| = |2|^{|A \times A|} = 2^{\k \cdot \k} \,.
    $$
    So for any $f \in A^A$ define $F(f) = g$ where $g \in 2^{A\times A}$ is defined by
    $$
    g(a_1, a_2) =
    \begin{cases}
      0 & f(a_1) \neq a_2 \\
      1 & f(a_1) = a_2
    \end{cases}
    $$
    for $(a_1, a_2) \in A \times A$.
    To show that $F$ is injective consider any $f,g \in A^A$ where $f \neq g$.
    Then there is an $a \in A$ such that $f(a) \neq g(a)$.
    Now let $a_1 = f(a)$ and $a_2 = g(a)$ so that
    $$
    f(a) = a_1 \neq a_2 = g(a) \,.
    $$
    Since $f(a) = a_1$ it follows by definition that $F(f)(a,a_1) = 1$.
    Similarly since $g(a) = a_2  \neq a_1$ it follows that $F(g)(a,a_1) = 0$.
    Hence we have
    $$
    F(f)(a, a_1) = 1 \neq 0 = F(g)(a, a_1)
    $$
    so that clearly $F(f) \neq F(g)$.
    Thus $F$ is injective.
  }
}

\exercise{5}{
  If $\abs{A} \leq \abs{B}$ and if $A \neq \es$, then there is a mapping of $B$ onto $A$.
  We later show, with the help of the Axiom of Choice, that the converse is also true: If there is a mapping of $B$ onto $A$, then $\abs{A} \leq \abs{B}$.
}
\sol{
  \qproof{
    Suppose that $|A| \leq |B|$ for sets $A$ and $B$ where $A \neq \es$.
    Then there is an $a \in A$.
    There is also an injective $f : A \to B$ so that $f^{-1}$ is a function from $\ran(f) \to A$.
    So let $g$ be a mapping from $B$ to $A$ defined by
    $$
    g(b) = \begin{cases}
      f^{-1}(b) & b \in \ran(f) \\
      a & b \notin \ran(f) \,.
    \end{cases}
    $$
    To show that $g$ is onto consider any $x \in A$ and let $b = f(a)$. Thus $b \in \ran(f)$ so that
    $$
    g(b) = f^{-1}(b) = f^{-1}(f(a)) = a \,.
    $$
    Hence $g$ is onto since $a$ was arbitrary.
  }
}

\exercise{6} {
  If there is a mapping of $B$ onto $A$, then $2^{\abs{A}} \leq 2^{\abs{B}}$.
  [Hint: Given $g$ mapping $B$ onto $A$, let $f(X) = \inv{g}[X]$, for all $X \ss A$.]
}
\sol{
  \qproof{
    Suppose that $f$ is a mapping from $B$ \emph{onto} $A$.
    We shall construct an injective $F: 2^A \to 2^B$ so that
    $$
    2^{|A|} = |2|^{|A|} = |2^A| \leq |2^B| = |2|^{|B|} = 2^{|B|} \,.
    $$
    So for any $g \in 2^A$ let $F(g) = h$ where $h \in 2^B$ is defined by
    $$
    h(b) = g(f(b))
    $$
    for $b \in B$.
    To show that $F$ is injective consider any $g_1, g_2 \in 2^A$ where $g_1 \neq g_2$.
    Then there is an $a \in A$ such that $g_1(a) \neq g_2(a)$.
    Since $f : B \to A$ is onto there is a $b \in B$ such that $f(b) = a$.
    Thus we have
    $$
    F(g_1)(b) = g_1(f(b)) = g_1(a) \neq g_2(a) = g_2(f(b)) = F(g_2)(b)
    $$
    so that $F(g_1) \neq F(g_2)$.
    Thus $F$ is injective.
  }
}

\exercise{7}{
  Use Cantor's Theorem to show that ``the set of all sets'' does not exist.
}
\sol{
  \qproof{
    Suppose that $X$ is the set of all sets.
    Consider any $Z \in \pset{X}$.
    Since clearly $Z$ is a set we have $Z \in X$.
    Thus since $Z$ was arbitrary it follows that $\pset{X} \subseteq X$ so that by Exercise~4.1.3 $|\pset{X}| \leq |X|$.
    However, this contradicts Cantor's Theorem, according to which $|\pset{X}| > |X|$.
    Thus $X$ cannot be the set of all sets.
  }
}

\exercise{8}{
  Let $X$ be a set and let $f$ be a one-to-one mapping of $X$ into itself such that $f[X] \pss X$.
  Then $X$ is infinite.
}
\sol{
  \qproof{
    For a set $X$ suppose that $f : X \to X$ is injective.
    Also suppose that $\ran(f)$ is a proper subset of $X$.

    Now suppose that $X$ is finite so that there is an $n \in \nats$ such that there is a bijective $g : n \to X$.
    We also note that clearly $g^{-1} : X \to n$ is also a bijection.
    Now define a function $h : n \to n$ by
    $$
    h(k) = (g^{-1} \circ f \circ g)(k) = g^{-1}(f(g(k))
    $$
    for any $k \in n$.
    Since $g$, $f$, and $g^{-1}$ are all injective it follows from Exercise~2.3.5 that $h$ is also injective.

    We now claim that $\ran(h)$ is proper subset of $n$.
    First we note that for any $m \in n$ we have
    $$
    g(h(m)) = g(g^{-1}(f(g(m)))) = f(g(m))
    $$
    since $g$ is a function.
    Now, since $\ran(f) \subset X$ there is an $x \in X$ such that $x \notin \ran(f)$.
    So let $k = g^{-1}(x)$ so that $g(k) = x$.
    Now suppose that there is an $m \in n$ such that $h(m) = k$.
    Then per the above we have
    $$
    f(g(m)) = g(h(m)) = g(k) = x \,,
    $$
    which is impossible since $x \notin \ran(f)$.
    So it must be that there is no such $m$ so that $k \notin \ran(h)$
    Hence since $k \in n$ it follows that $\ran(g) \subset n$.

    Now clearly $h$ is a surjective mapping from $n$ to $\ran(h)$.
    But since $h$ is also injective it is thus a bijection from $n$ to $\ran(n)$.
    However, according to Lemma~4.2.2 there is no bijective mapping from $n$ to $\ran(n)$ since $\ran(h) \subset n$.
    We have thus arrived at a contradiction so that, if the hypotheses hold, then $X$ cannot be finite.
    Hence by definition $X$ is infinite.
  }
}

\exercise{9}{
  Every countable set is Dedekind infinite.
}
\sol{
  \begin{lem}\label{lem:card:equidede}
    If sets $X$ and $Y$ are equipotent (i.e. $|X| = |Y|$) and $Y$ is Dedekind infinite then $X$ is also Dedekind infinite.
  \end{lem}
  \qproof{
    Since $X$ and $Y$ are equipotent there is a bijective $f : X \to Y$ so that $\inv{f}$ is also bijective.
    Also since $Y$ is Dedekind infinite there is a $Z \subset Y$ such that there is a bijective $g : Y \to Z$ so that $\inv{g}$ is also bijective.
    So since $Z \subset Y$ there is a $y \in Y$ such that $y \notin Z$.
    So let $S = X - \braces{\inv{f}(y)}$.
    Clearly since $\inv{f}(y) \in X$ it follows that $S \subset X$ since $\inv{f}(y) \notin S$.
    Now define $h: X \to S$ by
    $$
    h(x) = (\inv{f} \circ g \circ f)(x) = \inv{f}(g(f(x))))
    $$
    for $x \in X$.
    Since $f$ is a function this implies that
    $$
    f(h(x)) = f(\inv{f}(g(f(x)))) = g(f(x)) \,.
    $$
    Now suppose for a moment that there is an $x \in X$ such that $h(x) = \inv{f}(y)$.
    Then
    $$
    g(f(x)) = f(h(x)) = f(\inv{f}(y)) = y \,,
    $$
    which is impossible since $y \notin Z$ but $\ran(g) \subseteq Z$.
    Hence there is no such $x$ so that $h$ really is a map from $X$ to $S$ (as opposed to $X$ to $X$).

    Now, since $\inv{f}$, $g$, and $f$ are all injective it follows from Exercise~2.3.5 that $h$ is injective as well.
    Then consider any $s \in S$ and let $x = \inv{f}(\inv{g}(f(s)))$, noting that $\inv{g}(f(s))$ exists since $s \neq \inv{f}(y)$.
    Then we have
    $$
    h(x) = \inv{f}(g(f(\inv{f}(\inv{g}(f(s))))))) = \inv{f}(g(\inv{g}(f(s)))) = \inv{f}(f(s)) = s
    $$
    so that $h$ is surjective since $s$ was arbitrary.
    Hence $h$ is a bijective map from $X$ to $S$, and since $S \subset X$ this means that $X$ is Dedekind infinite.
  }

  \begin{lem}\label{lem:card:natdede}
    $\nats$ is Dedekind infinite.
  \end{lem}
  \qproof{
    Let $N = \nats - \braces{0}$ so that clearly $N$ is proper subset of $\nats$.
    Then we define the map $f : \nats \to N$ by
    $$
    f(n) = n+1
    $$
    for $n \in \nats$.
    Consider any $n,m \in \nats$ where $n \neq m$.
    Then clearly
    \gath{
      n \neq m \\
      n+1 \neq m+1 \\
      f(n) \neq f(m)
    }
    so that $f$ is injective.
    Now consider any $n \in N$ so that clearly $f(n-1) = (n-1) + 1 = n$, noting that since $n \neq 0$ we have $n \geq 1$ so that $n-1 \geq 0$.
    Hence $n-1 \in \nats$.
    This shows that $f$ is surjective.
    Hence $f$ is a bijection from $\nats$ onto a proper subset $N$ so that by definition $\nats$ is Dedekind infinite.
  }

  \mainprob
  \qproof{
    Suppose that $X$ is a countable set.
    Then by definition $X$ is equipotent to $\nats$.
    Hence since $\nats$ is Dedekind infinite (Lemma~\ref{lem:card:natdede}) it follows that $X$ is as well by Lemma~\ref{lem:card:equidede}.
  }
}

\exercise{10}{
  If $X$ contains a countable subset, then $X$ is Dedekind infinite.
}
\sol{
  \qproof{
    Suppose that $X$ is a set with a countable subset $Y$.
    Then by Exercise~5.1.9 $Y$ is Dedekind infinite so that there is a $Z \subset Y \subseteq X$ such that there is a bijective $f : Y \to Z$.
    So define the following $g: X \to X$ by
    $$
    g(x) = \begin{cases}
      f(x) & x \in Y \\
      x & x \notin Y
    \end{cases}
    $$
    for any $x \in X$.
    Now since $Z \subset Y$ there is a $y \in Y$ such that $y \notin Z$, noting that since $Y \subseteq X$, $y \in X$.

    First we claim that $y \notin \ran(g)$.
    So suppose that it is so that there is an $x \in X$ such that $g(x) = y$.
    If $x \in Y$ then by definition $g(x) = f(x) = y$, but this is a contradiction since $f : Y \to Z$ but $y \notin Z$.
    On the other hand if $x \notin Y$ then we have $g(x) = x = y$, which is also a contradiction since $y = x \in Y$.
    Since a contradiction follows in either case it must be that there is no such $x$ so that $y \notin \ran(g)$.
    Hence $\ran(g) \subset X$.

    Clearly $g$ is a surjective map from $X$ to $\ran(g)$ so we now show that it is injective.
    So consider any $x_1, x_2 \in X$ where $x_1 \neq x_2$.

    Case: $x_1 \in Y$ and $x_2 \in Y$.
    Then
    $$
    g(x_1) = f(x_1) \neq f(x_2) = g(x_2)
    $$
    since $f$ is injective.

    Case: $x_1 \notin Y$ and $x_2 \notin Y$.
    Then
    $$
    g(x_1) = x_1 \neq x_2 = g(x_2) \,.
    $$

    Case: $x_1 \in Y$ and $x_2 \notin Y$.
    Then $g(x_1) = f(x_1) \in Z$ but $g(x_2) = x_2 \notin Y$ so that  $x_2 \notin Z$ either since $Z \subset Y$.
    Hence $f(x_1) \neq x_2$ so that
    $$
    g(x_1) = f(x_1) \neq x_2 = g(x_2) \,.
    $$
    Thus in all cases $g(x_1) \neq g(x_2)$ so that $g$ is injective since $x_1$ and $x_2$ were arbitrary.

    Therefore we have shown that $g$ is a bijective map from $X$ to $\ran(g) \subset X$ so that $X$ is Dedekind infinite by definition.
  }
}

\exercise{11}{
  If $X$ is Dedekind infinite, then it contains a countable subset.
  [Hint: Let $x \in X - f[X]$; define $x_0 = x$, $x_1 = f(x_0)$, \ldots, $x_{n+1} = f(x_n)$, \ldots \,.
    The set $\braces{x_n \where n \in \nats}$ is countable.]
}
\sol{
  \qproof{
  Suppose that $X$ is a Dedekind infinite set.
  Then there is a $Y \subset X$ such that there is a bijective $f : X \to Y$.
  Since $Y \subset X$ there is an $x \in X$ such that $x \notin Y$.
  So first define $x_0 = x$ and then for $n \in \nats$ define $x_{n+1} = f(x_n)$.

  We claim that $x_n \neq x_m$ for any $n,m \in \nats$ where $n \neq m$, from which it clearly follows that
  $$
  Z = \braces{x_n \where n \in \nats}
  $$
  is a countable set.
  So consider any $n,m \in \nats$ where $n \neq m$
  Without loss of generality we can assume that $n < m$.
  Suppose that  $x_n = x_m$.
  We now show by induction that $x_{n-k} = x_{m-k}$ for all $n \geq k \geq 0$.
  If $k=0$ then we clearly have
  $$
  x_{n-k} = x_{n-0} = x_n = x_m = x_{m-0} = x_{m-k} \,.
  $$
  Now suppose that $x_{n-k} = x_{m-k}$.
  We then have
  $$
  f(x_{n-(k+1)}) = f(x_{n-k-1}) = x_{n-k} = x_{m-k} = f(x_{m-k-1}) = f(x_{m-(k+1)})
  $$
  Since $f$ is injective this implies that $x_{n-(k+1)} = x_{m-(k+1)}$ so that inductive proof is complete.
  So since this holds for $k=n$ we have that
  $$
  x_0 = x_{n-n} = x_{m-n} = f(x_{m-n-1}) \,,
  $$
  Noting that $m-n-1 \geq 0$ since $m \geq n +1$.
  But $x_0 = x \notin Y$ and $f(x_{m-n-1}) \in Y$ since $f:X \to Y$ so that we have a contradiction.
  So it must be that $x_n \neq x_m$.
  Hence $Z$ is countable.
  Since also clearly $Z \subseteq X$ the proof is complete.
  }
}

\exercise{12}{
  If $A$ and $B$ are Dedekind infinite, the $A \cup B$ is Dedekind infinite.
  [Hint: Use Exercise~1.11.]
}
\sol{
  \qproof{
    Suppose that sets $A$ and $B$ are both Dedekind infinite.
    Then $A$ contains a countable subset $C$ by Exercise~5.1.11.
    Clearly $C \subseteq A \cup B$ so that $C$ is a countable subset of $A \cup B$.
    Hence by Exercise~5.1.10 $A \cup B$ is Dedekind infinite.
  }
}

\exercise{13}{
  If $A$ and $B$ are Dedekind infinite, then $A \times B$ is Dedekind infinite.
  [Hint: Use Exercise~1.11.]
}
\sol{
  \qproof{
    Suppose that $A$ and $B$ are both Dedekind infinite.
    Then $A$ contains a countable subset $C$ by Exercise~5.1.11.
    Also since $B$ is Dedekind infinite it is not finite by Exercise~5.1.8.
    Hence $B \neq \es$ so that there is a $b \in B$.
    Clearly then the set
    $$
    D = \braces{(a,b) \where a \in C}
    $$
    is a countable subset of $A \times B$ so that $A \times B$ is Dedekind infinite by Exercise~5.1.10.
  }
}

\exercise{14}{
  If $A$ is infinite, then $\pset{\pset{A}}$ is Dedekind infinite.
  [Hint: For each $n \in \nats$, let $S_n = \braces{X \pss A \where \abs{X} = n}$.
    The set $\braces{S_n \where n \in \nats}$ is a countable subset of $\pset{\pset{A}}$.]
}
\sol{
  \begin{lem}\label{lem:card:infss}
    If $A$ is an infinite set then for any $n \in \nats$ there is a $B \subseteq A$ such that $|B| = n$.
  \end{lem}
  \qproof{
    Suppose that $A$ is an infinite set and consider any $n \in \nats$.
    Then $\cnats \leq |A|$ so that there is an injective $f : \nats \to A$.
    Now $n \in \nats$ but also $n \subseteq \nats$.
    So define the set
    $$
    B = \braces{f(k) \where k \in n} \,.
    $$
    Clearly $B \subseteq A$ and we show that $|B| = n$ by defining a mapping $g : n \to B$ by
    $$
    g(k) = f(k)
    $$
    for $k \in n$.
    Since $f$ is injective clearly $g$ is.
    Now consider any $b \in B$.
    By definition then there is a $k \in n$ such that $f(k) = b$.
    Hence $g(k) = f(k) = b$ so that $g$ is surjective.
    Hence since $g$ is bijective $|B| = n$ as desired.
  }

  \mainprob
  \qproof{
    Suppose that $A$ is infinite.
    Then for any $n \in \nats$ define
    $$
    S_n = \braces{X \in \pset{A} \where |X| = n} \,,
    $$
    noting that $S_n \neq \es$ by Lemma~\ref{lem:card:infss}.
    We also note that for $n,m \in \nats$ where $n \neq m$ we have $S_n \neq S_m$ since for any $X \in S_n$ and $Y \in S_m$ we have
    $$
    |X| = n \neq m = |Y|
    $$
    so that $X \neq Y$.
    From this it follows that
    $$
    S = \braces{S_n \where n \in \nats}
    $$
    is a countable set.
    Also, for an $S_n \in S$ we have that each $X \in S_n$ is in $\pset{A}$ so that $S_n \subseteq \pset{A}$.
    Hence each $S_n \in \pset{\pset{A}}$.
    Thus $S \subseteq \pset{\pset{A}}$.
    Since $S$ is countable it then follows that $\pset{\pset{A}}$ is Dedekind infinite by Exercise~5.1.10.
  }
}
