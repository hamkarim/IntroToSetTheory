\subsection{Infinite Sums and Products of Cardinal Numbers}

% Macros useful for this section
\def\uni{\bigcup_{i \in I}}
\def\sumi{\sum_{i \in I}}
\def\prodi{\prod_{i \in I}}
\def\prodj{\prod_{j \in J}}

\newcommand\unaj[1]{\bigcup_{j \in J_{#1}} A_j}
\exercise{1}{
  If $J_i$ ($i \in I$) are mutually disjoint sets and  $J = \bigcup_{i \in I} J_i$, and if $\k_j$ ($j \in J$) are cardinals, then
  \gath{
    \sum_{i \in I} \parens{ \sum_{j \in J_i} \k_j } = \sum_{j \in J} \k_j
  }
  (\emph{associativity} of $\sum$)
}
\sol{
  \qproof{
  Suppose that $\angles{A_j \where j \in J}$ are mutually disjoint sets where $\abs{A_j} = \k_j$ for every $j \in J$.
  Then, by definition, we have that
  \begin{gather}
    \sum_{j \in J} \k_j = \abs{\unaj{}} \,. \label{eqn:infsp:sumkjall}
  \end{gather}
  Now let $S = \braces{J_i \where i \in I}$, from which it is trivial to show that $\bigcup S = J$.
  It follows from Exercise~2.3.10 that
  \begin{gather}
    \unaj{} = \bigcup_{a \in \bigcup S} A_a = \bigcup_{C \in S} \parens{\bigcup_{a \in C} A_a}
    = \bigcup_{i \in I} \parens{\unaj{i}} \,. \label{eqn:infsp:dun}
  \end{gather}
  We claim that the sets $\angles{\unaj{i} \where i \in I}$ are mutually disjoint.
  So consider any $i_1$ and $i_2$ in $I$ where $i_1 \neq i_2$, and suppose that $\unaj{i_1}$ and $\unaj{i_2}$ are \emph{not} disjoint so that there is an $x$ where $x \in \unaj{i_1}$ and $x \in \unaj{i_2}$.
  Then there is a $j_1 \in J_{i_1}$ where $x \in A_{j_1}$ and a $j_2 \in J_{i_2}$ where $x \in A_{j_2}$.
  Now, since $\angles{J_i \where i \in I}$ are mutually disjoint and $i_1 \neq i_2$, it follows that $J_{i_1}$ and $J_{i_2}$ are disjoint.
  Therefore it has to be that $j_1 \neq j_2$ (since $j_1 \in J_{i_1}$ and $j_2 \in J_{i_2}$).
  But then $A_{j_1}$ and $A_{j_2}$ are not disjoint (since $x$ is in both) despite the fact that $j_1 \neq j_2$, which contradicts the fact that $\angles{A_j \where j \in J}$ are mutually disjoint.
  So it must be that in that $\unaj{i_1}$ and $\unaj{i_2}$ are disjoint, which proves the result since $i_1$ and $i_2$ were arbitrary.

  Since $\angles{\unaj{i} \where i \in I}$ have been shown to be mutually disjoint, it follows by definition that
  \begin{gather}
    \sum_{i \in I} \abs{\unaj{i}} = \abs{\bigcup_{i \in I} \parens{\unaj{i}}} \,. \label{eqn:infsp:sumunaj}
  \end{gather}
  Lastly, we also clearly have that $\angles{A_j \where j \in J_i}$ are mutually disjoint for any $i \in I$ (since $\angles{A_j \where j \in J}$ are mutually disjoint) so that
  \begin{gather}
    \sum_{j \in J_i} \k_j = \abs{\unaj{i}} \,. \label{eqn:infsp:sumkj}
  \end{gather}

  Putting this all together, we have
  \ali{
    \sum_{j \in J} \k_j &= \abs{\unaj{}} & \text{(by \eqref{eqn:infsp:sumkjall})} \\
    &= \abs{\bigcup_{i \in I} \parens{\unaj{i}}} & \text{(by \eqref{eqn:infsp:dun})} \\
    &= \sum_{i \in I} \abs{\unaj{i}} & \text{(by \eqref{eqn:infsp:sumunaj})} \\
    &= \sum_{i \in I} \parens{\sum_{j \in J_i} \k_j} & \text{(by \eqref{eqn:infsp:sumkj})}
  }
  as desired.
  }
}

\exercise{2}{
  If $\k_i \leq \l_i$ for all $i \in I$ then $\sum_{i \in I} \k_i \leq \sum_{i \in I} \l_i$.
}
\sol{
  \def\un{\bigcup_{i \in I}}
  \qproof{
    Suppose that $\angles{A_i \where i \in I}$ are mutually disjoint sets such that $\abs{A_i} = \k_i$ for all $i \in I$.
    Similarly, suppose that $\angles{B_i \where i \in I}$ are mutually disjoint sets such that $\abs{B_i} = \l_i$ for all $i \in I$.
    It then follows by definition that
    \ali{
      \sum_{i \in I} \k_i &= \abs{\un A_i} &
      \sum_{i \in I} \l_i &= \abs{\un B_i} \,.
    }
    Now, we have $\abs{A_i} = \k_i \leq \l_i = \abs{B_i}$ so that there is an injective function $f_i : A_i \to B_i$ for all $i \in I$.
    With the help of the Axiom of Choice, we can choose one of these functions for each $i \in I$ and form the system of functions $\braces{f_i}_{i \in I}$.

    We claim that $\braces{f_i}_{i \in I}$ is a compatible system of functions.
    To see this, consider any $i_1$ and $i_2$ in $I$.
    If $i_1 = i_2$ then consider any $x \in \dom(f_{i_1}) \cap \dom(f_{i_2}) = A_{i_1} \cap A_{i_2} = A_{i_1} \cap A_{i_1} = A_{i_1}$.
    Then clearly $f_{i_1}(x) = f_{i_2}(x)$ since $i_1 = i_2$ and $f_{i_1} = f_{i_2}$ is a function.
    On the other hand, if $i_1 \neq i_2$, then we have that $\dom(f_{i_1}) \cap \dom(f_{i_2}) = A_{i_1} \cap A_{i_2} = \es$ since $\angles{A_i \where i \in I}$ are mutually disjoint and $i_1 \neq i_2$.
    Hence it is vacuously true that $f_{i_1}(x) = f_{i_2}(x)$ for all $x \in \dom(f_{i_1}) \cap \dom(f_{i_2})$ since there is no such $x$.
    Since $i_1$ and $i_2$ were arbitrary, this shows that $\braces{f_i}_{i \in I}$ is a compatible system (see Definition 2.3.10).
    It then follows from Theorem~2.3.12 that $f = \un f_i$ is a function with domain $\un \dom(f_i) = \un A_i$.

    Though perhaps it may seem obvious, we show formally that $f(x) = f_i(x)$ for any $x \in A_i$ (for any $i \in I$).
    So consider any such $i \in I$ and $x \in A_i$.
    Then $(x, f(x)) \in f = \un f_i$ so that there is a $j \in I$ where $(x, f(x)) \in f_j$.
    Suppose for a moment that $j \neq i$ so that $x \in \dom(f_j) = A_j$.
    Since $x \in A_i$ and $x \in A_j$ but $i \neq j$, this contradicts the fact that $\angles{A_i \where i \in I}$ are mutually disjoint.
    Hence it must be that $j = i$ so that $(x, f(x)) \in f_j = f_i$.
    From this of course it follows that $f_i(x) = f(x)$ as desired.
    
    We also claim that $f(x) \in \un B_i$ for any $x \in \un A_i$ so that $\un B_i$ can be the codomain of $f$.
    This is easy to show: consider any $x \in \un A_i$ so that there is an $i \in I$ where $x \in A_i$.
    It then follows that $f(x) = f_i(x) \in B_i$ since $f_i$ is a function from $A_i$ to $B_i$.
    Therefore we clearly have $f(x) \in \un B_i$.
    This shows the result since $x$ was arbitrary.

    We also claim that $f$ is injective.
    So consider any $x_1$ and $x_2$ in $\un A_i$ where $x_1 \neq x_2$.
    Then there are $i_1$ and $i_2$ such that $x_1 \in A_{i_1}$ and $x_2 \in A_{i_2}$.
    If $i_1 = i_2$ then $f(x_1) = f_{i_1}(x_1) \neq f_{i_1}(x_2) = f_{i_2}(x_2) = f(x_2)$ since $f_{i_1} = f_{i_2}$ is injective.
    If $i_1 \neq i_2$ then $f(x_1) = f_{i_1}(x_1) \in B_{i_1}$ whereas $f(x_2) = f_{i_2}(x_2) \in B_{i_2}$.
    Since $\braces{B_i \where i \in I}$ are mutually disjoint and $i_1 \neq i_2$ it follows that $f(x_1) \neq f(x_2)$.
    Therefore $f$ is an injective function from $\un A_i$ to $\un B_i$ so that
    \gath{
      \sum_{i \in I} \k_i = \abs{\un A_i} \leq \abs{\un B_i} = \sum_{i \in I} \l_i
    }
    as desired.
  }
}

\def\sumn{\sum_{n=0}^\infty}
\exercise{3}{
  Find some cardinals $\k_n$, $\l_n$ ($n \in \nats$) such that $k_n < \l_n$ for all $n$, but $\sumn \k_n = \sumn \l_n$.
}
\sol{
  Let $\k_n = 1$ and $\l_n = 2$ for all $n \in \nats$.
  We claim that these satisfy the required properties.
  \qproof{
    Clearly we have $\k_n = 1 < 2 = \l_n$ for all $n \in \nats$.
    It then follows from Exercise~9.1.4 (and also the more general Theorem~9.1.3) that
    \gath{
      \sumn \k_n = \sumn 1 = \sum_{n \in \nats} 1 = \sum_{n < \al_0} 1 = \al_0 \cdot 1 = \al_0
      = \al_0 \cdot 2 = \sum_{n < \al_0} 2 = \sum_{n \in \nats} 2 = \sumn 2 = \sumn \l_n
    }
    as desired.
  }
}

\exercise{4}{
  Prove that $\k + \k + \cdots \text{($\l$ times)} = \l \cdot \k$.
}
\sol{
  \def\una{\bigcup_{\a < \l}}
  \qproof{
    First, if $\l = 0$ then, by convention, we have
    \gath{
      \k + \k + \cdots \text{($\l$ times)} = \k + \k + \cdots \text{(0 times)} = 0 = 0 \cdot \k = \l \cdot \k
    }
    regardless of what $\k$ is.
    So assume in what follows that $\l > 0$ so $\l \geq 1$.
    
    For each $\a < \l$, define $A_\a = \braces{(\a, \b) \where \b \in \k}$.
    Consider then any $\a_1 < \l$ and $\a_2 < \l$ where $\a_1 \neq \a_2$.
    Suppose that both $(x,y) \in A_{\a_1}$ and $(x,y) \in A_{\a_2}$.
    It then follows that $x = \a_1$ and $x = \a_2$ so that $x = \a_1 = \a_2$, which contradicts our assumption that $\a_1 \neq \a_2$!
    So it must be that no such $(x,y)$ exists so that $A_{\a_1}$ and $A_{\a_2}$ are disjoint.
    Since $\a_1$ and $\a_2$ were arbitrary, this shows that $\angles{A_\a \where \a < \l}$ are mutually disjoint sets.
    We also clearly have that $\abs{A_\a} = \k$ for each $\a < \l$.
    It therefore follows from the definition of cardinal summation that
    \gath{
      \k + \k + \cdots \text{($\l$ times)} = \sum_{\a < \l} \k = \abs{\una A_\a} \,.
    }

    Now we show that $\una A_\a = \l \times \k$.
    First consider any $(x,y) \in \una$ so that there is an $\a < \l$ where $(x,y) \in A_\a$.
    We then have that $x = \a$ and $y \in \k$.
    Therefore $x = \a < \l$ so that $x \in \l$ by the definition of $<$ for ordinal numbers.
    Hence $x \in \l$ and $y \in \k$ so that $(x,y) \in \l \times \k$, which shows that $\una A_\a \ss \l \times \k$ since $(x,y)$ was arbitrary.
    Now consider any $(x,y) \in \l \times \k$ so that $x \in \l$ and $y \in \k$.
    Let $\a = x \in \l$ so that $\a < \l$.
    Hence $(x,y) = (\a, y)$ for $\a < \l$ and $y \in \k$, which shows that $(x,y) \in A_\a$ so that clearly $(x,y) \in \una A_\a$.
    This shows that $\l \times \k \ss \una A_\a$ since again $(x,y)$ was arbitrary.
    Thus $\una A_\a = \l \times \k$ as desired.

    Putting all this together, we have
    \gath{
      \k + \k + \cdots \text{($\l$ times)} = \sum_{\a < \l} \k = \abs{\una A_\a} = \abs{\l \times \k} 
      = \l \cdot \k
    }
    by the definition of cardinal multiplication since obviously $\abs{\l} = \l$ and $\abs{\k} = \k$.
  }
}

\exercise{5}{
  Prove the \emph{distributive} law:
  \gath{
    \l \cdot \parens{\sumi \k_i} = \sumi \parens{\l \cdot \k_i} \,.
  }
}
\sol{
  \qproof{
    Suppose that $\angles{A_i \where i \in I}$ are mutually disjoint sets such that $\abs{A_i} = \k_i$ for all $i \in I$.
    Then by definition $\sumi \k_i = \abs{\uni A_i}$.
    Also suppose that $B$ is a set such that $\abs{B} = \l$.

    We claim first that $B \times \uni A_i = \uni \parens{B \times A_i}$.
    This is easy to show since, for any $x$ and $y$, we have
    \ali{
      (x, y) \in B \times \uni A_i &\bic x \in B \land \exists i \in I (y \in A_i) \\
      &\bic \exists i \in I (x \in B \land y \in A_i) \\
      &\bic \exists i \in I ((x,y) \in B \times A_i) \\
      &\bic (x,y) \in \uni (B \times A_i) \,.
    }
    We then have
    \ali{
      \l \cdot \sumi \k_i &= \l \cdot \abs{\uni A_i} & \text{(by the definition of cardinal summation)} \\
      &= \abs{B \times \uni A_i} & \text{(by the definition of cardinal multiplication)} \\
      &= \abs{\uni \parens{B \times A_i}} & \text{(by what was just shown above)} \\
      &= \sumi \abs{B \times A_i} & \text{(by the definition of cardinal summation)} \\
      &= \sumi \parens{\l \cdot \k_i} & \text{(by the definition of cardinal multiplication)}
    }
    as desired.
    We note that $\abs{\uni (B \times A_i)} = \sumi \abs{B \times A_i}$ works since the sets $\angles{B \times A_i \where i \in I}$ are mutually disjoint.
    This is easy to see by considering $i$ and $j$ in $I$ where $i \neq j$.
    Then, if $(x,y) \in B \times A_i$ and also $(x,y) \in B \times A_j$, it follows that $y \in A_i$ and $y \in A_j$, which cannot be since $i \neq j$ and $\angles{A_i \where i \in I}$ are mutually disjoint.
    Hence it must be that there is no such ordered pair $(x,y)$ so that $B \times A_i$ and $B \times A_j$ are disjoint, which proves the result since $i$ and $j$ were arbitrary.
  }
}

\exercise{6}{
  $\abs{\uni A_i} \leq \sumi \abs{A_i}$.
}
\sol{
  \qproof{
    First let $\angles{B_i \where i \in I}$ be mutually disjoint sets where $\abs{B_i} = \abs{A_i}$ for every $i \in I$.
    It then follows that $\sumi \abs{A_i} = \abs{\uni B_i}$ by definition.
    For each $i \in I$ we can choose a bijection $f_i$ from $A_i$ to $B_i$ by the Axiom of Choice since $\abs{A_i} = \abs{B_i}$.
    We construct a function $f : \uni A_i \to \uni B_i$.
    For each $x \in \uni A_i$ we have that $x \in A_j$ for some $j \in I$.
    We choose one such $j$, which requires the Axiom of Choice, and set $f(x) = f_j(x)$.
    Clearly $f(x) = f_j(x) \in B_j$ so that then $f(x) \in \uni B_i$, which shows that $\uni B_i$ can be the codomain for $f$.

    We show that $f$ is injective.
    So consider any $x_1$ and $x_2$ in $\uni A_i$ where $x_1 \neq x_2$.
    Then we have chosen unique $j_1$ and $j_2$ where $f(x_1) = f_{j_1}(x)$ and $f(x_2) = f_{j_2}(x)$.
    If $j_1 = j_2$ then we have $f(x_1) = f_{j_1}(x_1) \neq f_{j_1}(x_2) = f_{j_2}(x_2) = f(x_2)$ since $f_{j_1} = f_{j_2}$ is injective and $x_1 \neq x_2$.
    If $j_1 \neq j_2$ then $f(x_1) = f_{j_1}(x_1) \in B_{j_1}$ whereas $f(x_2) = f_{j_2}(x_2) \in B_{j_2}$.
    Since $j_1 \neq j_2$ and $\angles{B_i \where i \in I}$ are mutually disjoint, it follows that $f(x_1) \neq f(x_2)$.
    Since this is true in both cases and $x_1$ and $x_2$ were arbitrary, it shows that $f$ is injective.

    Since $f$ is injective we, we have
    \gath{
      \abs{\uni A_i} \leq \abs{\uni B_i} = \sumi \abs{A_i}
    }
    as desired.
  }
}


\exercise{7}{
  If $J_i$ ($i \in I)$ are mutually disjoint sets and $J = \uni J_i$, and if $\k_j$ ($j \in J$) are cardinals, then
  \gath{
    \prodi \parens{\prod_{j \in J_i} \k_j} = \prod_{j \in J} \k_j
  }
  (\emph{associativity} of $\prod$).
}
\sol{
  \qproof{
    Suppose that $\angles{A_j \where j \in J}$ are sets where $\abs{A_j} = \k_j$ for each $j \in J$.
    It then follows that
    \begin{gather}
      \prodj \k_j = \abs{\prodj A_j} \,. \label{eqn:infsp:prodJ}
    \end{gather}
    Now, for any $i \in I$, set $B_i = \prod_{j \in J_i} A_j$ so that
    \begin{gather}
      \prod_{j \in J_i} \k_j = \abs{\prod_{j \in J_i} A_j} = \abs{B_i} \label{eqn:infsp:prodJi}
    \end{gather}
    by definition.

    Now we construct a bijection $f$ from $\prodj A_j$ to $\prodi B_i$.
    So consider any $a \in \prodj A_j$ so that $a = \angles{a_j \where j \in J}$ where $a_j \in A_j$ for every $j \in J$.
    Now, for each $i \in I$, set $a_i' = \angles{a_j \where j \in J_i}$, noting that clearly $j \in J = \bigcup_{i \in I} J_i$ for each $j \in J_i$ so that $a_j$ has been defined as in the range of $a$.
    We then have that $a_j \in A_j$ for $j \in J_i$ so that $a_i' \in \prod_{j \in J_i} A_j = B_i$.
    So set $b = \angles{a_i' \where i \in I}$ so that clearly $b \in \prodi B_i$, and set $f(a) = b$.
    Since $f(a) = b \in \prodi B_i$ for any $a \in \prodj A_j$, we have that $f$ is indeed a function from $\prodj A_j$ into $\prodi B_i$.

    We claim first that $f$ is injective.
    So consider any $\a$ and $\b$ in $\prodj A_j$ where $\a \neq \b$.
    It then follows that $\a = \angles{\a_j \where j \in J}$ and $\b = \angles{\b_j \where j \in J}$ where both $\a_j$ and $\b_j$ are in $A_j$ for any $j \in J$.
    Now, since $\a \neq \b$ it follows that there is a $j_0 \in J$ such that $\a_{j_0} \neq \b_{j_0}$.
    Also there is an $i_0 \in I$ such that $j_0 \in J_{i_0}$ since $j_0 \in J = \bigcup_{i \in I} J_i$.
    Now let $\a_i' = \angles{\a_j \where j \in J_i}$ and $\b_i' = \angles{\b_j \where j \in J_i}$ for $i \in I$.
    We then have that $\a_{i_0}' \neq \b_{i_0}'$ since $j_0 \in J_{i_0}$ and $\a_{j_0} \neq \b_{j_0}$.
    Clearly then $f(\a) = \angles{\a_i' \where i \in I}$ and $f(\b) = \angles{\b_i' \where i \in I}$ by definition so that $f(\a) \neq f(\b)$ since $i_0 \in I$ and $\a_{i_0}' \neq \b_{i_0}'$.
    This shows that $f$ is injective since $\a$ and $\b$ were arbitrary.

    We also claim that $f$ is onto.
    Consider any $b \in \prodi B_i$ so that $b = \angles{b_i \where i \in I}$ where each $b_i \in B_i$ for $i \in I$.
    So, for any $i \in I$, we have that $b_i \in B_i = \prod_{j \in J_i} A_j$ so that $b_i = \angles{a_{ij} \where j \in J_i}$ where each $a_{ij} \in A_j$ for $j \in J_i$.
    Now we construct a function $g$.
    So consider any $j_0 \in J$ so that there is a unique $i_0 \in I$ such that $j_0 \in J_{i_0}$, where the uniqueness clearly follows from the fact that $\angles{J_i \where i \in I}$ are mutually disjoint.
    Then simply set $g(j_0) = a_{i_0 j_0} \in A_{j_0}$ so that clearly $g \in \prodj A_j$.
    If we then set $a_i' = \angles{g(j) \where j \in J_i} = \angles{a_{ij} \where j \in J_i} = b_i$ for all $i \in I$, then $f(g) = \angles{a_i' \where i \in I} = \angles{b_i \where i \in I} = b$.
    Since $b$ was arbitrary this shows that $f$ is indeed onto.

    It may not have been obvious, but the uniqueness of $i_0 \in I$ for any $j_0 \in J$ (such that $j_0 \in J_{i_0}$) when constructing $g$ was critical for this proof.
    To see why, suppose that for some $j_0 \in J$ there are distinct $i_1$ and $i_2$ in $I$ such that $j_0 \in J_{i_1}$ and $j_0 \in J_{i_2}$.
    Then it could very well be that $a_{i_1 j_0} \neq a_{i_2 j_0}$ (though they would both be in $A_{j_0}$) and we would have to choose one to be $g(j_0)$.
    Supposing we choose $g(j_0) = a_{i_1 j_0}$ then we would have $a_{i_2}' = \angles{g(j) \where j \in J_{i_2}}$ so that $a_{i_2}' \neq b_{i_2}$ since $a_{i_2}'(j_0) = g(j_0) = a_{i_1 j_0} \neq a_{i_2 j_0} = b_{i_2}(j_0)$.
    If we had set $g(j_0) = a_{i_2 j_0}$ instead then, by the same argument, we would have $a_{i_1}' \neq b_{i_1}$.
    Clearly in either case this would break the proof since we would have $f(g) = \angles{a_i' \where i \in I} \neq \angles{b_i \where i \in I} = b$.
    In fact, in this case there would be no $g \in \prodj A_j$ such that $f(g) = b$ since we cannot choose a value for $g(j_0)$ for which $a_i' = b_i$ for all $i \in I$.
    
    Returning from our digression, we have shown that $f$ is a bijection from $\prodj A_j$ to $\prodi B_i$ so that
    \begin{gather}
      \abs{\prodj A_j} = \abs{\prodi B_i} \,. \label{eqn:infsp:pJpI}
    \end{gather}
    Therefore we have
    \ali{
      \prodi \parens{\prod_{j \in J_i} \k_j} &= \prodi \abs{B_i} & \text{(by \eqref{eqn:infsp:prodJi})} \\
      &= \abs{\prodi B_i} & \text{(by the definition of cardinal product)} \\
      &= \abs{\prodj A_j} & \text{(by \eqref{eqn:infsp:pJpI})} \\
      &= \prodj \k_j & \text{(by \eqref{eqn:infsp:prodJ})}
    }
    as desired.
  }
}

\exercise{8}{
  If $\k_i \leq \l_i$ for all $i \in I$, then
  \gath{
    \prodi \k_i \leq \prodi \l_i \,.
  }
}
\sol{
  \qproof{
    Suppose sets $\angles{A_i \where i \in I}$ and $\angles{B_i \where i \in I}$ where $\abs{A_i} = \k_i$ and $\abs{B_i} = \l_i$ for all $i \in I$.
    Then, by the definition of the cardinal product, we have that
    \ali{
      \prodi \k_i &= \abs{\prodi A_i} &
      \prodi \l_i &= \abs{\prodi B_i} \,.
    }
    Now, for any $i \in I$, we also have
    \gath{
      \abs{A_i} = \k_i \leq \l_i = \abs{B_i}
    }
    so that we can choose an injective $f_i : A_i \to B_i$ (with the help of the Axiom of Choice).

    We then construct a function $f$ from $\prodi A_i$ to $\prodi B_i$.
    So for any $a \in \prodi A_i$ we have that $a = \angles{a_i \where i \in I}$ where each $a_i \in A_i$.
    Thus $a_i \in A_i = \dom(f_i)$ for each $i \in I$ so that $f_i(a_i) \in B_i$.
    We then define $f(a) = \angles{f_i(a_i) \where i \in I}$ so that clearly $f(a) \in \prodi B_i$ and hence $f$ is a function into $\prodi B_i$.

    We claim that $f$ as defined above is injective.
    To this end, consider any $\a$ and $\b$ in $\prodi A_i$ where $\a \neq \b$.
    It then follows that $\a = \angles{\a_i \where i \in I}$ and $\b = \angles{\b_i \where i \in I}$ where both $\a_i$ and $\b_i$ are elements of $A_i$ for each $i \in I$.
    Since $\a \neq \b$ we must have that there is an $i_0 \in I$ such that $\a_{i_0} \neq \b_{i_0}$.
    It then follows that $f_{i_0}(\a_{i_0}) \neq f_{i_0}(\b_{i_0})$ since $f_{i_0}$ is injective.
    Therefore clearly $f(\a) = \angles{f_i(\a_i) \where i \in I} \neq \angles{f_i(\b_i) \where i \in I} = f(\b)$.
    This shows that $f$ is injective since $\a$ and $\b$ were arbitrary.

    Finally, since $f$ is an injective function from $\prodi A_i$ to $\prodi B_i$, we have that
    \gath{
      \prodi \k_i = \abs{\prodi A_i} \leq \abs{\prodi B_i} = \prodi \l_i
    }
    as desired.
  }
}

\def\prodn{\prod_{n=0}^\infty}
\exercise{9}{
  Find some cardinals $\k_n$, $\l_n$ ($n \in \nats$) such that $\k_n < \l_n$ for all $n$, but $\prodn \k_n = \prodn \l_n$.
}
\sol{
  Let $\k_n = 2$ and $\l_n = \al_0$ for all $n \in \nats$.
  We claims that these satisfy the requirements.
  \qproof{
    Clearly $\k_n = 2 < \al_0 = \l_n$ for every $n \in \nats$.
    However, by Exercise~9.1.10 and Theorem~5.2.2c, we then have
    \gath{
      \prodn \k_n = \prodn 2 = \prod_{n < \al_0} 2 = 2^{\al_0}
      = \al_0^{\al_0} = \prod_{n < \al_0} \al_0 = \prodn \al_0 = \prodn \l_n
    }
  }
}

\def\proda{\prod_{\a < \l}}
\exercise{10}{
  Prove that $\k \cdot \k \cdot \cdots \text{($\l$ times)} = \k^\l$.
}
\sol{
  \qproof{
    First, we clearly have that
    \gath{
      \k \cdot \k \cdot \cdots \text{($\l$ times)} = \proda \k \,.
    }
    So let $A$ and $B$ be sets such that $\abs{A} = \k$ and $\abs{B} = \l$.
    Then by definition we have $\proda \k = \abs{\proda A}$ and $\k^\l = \abs{A^B}$.
    We also have that there is a bijective $g: \l \to B$ since $\abs{\l} = \l = \abs{B}$.
    We construct a bijection $f$ from $A^B$ to $\proda A$.
    So, for any $h \in A^B$ let $f(h) = h \circ g$.
    Clearly, since $g: \l \to B$ and $h: B \to A$, it follows that $f(h) : \l \to A$ and hence clearly $f(h) \in \proda A$ (since $f(h)(\a) \in A$ for each $\a < \l$) so that $f$ is a function from $A^B$ to $\proda A$.

    We also claim that $f$ is injective.
    So consider any $h_1$ and $ h_2$ in $A^B$ where $h_1 \neq h_2$.
    It then follows that  there must be a $b \in B$ where $h_1(b) \neq h_2(b)$.
    Then let $\a = \inv{g}(b)$, noting that $\inv{g}$ is a bijective function from $B$ to $\l$ since $g$ is bijective.
    We then have that
    \ali{
      f(h_1)(\a) &= (h_1 \circ g)(\a) = h_1(g(\a)) = h_1(g(\inv{g}(b))) = h_1(b) \\
      &\neq h_2(b) = h_2(g(\inv{g}(b))) = h_2(g(\a)) = (h_2 \circ g)(\a) = f(h_2)(\a)
    }
    so that $f(h_1) \neq f(h_2)$.
    Since $h_1$ and $h_2$ were arbitrary this shows that $f$ is indeed injective.

    We also claim that $f$ is onto.
    So consider any $h' \in \proda A$ so that $h'$ is a function from $\l$ to $A$ (since $h'(\a) \in A$ for each $\a < \l$).
    Then let $h = h' \circ \inv{g}$ so that clearly $h$ is a function from $B$ to $A$ since $\inv{g} : B \to \l$ and $h' : \l \to A$.
    We then have that
    \gath{
      f(h) = h \circ g = (h' \circ \inv{g}) \circ g = h' \circ (\inv{g} \circ g) = h' \circ i_\l = h'
    }
    where $i_\l$ is the identity function from $\l$ to $\l$.
    Since $h'$ was arbitrary, this shows that $f$ is indeed onto.

    Thus, since we have shown that $f$ is a bijection, it follows that
    \gath{
      \k \cdot \k \cdot \cdots \text{($\l$ times)} = \proda \k = \abs{\proda A} = \abs{A^B} = \k^\l
    }
    as desired.
  }
}

\exercise{11}{
  Prove the formula $\parens{\prodi \k_i}^\l = \prodi \parens{\k_i}^\l$.
  [Hint: Generalize the proof of the special case $\parens{\k \cdot \l}^\mu = \k^\mu \cdot \l^\mu$, given in Theorem~1.7 of Chapter~5.]

  Note that the hint differs from that in the book; see the Errata List.
}
\sol{
  \qproof{
    First suppose that $\angles{A_i \where i \in I}$ are sets where $\abs{A_i} = \k_i$ for all $i \in I$.
    Also suppose that $B$ is a set such that $\abs{B} = \l$.
    It then follows that
    \gath{
      \parens{\prodi \k_i}^\l = \abs{\prodi A_i}^\l = \abs{\parens{\prodi A_i}^B}
    }
    and
    \gath{
      \prodi \k_i^\l = \prodi \abs{A_i^B} = \abs{\prodi A_i^B} \,.
    }
    We then aim to construct a bijection $F$ from $\prodi A_i^B$ to $\parens{\prodi A_i}^B$, which clearly would show the result since we would have by the above that
    \gath{
      \parens{\prodi \k_i}^\l = \abs{\parens{\prodi A_i}^B} = \abs{\prodi A_i^B} = \prodi \k_i^\l  \,.
    }

    So suppose that $f \in \prodi A_i^B$ so that $f_i = f(i) \in A_i^B$ for every $i \in I$.
    Then define a function $g$ such that, for any $b \in B$, $g(b) = \angles{f_i(b) \where i \in I}$.
    We then have that, for any $i \in I$, $f_i(b) \in A_i$ since $f_i$ is a function from $B$ to $A_i$.
    Therefore $g(b) \in \prodi A_i$, and hence $g$ is a function from $B$ to $\prodi A_i$ so that $g \in \parens{\prodi A_i}^B$.
    Naturally then we set $F(f) = g$ so that $F$ is indeed a function from $\prodi A_i^B$ to $\parens{\prodi A_i}^B$.

    To show that $F$ is injective consider any $f$ and $f'$ in $\prodi A_i^B$ where $f \neq f'$.
    It then follows that there is an $i_0 \in I$ such that $f_{i_0} = f(i_0) \neq f'(i_0) = f_{i_0}'$.
    Then, since $f_{i_0}$ and $f_{i_0}'$ are both in $A_{i_0}^B$, it follows that they are both functions from $B$ to $A_{i_0}$.
    Since $f_{i_0} \neq f_{i_0}'$ we have that there is a $b \in B$ such that $f_{i_0}(b) \neq f_{i_0}'(b)$.
    We then have that $g(b) = \angles{f_i(b) \where i \in I} \neq \angles{f_i'(b) \where i \in I} = g'(b)$ since $f_{i_0}(b) \neq f_{i_0}'(b)$ and $i_0 \in I$.
    Therefore $F(f) = g \neq g' = F(f')$, which shows that $F$ is injective since $f$ and $f'$ were arbitrary.

    To show that $F$ is also onto, consider any $g' \in \parens{\prodi A_i}^B$ so that $g'$ is a function from $B$ to $\prodi A_i$.
    Therefore, for any $b \in B$, $g'(b)$ is a function on $I$ such that $g'(b)(i) \in A_i$ for each $i \in I$.
    Clearly we have $g'(b) = \angles{g'(b)(i) \where i \in I}$.
    Then, for each $i \in I$, we define a function $f_i$ on $B$ such that $f_i(b) = g'(b)(i)$.
    Then, since $f_i(b) = g'(b)(i) \in A_i$, clearly each $f_i$ is a function from $B$ to $A_i$.
    Therefore $f_i \in A_i^B$.
    We then set $f = \angles{f_i \where i \in I}$ so that clearly $f \in \prodi A_i^B$.
    Now, we then set $g = F(f)$ so that, by the definition of $F$, $g$ is a function on $B$ such $g(b) = \angles{f_i(b) \where i \in I}$ for any $b \in B$.
    We then have that $g(b) = \angles{f_i(b) \where i \in I} = \angles{g'(b)(i) \where i \in I} = g'(b)$ by the definition of each $f_i$.
    Since $b \in B$ was arbitrary, this shows that $F(f) = g = g'$.
    Thus $F$ is onto since $g'$ was arbitrary.

    We have shown that $F$ is a bijection so that the result follows as described above.
  }
}

\exercise{12}{
  Prove the formula
  \gath{
    \prodi \parens{\k^{\l_i}} = \k^{\sumi \l_i} \,.
  }
  [Hint: Generalize the proof of the special case $\k^\l \cdot \k^\mu = \k^{\l + \mu}$ given in Theorem~1.7(a) of Chapter~5.]
}
\sol{
  \qproof{
    First, suppose that $A$ is a set such that $\abs{A} = \k$, and that $\angles{B_i \where i \in I}$ are mutually disjoint sets such that $\abs{B_i} = \l_i$ for each $i \in I$.
    Then, by the definitions of cardinal products and sums, we have
    \gath{
      \prodi \parens{\k^{\l_i}} = \prodi \abs{A^{B_i}} = \abs{\prodi A^{B_i}}
    }
    and
    \gath{
      \k^{\sumi \l_i} = \k^{\abs{\uni B_i}} = \abs{A^{\uni B_i}} \,.
    }
    We now construct a bijective function F from $\prodi A^{B_i}$ to $A^{\uni B_i}$, which clearly shows the desired result since we then have
    \gath{
      \prodi \parens{\k^{\l_i}} = \abs{\prodi A^{B_i}} = \abs{A^{\uni B_i}} = \k^{\sumi \l_i} \,.
    }

    So, for any $f = \angles{f_i \where i \in I} \in \prodi A^{B_i}$, we have that each $f_i$ is a function from $B_i$ to $A$.
    It then follows from the fact that $\angles{B_i \where i \in I}$ are mutually disjoint that $\braces{f_i}_{i \in I}$ is a compatible system of functions.
    We then that $g = \uni f_i$ is a function from $\uni \dom(f_i) = \uni B_i$ to $A$ by Theorem~2.3.12.
    Hence $g \in A^{\uni B_i}$, and we of course set $F(f) = g$ so that $F$ is a function from $\prodi A^{B_i}$ to $A^{\uni B_i}$.

    To show that $F$ is injective, consider any $f$ and $f'$ in $\prodi A^{B_i}$ where $f \neq f'$.
    Let $g = F(f)$ and $g' = F(f')$.
    It then follows that there is an $i_0 \in I$ such that $f_{i_0} \neq f_{i_0}'$.
    Then, it has to be that there is a $b \in B_{i_0}$ where $f_{i_0}(b) \neq f_{i_0}'(b)$.
    Clearly also $b \in \uni B_i$ since $i_0 \in I$ so that $b \in \dom(g)$ and $b \in \dom(g')$ (since by the definition of $F$ we have that $g$ and $g'$ are functions with domain $\uni B_i$).
    Moreover, we have that $g(b) = f_{i_0}(b)$ and $g'(b) = f_{i_0}'(b)$, which again follows from the fact that $\angles{B_i \where i \in I}$ are mutually disjoint or equivalently that $\braces{f_i}_{i \in I}$ is a compatible system.
    Hence we have $g(b) = f_{i_0}(b) \neq f_{i_0}'(b) = g'(b)$ so that $F(f) = g \neq g' = F(f')$.
    This shows that $F$ is injective since $f$ and $f'$ were arbitrary.

    To see that $F$ is onto, consider any $g' \in A^{\uni B_i}$.
    For each $i \in I$, define $f_i = g' \rest B_i$, which is clearly a function from $B_i$ to $A$ so that $f_i \in A^{B_i}$.
    Then let $f = \angles{f_i \where i \in I}$ so that $F(f) = g = \uni f_i$, and clearly $f \in \prodi A^{B_i} = \dom(F)$.
    Now consider any $(x,y) \in g = \uni f_i$ so that there is an $i_0 \in I$ where $(x,y) \in f_{i_0} = g' \rest B_{i_0}$.
    Therefore $(x,y) \in g'$ since obviously $g' \rest B_{i_0} \ss g'$.
    Consider next any $(x,y) \in g'$ so that $x \in \dom(g') = \uni B_i$ and $y = g'(x)$.
    Then there is an $i_0 \in I$ where $x \in B_{i_0}$ so that $(x,y) \in g' \rest B_{i_0} = f_{i_0}$.
    Hence clearly $(x,y) \in \uni f_i = g$ since $i_0 \in I$.
    Therefore $g \ss g'$ and $g' \ss g$ so that $F(f) = g = g'$, which shows that $F$ is onto since $g'$ was arbitrary.

    We have thus shown that $F$ is a bijection so that the result follows.
  }
}

\def\uni{\bigcup_{i \in I}}
\def\prodi{\prod_{i \in I}}
\exercise{13}{
  Prove that if $1 < \k_i \leq \l_i$ for all $i \in I$, then $\sumi \k_i \leq \prodi \l_i$.
}
\sol{
  \begin{lem}\label{lem:cardar:csucc}
    If $A$ is a set and $n < \abs{A}$ for a finite cardinal (i.e. natural number) $n$, then $n+1 \leq \abs{A}$.
  \end{lem}
  \qproof{
    Since $n < \abs{A}$, it follows that there is an injective function $f: n \to A$ but that $f$ cannot be onto.
    Thus there is an $a \in A$ such that $a \notin \ran(f)$.
    Noting that $n+1 = n \cup \braces{n}$, for any $k \in n+1$, define a mapping $g$ by
    \gath{
      g(k) = \begin{cases}
        f(k) & k \in n \\
        a & k = n \,.
      \end{cases}
    }
    Clearly in either of these cases we have that $g(k) \in A$ so that $g$ is in fact a function from $n+1$ into $A$.

    To see that $g$ is injective, consider any $k_1$ and $k_2$ in $n+1$ where $k_1 \neq k_2$.
    If $k_1 = n$ then $g(k_1) = a$ and it has to be that $k_2 \in n$ since $k_2 \neq k_1 = n$.
    Hence clearly $g(k_2) = f(k_2) \in \ran(f)$ whereas $g(k_1) = a \notin \ran(f)$ so that $g(k_1) \neq g(k_2)$.
    If $k_1 \in n$ but $k_2 = n$ then this is analogous the previous case, so assume that also $k_2 \in n$.
    Here clearly $g(k_1) = f(k_1) \neq f(k_2) = g(k_2)$ since $f$ is injective and $k_1 \neq k_2$.
    Therefore, in all cases we have that $g(k_1) \neq g(k_2)$, which shows that $g$ is injective since $k_1$ and $k_2$ were arbitrary.

    The existence of the injection $g$ thus shows that that $n+1 \leq \abs{A}$ as desired.
  }
  
  \mainprob

  \qproof{
    Suppose that $\angles{A_i \where i \in I}$ are mutually disjoint sets where $\abs{A_i} = \k_i$ for every $i \in I$, and that $\angles{B_i \where i \in I}$ are sets where $\abs{B_i} = \l_i$ for each $i \in I$.
    Since we have $\abs{A_i} = \k_i \leq \l_i = \abs{B_i}$ , we can assume that $A_i \ss B_i$ (for every $i \in I$).
    We show the result by constructing an injective function that maps $\uni A_i$ into $\prodi B_i$, since we clearly would then have
    \gath{
      \sum_{i \in I} \k_i = \abs{\uni A_i} \leq \abs{\prodi B_i} = \prodi \l_i
    }
    by the definitions of cardinal sum and product, which is the desired result.

    First, if $I = \es$ then we have that $\sum_{i \in I} \k_i = \abs{\bigcup_{i \in \es} A_i} = \abs{\es} = 0$.
    The only function with domain $I = \es$ is $\es$ itself so that $\prodi \l_i = \abs{\prod_{i \in \es} B_i} = \abs{\braces{\es}} = 1$.
    Hence clearly $\sumi \k_i = 0 \leq 1 = \prodi \l_i$ so that the hypothesis is true.
    So assume in what follows that $I \neq \es$ so that there is an $i_0 \in I$.

    Now, for every $i \in I$, since $1 < \k_i \leq \l_i = \abs{B_i}$, it follows from Lemma~\ref{lem:cardar:csucc} that $2 \leq \abs{B_i}$.
    Hence, for each $i \in I$ we can choose two distinct elements $\a_i$ and $\b_i$ from $B_i$, though this requires the Axiom of Choice.
    So, for any $x \in \uni A_i$, there is an $i_x \in I$ such that $x \in A_{i_x}$.
    Them, for any $i \in I$, set
    \gath{
      a_i = \begin{cases}
        x & i = i_x \\
        \a_i & i \neq i_x \text{ and } i = i_0 \text{ and } x = \a_{i_x} \\
        \b_i & i \neq i_x \text{ and } i = i_0 \text{ and } x \neq \a_{i_x} \\
        \b_i & i \neq i_x \text{ and } i \neq i_0 \text{ and } x = \a_{i_x} \\
        \a_i & i \neq i_x \text{ and } i \neq i_0 \text{ and } x \neq \a_{i_x}
      \end{cases}
    }
    and let $f(x) = \angles{a_i \where i \in I}$.
    Clearly, for any such $x$ and $i \in I$, we have $a_i = x \in A_{i_x} = A_i$ if $i = i_x$ so that $a_i \in B_i$ since $A_i \ss B_i$.
    In the other cases either $a_i = \a_i \in B_i$ or $a_i = \b_i \in B_i$.
    Therefore $f(x) = \angles{a_i \where i \in I} \in \prodi B_i$ so that $f$ is a function from $\uni A_i$ into $\prodi B_i$.

    We also claim that $f$ is injective.
    To see this, consider $x$ and $y$ in $\uni A_i$ where $x \neq y$, and let $i_x$ and $i_y$ be those elements of $I$ such that $x \in A_{i_x}$ and $y \in A_{i_y}$.
    Also let $f(x) = \angles{a_i \where i \in I}$ and $f(y) = \angles{b_i \where i \in I}$.

    The following involves a lot of messy case work, so we shall number the cases for easy reference:
    \begin{enumerate}
    \item Case: $i_x = i_y$.
      Then $a_{i_x} = x \neq y = b_{i_y} = b_{i_x}$.
    \item Case: $i_x \neq i_y$.
      \begin{enumerate}
      \item Case: $i_x \neq i_0$ and $i_y \neq i_0$.
        \begin{enumerate}
        \item Case: $a_{i_x} = x = \a_{i_x}$.
          \begin{enumerate}
          \item Case: $b_{i_y} = y = \a_{i_y}$.
            Then, since $i_x \neq i_y$ and $i_x \neq i_0$ and $y = \a_{i_y}$, we have $b_{i_x} = \b_{i_x} \neq \a_{i_x} = a_{i_x}$.
          \item Case: $b_{i_y} = y \neq \a_{i_y}$.
            Then, since $i_0 \neq i_x$ and $i_0 = i_0$ and $x = \a_{i_x}$, we have $a_{i_0} = \a_{i_0}$.
            Also, since $i_0 \neq i_y$ and $i_0 = i_0$ and $y \neq \a_{i_y}$, we have $b_{i_0} = \b_{i_0} \neq \a_{i_0} = a_{i_0}$.
          \end{enumerate}
        \item Case: $a_{i_x} = x \neq \a_{i_x}$.
          \begin{enumerate}
          \item Case: $b_{i_y} = y = \a_{i_y}$.
            This is analogous to case 2.a.i.B with $a_{i_x} = x$ and $b_{i_y} = y$ reversed so that again $a_{i_0} \neq b_{i_0}$.
          \item Case: $b_{i_y} = y \neq \a_{i_y}$.
            Then, since $i_x \neq i_y$ and $i_x \neq i_0$ and $y \neq \a_{i_y}$, we have $b_{i_x} = \a_{i_x} \neq a_{i_x}$.
          \end{enumerate}
        \end{enumerate}
      \item Case: $i_x \neq i_0$ and $i_y = i_0$.
        \begin{enumerate}
        \item Case: $a_{i_x} = x = \a_{i_x}$.
          \begin{enumerate}
          \item Case: $b_{i_y} = y = \a_{i_y}$.
            Since $i_x \neq i_y$ and $i_x \neq i_0$ and $y = \a_{i_y}$, we have $b_{i_x} = \b_{i_x} \neq \a_{i_x} = a_{i_x}$.
          \item Case: $b_{i_y} = y \neq \a_{i_y}$.
            Since $i_y \neq i_x$ and $i_y = i_0$ and $x = \a_{i_x}$, we have $a_{i_y} = \a_{i_y} \neq b_{i_y}$.
          \end{enumerate}
        \item Case: $a_{i_x} = x \neq \a_{i_x}$.
          \begin{enumerate}
          \item Case: $b_{i_y} = y = \a_{i_y}$.
            Since $i_y \neq i_x$ and $i_y = i_0$ and $x \neq \a_{i_x}$, we have $a_{i_y} = \b_{i_y} \neq \a_{i_y} = b_{i_y}$.
          \item Case: $b_{i_y} = y \neq \a_{i_y}$.
            Since $i_x \neq i_y$ and $i_x \neq i_0$ and $y \neq \a_{i_y}$, we have $b_{i_x} = \a_{i_x} \neq a_{i_x}$.
          \end{enumerate}
        \end{enumerate}
      \item Case: $i_x = i_0$ and $i_y \neq i_0$.
        This is analogous  to the previous case 2.b with the roles of $i_x$ and $i_y$ reversed.
      \item Case: $i_x = i_0$ and $i_y = i_0$.
        This is impossible since $i_x \neq i_y$.
      \end{enumerate}
    \end{enumerate}
    Thus, in all cases, there is an $i \in I$ such that $a_i \neq b_i$ so that clearly $f(x) = \angles{a_i \where i \in I} \neq \angles{b_i \where i \in I} = f(y)$.
    This shows that $f$ is injective since $x$ and $y$ were arbitrary.
    This completes the proof as described above.
  }
}

\def\prodI{\prod_{\a \in I}}
\def\prodJ{\prod_{\a \in J}}
\exercise{14}{
  Evaluate the cardinality of $\prod_{0 < \a < \w_1} \a$.
  [Answer: $2^{\al_1}$.]
}
\sol{
  \begin{lem} \label{lem:cardar:al2}
    If $\a$ and $\b$ are ordinals and $\a \leq \b$, then $\al_\a^{\al_\b} = 2^{\al_\b}$.
  \end{lem}
  \qproof{
    First we have that
    \gath{
      2^{\al_\b} \leq \al_\a^{\al_\b}
    }
    by property (n) after Lemma~5.6.1 since clearly $2 \leq \al_\a$.
    We also have
    \ali{
      \al_\a^{\al_\b} &\leq \parens{2^{\al_\a}}^{\al_\b} & \text{(again by property (n), and Theorems~5.1.8 and 5.1.9)} \\
      &= 2^{\al_\a \cdot \al_\b} & \text{(by Theorem~5.1.7b)} \\
      &= 2^{\al_\b} \,. & \text{(by Corollary~7.2.2 since $\a \leq \b$)}
    }
    Hence it follows from the \cbthrm{} that $\al_\a^{\al_\b} = 2^{\al_\b}$ as desired.
  }

  \mainprob
  
  We claim that $\abs{\prod_{0 < \a < \w_1} \a} = 2^{\al_1}$.
  \qproof{
    First, let $I = \braces{\a \where 0 < \a < \w_1}$ so that clearly $\prod_{0 < \a < \w_1} \a = \prodI \a$.
    It is trivial to show that the mapping
    \gath{
      f(\a) = \begin{cases}
        \a+1 & 0 \leq \a < \w_0 \\
        \a & \w_0 \leq \a < \w_1
      \end{cases}
    }
    for $\a \in \w_1$ is a bijection from $\w_1$ to $I$, and hence $\abs{I} = \abs{\w_1} = \al_1$.
    Also, since $\a < \w_1$ for any $\a \in I$, it follows from Lemma~\ref{lem:aleph:alephadd:alephlt} that $\abs{\a} \leq \al_0$.
    Therefore, we first have
    \ali{
      \abs{\prodI \a} &= \prodI \abs{\a} & \text{(by the definition of the cardinal product)} \\
      &\leq \prodI \al_0 & \text{(by Exercise~9.1.8 since $\abs{\a} \leq \al_0$ for all $\a \in I$)} \\
      &= \al_0^{\abs{I}} & \text{(by Exercise~9.1.10)} \\
      &= \al_0^{\al_1} & \text{(since we have shown that $\abs{I} = \al_1$)} \\
      &= 2^{\al_1} \,. & \text{(by Lemma~\ref{lem:cardar:al2})}
    }
    Now, let $J = \braces{\a \where 1 < \a < \w_1}$.
    For any $a = \angles{a_2, a_3, \ldots, a_\w, a_{\w+1},\ldots} = \angles{a_\a \where \a \in J} \in \prodJ \a$, it is easy to show that the function that maps this to $b = \angles{0, a_2, a_3, \ldots, a_\w, a_{\w+1}, \ldots} \in \prodI \a$ is bijective so that $\abs{\prodJ \a} = \abs{\prodI \a}$.
    It should also be clear that $\abs{J} = \abs{I} = \al_1$.
    Lastly, we clearly have $2 \leq \abs{a}$ for all $a \in J$ since $1 < \a$.
    We then have
    \ali{
      2^{\al_1} &= 2^{\abs{J}} & \text{(since $\abs{J} = \al_1$)} \\
      &= \prodJ 2 & \text{(by Exercise~9.1.10)} \\
      &\leq \prodJ \abs{\a} & \text{(by Exercise~9.1.8 since $2 \leq \abs{\a}$ for all $\a \in J$)} \\
      &= \abs{\prodJ \a} & \text{(by the definition of the cardinal product)} \\
      &= \abs{\prodI \a} & \text{(by what was shown above)}
    }
    It therefore follows from the \cbthrm{} that $\abs{\prod_{0 < \a < \w_1} \a} = \abs{\prodI \a} = 2^{\al_1}$ as desired.
  }
}

\exercise{15}{
  Justify the existence of the function $f$ in the proof of Lemma~9.1.2 in detail by the axioms of set theory.
}
\sol{
  First, for any $i \in I$, we know that $\abs{A_i} = \abs{A_i'}$ so that there exists a bijection from $A_i$ onto $A_i'$.
  It was shown in Exercise~2.3.9a that the set $A_i'^{A_i}$ exists so that $B_i = \braces{h_i \in A_i'^{A_i} \where \text{$h_i$ is a bijection}}$ exists by the Axiom Schema of Comprehension.
  Clearly then $B_i \neq \es$ for every $i \in I$.
  Since $B_i$ is uniquely defined for each $i \in I$, it follows from the Axiom Schemas of Replacement and Comprehension that the set $\braces{B_i \where i \in I}$ exists, which is a system of nonempty sets.
  It then follows from Exercise~2.3.9b that $\prod_{i \in I} B_i$ exists, and by the Axiom of Choice that there is an $F \in \prod_{i \in I} B_i$, i.e. $F = \angles{f_i \where i \in I}$ in the notation of the proof.
  Now $F$ is a function on $I$ in which $F(i) = f_i \in B_i$ for every $i \in I$.
  By an application of the Axiom Schema of Comprehension, $\ran(F) = \braces{f_i \where i \in I}$ exists, and we have that $f = \bigcup \ran(F)$, which exists by the Axiom of Union, noting that then $f = \bigcup_{i \in I} f_i$ as in the proof.
}
