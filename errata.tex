\documentclass{report}
\usepackage{amsmath}
\usepackage{amssymb}
\usepackage[hidelinks]{hyperref}

% Common sets
\def\nats{{\boldsymbol{N}}}
\def\ints{{\boldsymbol{Z}}}
\def\rats{{\boldsymbol{Q}}}
\def\prats{\rats^+}
\def\reals{{\boldsymbol{R}}}
\def\es{\varnothing}

% Common variables/symbols
\def\vphi{\varphi}
\def\a{\alpha}
\def\b{\beta}
\def\g{\gamma}
\def\d{\delta}
\def\e{\varepsilon}
\def\z{\zeta}
\def\k{\kappa}
\def\l{\lambda}
\def\n{\nu}
\def\r{\rho}
\def\s{\sigma}
\def\t{\tau}
\def\x{\xi}
\def\w{\omega}
\def\W{\Omega}
\def\al{\aleph}

% For set-buider notation
\def\where{\,|\,}

% Other stuff
\def\dom{\mathrm{dom}\,}
\def\ran{\mathrm{ran}\,}

% Cardinality shortcuts
\def\cnats{{\al_0}}
\def\ccont{{2^\cnats}}

% Equivalence classes
\newcommand\eclass[2]{\squares{#1}_{#2}}

% Shortcuts that make writing easier
\newcommand\parens[1]{\left( #1 \right)}
\newcommand\squares[1]{\left[ #1 \right]}
\newcommand\braces[1]{\left\{ #1 \right\}}
\newcommand\angles[1]{\left\langle #1 \right\rangle}
\newcommand\ceil[1]{\left\lceil #1 \right\rceil}
\newcommand\floor[1]{\left\lfloor #1 \right\rfloor}
\newcommand\abs[1]{\left| #1 \right|}
\newcommand\dabs[1]{\left\| #1 \right\|}
\newcommand\vect[1]{\mathrm{\mathbf{#1}}}
\newcommand\conj[1]{\overline{#1}}
\newcommand\pset[1]{\mathcal{P}\left(#1\right)}
\newcommand\inv[1]{#1^{-1}}
\newcommand\prop[1]{\mathbf{#1}}
\def\rest{\restriction}
\newcommand\tet[2]{{^{#1}#2}}

% Families of set
\def\famF{\mathcal{F}}

% These are needed so that half-open intervals do not cause auto-indentation issues due to unmatched brackets
\newcommand\clop[1]{[#1)}
\newcommand\ilab[1]{#1)}

% Other miscellaneous stuff
\def\ss{\subseteq}
\def\pss{\subset}
\def\Seq{\mathrm{Seq}}
\def\prece{\preccurlyeq}
\def\sd{\,\triangle\,}

% Environment shortcuts
\newcommand\gath[1]{\begin{gather*} #1 \end{gather*}}
\newcommand\ali[1]{\begin{align*} #1 \end{align*}}
\newcommand\qproof[1]{\begin{proof} #1 \end{proof}}

% Environment for indenting nested paragraphs (useful in case trees)
\newenvironment{indpar}
{
    \begin{adjustwidth}{1cm}{}
}{
    \end{adjustwidth}
}

% Exercise, Theorem, and solution shortcuts
\newcommand\exercise[2]{{
    \renewcommand\label[1]{} % Needed to suppress multiply-defined label warning since the same question numbers are used in different sections
    \setcounter{subsubsection}{#1-1} % Subsubsections are used to that lemmas have the full nested numbering
    \stepcounter{subsubsection} % Need to increment this so that the lemma counter gets reset
    \setcounter{question}{#1-1} % Manually set the question number since we always know the exercise number
    \question{#2} % The actual exam class question
}}
\newcommand\exerciseapp[3]{
    \setqf{#2}
    \exercise{#1}{#3}
    \setqf{}
}   
\newcommand\theorem[2]{{
    \renewcommand\label[1]{} % Needed to suppress multiply-defined label warning since the same question numbers are used in different sections
    \setcounter{subsubsection}{#1-1} % Subsubsections are used to that lemmas have the full nested numbering
    \stepcounter{subsubsection} % Need to increment this so that the lemma counter gets reset
    \setcounter{question}{#1-1} % Manually set the question number since we always know the theorem number
    \question{#2} % The actual exam class question
}}
\newcommand\theoremapp[3]{
    \setqf{#2}
    \theorem{#1}{#3}
    \setqf{}
}
\newcommand\sol[1]{\begin{solution} #1 \end{solution}}

% Main problem and theorem labels
\def\mainprob{\textbf{Main Problem.}}
\def\mainthrm{\textbf{Main Theorem.}}


\title{
  \booktitle \\
  \ \\
  Errata List
}

\begin{document}

\maketitle

\begin{enumerate}

\item Page 62, Exercise 3.5.4.
  The hint should read ``Let $h(x) = A - x$; \ldots'' instead of ``Let $h(x) = B - x$; \ldots''.
  (Confirmed by Dr. Jech)

\item Page 101, Exercise 5.2.3.
  This means that that the countable dense subset is dense in $P$ rather than simply dense with respect to itself.
  This has to be the case because, if were merely dense with respect to itself, then \emph{any} larger linearly ordered set containing the subset would have the same property so that it is impossible to put a bound on the size of that set (Confirmed)

\item Page 114, Exercise 6.3.5 part (c).
  This should read ``\ldots, then $f[A] \in V_\w$.'' instead of ``\ldots, then $f[X] \in V_\w$.'' since $X$ has not been previously defined.
  (Unconfirmed)

\item Page 140, bottom.
  The reference to Assumption~1.7 in Chapter~4 should really be Assumption~1.8.
  (Unconfirmed)

\item Page 142, bottom.
  The reference to Theorem~4.4 in Chapter~7 should be Theorem~4.4 in Chapter~6.
  (Unconfirmed)

\item Page 143 top.
  In the proof of Theorem~8.1.13 when showing that (c) implies (a) the sentence, ``Let $F$ be the system of all functions $f$ for which $\dom{f} \ss S$ and $f(X) \in X$ holds for any $X \in S$.'' should be, ``\ldots and $f(X) \in X$ holds for any $X \in \dom{f}$.''
  This is because, if $\dom{f} \pss S$, then there is an $X \in S$ where $X \notin \dom{f}$ so that $f(X)$ is not defined.
  (Unconfirmed)

\item Page 143, middle.
  In the proof of Theorem~8.1.14 there are two times in the first paragraph where statements are made for all or for some $\x \in A$.
  These should be $\x < \l$.
  (Unconfirmed)

\item Page 158, bottom.
  Near the end of the first paragraph of the proof of Theorem~9.1.7 (K{\"o}nig's Theorem), the sentence, ``If $i_x \neq i_y = i$, then $a_i = d_i \notin A$ while $b_i = y \in A$.'' should be, ``If $i_x \neq i_y = i$, then $a_i = d_i \notin A_i$ while $b_i = y \in A_i$.''
  (Unconfirmed)

\item Page 160, Exercise 9.1.11.
  The special case mentioned in the hint should be $\parens{\k \cdot \l}^\mu = \k^\mu \cdot \l^\mu$.
  This is evidenced by the fact that $\parens{\k^\mu}^\l = \parens{\k^\l}^\mu$ does not make sense in the context of the problem (i.e. it is not a special case) as well as by the fact that it is not part of Theorem~5.1.7.
  (Unconfirmed)

\end{enumerate}

\end{document}
