\question{6.1.1}

\begin{solution}
    Clearly $L = \reals$ is a linearly ordered set.
    So let
    $$
    S = \braces{x \in L \where x \leq 0} \,.
    $$
    Consider any $a \in S$ and any $x < a$ so that we have
    $$
    x < a \leq 0 \,.
    $$
    Hence $x \in S$ also so that by definition $S$ is an initial segment of $\reals$.
    Now suppose that $S$ does have the form
    $$
    S = \braces{x \in L \where x < a}
    $$
    for some $a \in L$.
    Since $0 \leq 0$ clearly $0 \in S$ by the original definition so that by the above $0 < a$.
    But now consider $a/2$, which is clearly in $L = \reals$.
    By the above $a/2 < a$ since $a > 0$ so $a/2 \in S$ but we also have $0 < a/2$ (hence it is not true that $a/2 \leq 0$) since $0 < a$ so that by the original definition $a/2 \notin S$.
    Since we have a contradiction it must be that $S$ cannot be expressed in such a form. \qedsymbol
\end{solution}

\question{6.1.2}

\begin{solution}
    Since $\w = \nats$ and $\w + 1 = \w \cup \braces{\w}$ clearly $\w$ is a proper subset of $\w + 1$ (since $\w \notin \w$ but $\w \in \w + 1$).
    Now consider any $a \in \w = \nats$ and any $x < a$.
    Then clearly also $x \in \nats$ so $x \in \w$.
    Thus $\w$ is an initial segment of $\w + 1$.
    Then, since it has already been shown that both $\w$ and $\w + 1$ are well-ordered sets, it follows from Corollary~6.1.5a that they  cannot be isomorphic.
\end{solution}

\question{6.1.3}

\begin{solution}
    \begin{statement}{Lemma 6.1.3.1}
        Suppose that $A$ is a subset of $\nats$ (including $A = \nats$).
        Then every initial segment of $A$ with the standard ordering is finite.
    \end{statement}

    \proof{
        Consider any initial segment $S$ of $(A, <)$.
        Then by Lemma~6.1.2 there is an $n \in A \ss \nats$ such that $S = \braces{k \in A \where k < n}$.
        So consider any $k \in S$ so that $k \in A \ss \nats$ and $k < n$.
        Then by the definition of $<$ we have that $k \in n$.
        Since $k$ was arbitrary this shows that $S \ss n$ so that $|S| \leq n$.
        From this it clearly follows that $S$ is finite since $n$ is. \qedsymbol
    }

    \mainprob

    Throughout the following let $<$ denote the standard well-ordering on $\nats$ and let $R$ be the set of all well-orderings defined on $\nats$.

    First we construct an injective $F: R \to \nats^\nats$.
    So for any $\prec \in R$ we have that $(\nats, <)$ and $(\nats, \prec)$ are two well-orderings of $\nats$.
    Consider then Theorem~6.1.3.
    We show that (c) cannot be the case, i.e. that an initial segment of $(\nats, <)$ cannot be isomorphic to $(\nats, \prec)$.
    So suppose that this is the case so that $f$ is an isomorphism from an initial segment $S$ of $(\nats, <)$ to $(\nats, \prec)$.
    Then by  Lemma~6.1.3.1 $S$ is finite whereas $\nats$ is infinite, but since $f$ is a bijection this is impossible since it would imply that $|S| = |\nats| = \N$.
    Hence it must be that (a) $(\nats, <)$ is isomorphic to $(\nats, \prec)$ or (b) the former is isomorphic to an initial segment of the latter.
    In either case such an isomorphism $f$ is unique by Corollary~6.1.5c.
    So define $F(\prec) = f$, noting that clearly $f \in \nats^\nats$.

    Now we show that $F$ is injective by considering two $\prec_1, \prec_2 \in R$ where $\prec_1 \neq \prec_2$.
    Without loss of generality we can the assume that there is an $(n,m) \in \prec_1$ where $(n,m) \notin \prec_2$.
    Thus $n \prec_1 m$ but since $\prec_2$ is a linear, strict ordering and $\lnot (n \prec_2 m)$ it has to be that $m \prec_2 n$ since $n \neq m$.
    Now let $f_1 = F(\prec_1)$ and $f_2 = F(\prec_2)$.
    Since both $f_1$ and $f_2$ are bijective there are $k_1,l_1,k_2,l_2 \in \nats$ such that
    \ali{
        f_1(k_1) &= n & f_2(k_2) &= n \\
        f_1(l_1) &= m & f_2(l_2) &= m
    }
    Since $f_1$ is an isomorphism and $f_1(k_1) = n \prec_1 m = f_1(l_1)$ it follows that
    $$
    k_1 < l_1
    $$
    and similarly since $f_2$ is an isomorphism and $f_2(l_2) = m \prec_2 n = f_2(k_2)$ it follows that
    $$
    l_2 < k_2 \,.
    $$
    Now we claim that either $f_1(k_1) \neq f_2(k_1)$ or $f_1(l_1) \neq f_2(l_1)$.
    Either case shows that $F(\prec_1) = f_1 \neq f_2 = F(\prec_2)$ so that $F$ is injective.
    To this end suppose that $f_1(k_1) = f_2(k_1) = n = f_2(k_2)$.
    Then since $f_2$ is injective it follows that $k_1 = k_2$.
    Hence with the above we have
    $$
    l_2 < k_2 = k_1 < l_1
    $$
    so that $m = f_2(l_2) \prec_2 f_2(l_1)$ and hence $m \neq f_2(l_1)$.
    Thus we have $f_1(l_1) = m \neq f_2(l_1)$ so that the disjunction is shown (since $\lnot P \to Q \equiv P \lor Q$) and $F$ is injective.

    Hence since $F: R \to \nats^\nats$ is injective we have that
    $$
    |R| \leq |\nats^\nats| = \N^\N = \NI
    $$
    by Theorem~5.2.2c.

    Now suppose that $B$ is the set of all bijections from $\nats$ to $\nats$.
    We then construct an injective $G: 2^\nats \to B$.
    So for any infinite sequence $a \in 2^\nats$ we define an $f \in \nats^\nats$ by
    \ali{
        f(2n) &= \begin{cases}
           2n & a_n = 0 \\
           2n+1 & a_n = 1
        \end{cases}
        &
        f(2n+1) &= \begin{cases}
           2n+1 & a_n = 0 \\
           2n & a_n = 1 \,.
        \end{cases}
    }
    for $n \in \nats$, i.e we swap $2n$ and $2n+1$ if $a_n = 1$ and leave them alone if $a_n = 0$.
    We then assign $G(a) = f$.
    It is trivial but tedious to show that $f$ is bijective so that indeed $f \in B$.

    Now consider any $a, b \in 2^\nats$ where $a \neq b$ and let $f = G(a)$ and $g = G(b)$.
    Since $a \neq b $ there is an $n \in \nats$ where $a_n \neq b_n$.
    Without loss of generality we can assume that $a_n = 0 \neq 1 = b_n$.
    Then we have
    $$
    f(2n) = 2n \neq 2n+1 = g(2n)
    $$
    since $a_1 = 0$ but $b_n = 1$.
    Hence $f \neq g$ so that $G$ is injective, from which it follows that $|2^\nats| \leq |B|$.

    Lastly we construct an injective $H : B \to R$.
    So for an $f \in B$ define
    $$
    \prec = \braces{(f(n), f(m)) \where (n,m) \in \nats \times \nats \land n < m}
    $$
    and set $H(f) = \prec$.
    Clearly by definition since $f$ is bijective it is an isomorphism from $(\nats, <)$ to $(\nats, \prec)$.
    This means that $(\nats, \prec)$ is isomorphic to $(\nats, <)$ so that clearly $\prec$ is a well-ordering since $<$ is.
    Hence indeed $H(f) = \prec \in R$.

    Now we show that $H$ is injective.
    So consider $f_1, f_2 \in B$ where $\prec_1 = H(f_1) = H(f_2) = \prec_2$.
    Then $\inv{f_1} \circ f_2$ is an isomorphism from $(\nats, \prec_1)$ to $(\nats, \prec_2)$
    But since $\prec_1 = \prec_2$ these are the same well-ordered set so that it follows from Corollary~6.1.5b that the only isomorphism between them is the identity $i_\nats$.
    Hence $\inv{f_1} \circ f_2 = i_\nats$, from which it follows that $f_1 = f_2$.
    Therefore $H$ is injective so that $|B| \leq |R|$.

    Putting this together results in
    $$
    \NI = |2^\nats| \leq |B| \leq |R| \,.
    $$
    It then follows from the Cantor-Bernstein Theorem that $|R| = \NI$ as desired. \qedsymbol
    
\end{solution}

\question{6.1.4}

\begin{solution}
	Let $A$ be an infinite subset of $\nats$.
    Then $(A, <)$ (where $<$ is the standard well-ordering of $\nats$) is a well-ordering since any $B \ss A$ is also a subset of $\nats$ and therefore has a least element.
    Hence by Theorem~6.1.3 either:
    \begin{enumerate}
        \item $(A,<)$ and $(\nats,<)$ are isomorphic, \\
        \item An initial segment of $(A,<)$ is isomorphic to $(\nats,<)$, or \\
        \item $(A,<)$ is isomorphic to an initial segment of $(\nats,<)$
    \end{enumerate}
    We show that they must be isomorphic (1) by showing that (2) and (3) lead to contradictions.

    Suppose (2), i.e. that an initial segment $S$ of $(A,<)$ is isomorphic to $(\nats,<)$.
    Then since $A \ss \nats$ it follows from Lemma~6.1.3.1 that $S$ is finite.
    But since this is isomorphic to $\nats$ it means that $|S| = |\nats| = \N$, which is a contradiction!

    Now suppose (3) so that $(A,<)$ is isomorphic to an initial segment $S$ of $(\nats, <)$.
    Again Lemma~6.1.3.1 tells us that $S$ is finite whereas $A$ is infinite.
    But since they are isomorphic this implies that $|S| = |A| = \N$, which is again a contradiction!

    Hence it has to be that $(A,<)$ and $(\nats,<)$ are isomorphic. \qedsymbol
\end{solution}

\question{6.1.5}

\begin{solution}
	Suppose that $(W, \prec)$ is the sum and associated order as defined in Lemma~4.4.5.
    By that lemma $\prec$ is a linear ordering but we must show that it is also a well-ordering.

    First we note that clearly $W_1$ and $W_2$ are both well-orderings since they are both isomorphic to $(\nats, <)$.
    So consider any non-empty subset of $A$ of $W = W_1 \cup W_2$.
    Let $A_1 = A \cap W_1$ and $A_2 = A \cap W_2$ so that clearly they are disjoint since $W_1$ and $W_2$ are and $A_1 \ss W_1$ and $A_2 \ss W_2$.
    Also since $A$ is not empty either $A_1$ or $A_2$ (or both) are also not empty.
    If $A_1$ is not empty then since $A_1 \ss W_1$ and $(W_1, <_1)$ is a well-ordering there is a least element $a \in A_1$.
    Otherwise if $A_1$ is empty then $A_2$ is not and it has a least element $a$ since it is a non-empty subset of the well-ordered $(W_2, <_2)$.
    Now consider any $b \in A$ so that also $b \in W$.
    If $b \in W_1$ then $b \in A_1$ so that $A_1$ is not empty.
    In this case since $a$ is the least element of $A_1$ we have $a \leq_1 b$ so that by definition $a \preceq b$.
    On the other hand if $b \in W_2$ then $b \in A_2$.
    If $A_1$ was empty then $a$ is the least element of $A_2$ and $b \in A_2$ so that again $a \leq_2 b$, hence by definition $a \preceq b$.
    If $A_1$ is not empty then $a \in A_1 \ss W_1$ so that by the definition of the sum $(W, \prec)$ we have that $a \prec b$ since $b \in W_2$.
    Hence also $a \preceq b$.
    Thus in all cases $a \preceq b$ so that $a$ is the least element of $A$ since $b$ was arbitrary.

    Now we show that $(W, \prec)$ is isomorphic to $(\w + \w, <)$.
    First, since $(W_1, <_1)$ and $(W_2, <_2)$ are both isomorphic to $(\nats, <)$ let $f_1 : W_1 \to \nats$ and $f_2 : W_2 \to \nats$ be isomorphisms.
    Now we define $g : W \to \w + \w$ by
    $$
    g(w) = \begin{cases}
         f_1(w) & w \in W_1 \\
         \w + f_2(w) & w \in W_2
    \end{cases}
    $$
    for $w \in W = W_1 \cup W_2$, noting that $g$ is well defined since $W_1$ and $W_2$ are disjoint.
    Clearly since $\ran(f_1) = \ran(f_2) = \nats$ we have that $g(w) \in \w + \w$ for all $w \in W$.

    Consider any $k \in \w + \w$ so that $k \in \nats$ or $k = \w + n$ for some $n \in \nats$.
    In the former case let $w = \inv{f_1}(k)$, which exists since $f_1$ is bijective.
    Thus $w \in W_1$ so that by definition $g(w) = f_1(w) = k$.
    In the latter case let $w = \inv{f_2}(n)$, which exists since $f_2$ is bijective.
    Thus $w \in W_2$ so that by definition $g(w) = \w + f_2(w) = \w + n = k$.
    This shows that $g$ is surjective.

    Now we show that $g$ is an increasing function.
    So consider any $w_1, w_2 \in W$ where $w_1 \prec w_2$.

    Case: $w_1, w_2 \in W_1$.
    Then since $w_1 \prec w_2$ we have that $w_1 <_1 w_2$.
    It then follows that $g(w_1) = f(w_1) < f_(w_2) = g(w_2)$ since $f_1$ is an isomorphism.

    Case: $w_1, w_2 \in W_2$.
    Then since $w_1 \prec w_2$ we have that $w_1 <_2 w_2$.
    It then follows that $f_2(w_1) < f_2(w_2)$ since $f_2$ is an isomorphism.
    Hence we clearly then have $g(w_1) = \w + f_2(w_1) < \w + f_2(w_2) = g(w_2)$.

    Case: $w_1 \in W_1$ and $w_2 \in W_2$.
    Then we have that $g(w_1) = f_1(w_1) \in \nats$ and $g(w_2) = \w + f_2(w_2)$ so that clearly $g(w_1) < \w \leq g(w_2)$ since $f_2(w_2) \in \nats$.

    Case: $w_2 \in W_1$ and $w_1 \in W_2$.
    If this were the case then by the definition of $\prec$ we would have that $w_2 \prec w_1$, which contradicts the established hypothesis that $w_1 \prec w_2$.
    Hence this case is impossible.

    Hence in all cases $g(w_1) < g(w_2)$ so that $g$ is increasing.
    Therefore it is also injective and an isomorphism (since we've shown that it is surjective as well).
    Thus we've shown that $W$ is isomorphic to $\w + \w$ as desired. \qedsymbol
\end{solution}

\question{6.1.6}

\begin{solution}
	Suppose that $\prec$ is the lexicographic ordering of $\nats \times \nats$.
    Now we define $f: \nats \times \nats \to \w \cdot \w$ by
    $$
    f(n,m) = \w \cdot n + m
    $$
    for any $(n,m) \in \nats \times \nats$.
    Clearly $f(n,m) \in \w \times \w$.

    First we show that $f$ is surjective.
    So consider any $k \in \w \cdot \w$ so that there are $n,m \in \nats$ where $k = \w \cdot n + m$.
    Then we clearly have that $f(n,m) = \w \cdot  n + m = k$.
    Since clearly $(n,m) \in \nats \times \nats$ it follows that $f$ is surjective.

    Now we show that $f$ is an increasing function.
    To this end consider any $(n_1, m_1), (n_2, m2) \in \nats \times \nats$ where $(n_1, m_1) \prec (n_2, m_2)$.

    Case: $n_1 = n_2$.
    Then since $(n_1, m_1) \prec (n_2, m_2)$ it must be that $m_1 < m_2$.
    Hence we have that $f(n_1, m_1) = \w \cdot n_1 + m_1 = \w \cdot n_2 + m_1 < \w \cdot n_2 + m_2 = f(n_2, m_2)$.

    Case: $n_1 \neq n_2$.
    Then since $(n_1, m_1) \prec (n_2, m_2)$ it must be that $n_1 < n_2$.
    Hence we have that $f(n_1, m_1) = \w \cdot n_1 + m_1 < \w \cdot n_2 \leq \w \cdot n_2 + m_2 = f(n_2, m_2)$.

    Thus in all cases $f(n_1, m_1) < f(n_2, m_2)$ so that $f$ is increasing.
    It then follows that $f$ is injective and isomorphic.
    Hence $(\nats \times \nats, \prec)$ is isomorphic to $\w \cdot \w$. \qedsymbol
\end{solution}

\question{6.1.7}

\begin{solution}
	NOTE: This problem is looking ahead to future sections where order types and how to compare them are defined.

    Suppose that $\a$ is the order type of $W$ and that $f : W \to \a$ is the isomorphism.
    Now let $\b = S(\a) = \a \cup \braces{\a}$.
    We then claim that $W'$ is isomorphic to $\b$.
    So define a $g : W' \to \b$ by
    $$
    g(w) = \begin{cases}
         f(w) & w \in W \\
         \a & w \notin W
    \end{cases}
    $$
    for $w \in W'$.
    Clearly we have that $g(w) \in \b$ for any $w \in W'$.

    Now consider any $x \in \b$.
    If $x = \a$ then set $w = a \notin W$ so that $g(w) = \a = x$.
    If $x \neq \a$ then $x \in \a$ so set $w = \inv{f}(x)$ so that then $w \in W$.
    We then have that $g(w) = f(w) = f(\inv{f}(x)) = x$.
    Therefore $g$ is surjective.

    Now consider any $w_1, w_2 \in W'$ where $w_1 < w_2$

    Case: $w_1, w_2 \in W$.
    Then since $f$ is an isomorphism and $w_1 < w_2$ we have that $g(w_1) = f(w_1) < f_(w_2) = g(w_2)$.

    Case: $w_1 \in W$ and $w_2 = a$.
    Then $g(w_1) = f(w_1) \in \a$ and $w_2 \notin W$ so that $g(w_2) = \a$.
    Hence $g(w_1) \in g(w_2)$ so that by the definition of $<$ we have that $g(w_1) < g(w_2)$.

    Note that these cases are exhaustive since it can't be that $w_1 = w_2 = \a$ since $w_1 < w_2$ (and therefore $w_1 \neq w_2$).
    It also cannot be that $w_2 \in W$ but $w_1 = a$ since then it would be that $w_2 \leq w_1$ since $a$ is the greatest element of $W'$, which contradicts $w_1 < w_2$.
    Thus in all cases $g(w_1) < g(w_2)$ so that $g$ is increasing, and therefore injective and an isomorphism.

    Hence $\b$ is the order type of $W'$, $\a$ is the order type of $W$, and $\a < \b$ since $\a \in \b$. \qedsymbol
\end{solution}

\question{6.1.8}

\begin{solution}
	Let $\prec$ be the lexicographic ordering of $W = \nats \times \braces{0,1}$ and $\prec'$ be the lexicographic ordering of $W' = \braces{0,1} \times \nats$.

    First we define $f : W \to \w$ by
    $$
    f(n,m) = 2n + m
    $$
    for $(n, m) \in W$.
    Clearly each $f(n,m) \in \nats = \w$.

    Now consider any $k \in \w = \nats$.
    If $k$ is even then $k = 2n$ for some $n \in \nats$ so set $w = (n, 0) \in W$.
    Then clearly $f(w) = f(n,0) = 2n = k$.
    On the other hand if $k$ is even then $k = 2n + 1$ for some $n \in \nats$ so set $w = (n,1) \in W$.
    Then clearly $f(w) = f(n,1) = 2n+1 = k$.
    This shows that $f$ is surjective.

    Now consider any $w_1 = (n_1,m_1)$ and $w_2 = (n_2, m_2)$ in $W$ where $w_1 \prec w_2$.

    Case: $n_1 = n_2$.
    Then since $w_1 \prec w_2$ it has to be that $m_1 < m_2$, and since $m_1,m_2 \in \braces{0,1}$ it has to be that $m_1=0$ and $m_2 = 1$.
    From this it follows that
    $$
    f(w_1) = f(n_1,m_1) = f(n_1,0) = 2n_1 < 2n_1 + 1 = 2n_2 + 1 = f(n_2,1) = f(n_2,m_2) = f(w_2) \,.
    $$

    Case: $n_1 \neq n_2$.
    Then since $w_1 \prec w_2$ it has to be that $n_1 < n_2$.
    Then $n_1 + 1 \leq n_2$ and since also $m_1 < 2$ we have
    $$
    f(w_1) = f(n_1, m_1) = 2n_1 + m_1 < 2n_1 + 2 = 2(n_1 + 1) \leq 2n_2 \leq 2n_2 + m_2 = f(n_1, m_2) = f(w_2) \,.
    $$
    Hence in all cases $f(w_1) < f(w_2)$ so that $f$ is increasing and therefore injective and isomorphic.
    Therefore $W$ is isomorphic to $\w$.

    Now we define $g: W' \to \w+\w$ by
    $$
    g(n,m) = \begin{cases}
    	m & n = 0 \\
        \w + m & n = 1
    \end{cases}
    $$
    for $(n,m) \in W'$.
    Clearly since $m \in \nats$ we have that $g(n,m) \in \w+\w$ for all $(n,m) \in W'$.

    Now consider any $\a \in \w + \w$.
    If $\a \in \w = \nats$ then $(0, \a) \in W'$ and $g(0,\a) = \a$.
    On the other hand if $\a = \w + m$ for some $m \in \nats$ then $(1, m) \in W'$ and $g(1,m) = \w + m = \a$.
    Therefore $g$ is surjective.

    Now consider any $w_1 = (n_1,m_1)$ and $w_2 = (n_2, m_2)$ in $W'$ where $w_1 \prec' w_2$.

    Case: $n_1 = n_2$.
    Then since $w_1 \prec' w_2$ it has to be that $m_1 < m_2$.
    If $n_1 = n_2 = 0$ then
    $$
    g(w_1) = g(n_1,m_1) = g(0,m_1) = m_1 < m_2 = g(0, m_2) = g(n_2,m_2) = g(w_2) \,.
    $$
    On the other hand if $n_1 = n_2 = 1$ then
    $$
    g(w_1) = g(n_1,m_1) = g(1,m_1) = \w + m_1 < \w + m_2 = g(1, m_2) = g(n_2,m_2) = g(w_2) \,.
    $$

    Case: $n_1 \neq n_2$.
    Then since $w_1 \prec' w_2$ it has to be that $n_1 < n_2$.
    Moreover since $n_1,n_2 \in \braces{0,1}$ it has to be that $n_1 = 0$ and $n_2 = 1$ so that
    $$
    g(w_1) = g(n_1,m_1) = g(0,m_1) = m_1 < \w + m_2 = g(1,m_2) = g(n_2,m_2) = g(w_2) \,.
    $$

    Hence in all cases $g(w_1) < g(w_2)$ so that $g$ is increasing and therefore injective and an isomorphism.
    Therefore $W'$ is isomorphic to $\w + \w$.

    Now, since $w \in \w+\w$ we have that $\w < \w+\w$ and so are distinct ordinals.
    Therefore by the remarks following Theorem~6.2.10 $\w$ and $\w+\w$ are not isomorphic.
    If $W$ and $W'$ were isomorphic with $h$ as the isomorphism then $g \circ h \circ \inv{f}$ would be an isomorphism from $\w$ to $\w + \w$, which is impossible.
    So it must be that $W$ and $W'$ are not isomorphic. \qedsymbol
\end{solution}

\question{6.2.1}

\begin{solution}
	($\to$) Suppose that $X$ is a transitive set and consider any $x \in X$.
    Then $x \ss X$ since $X$ is transitive.
    Thus $x \in \pset{X}$ so that since $x$ was arbitrary $X \ss \pset{X}$.

    ($\leftarrow$) Now suppose that $X \ss \pset{X}$ and consider any $x \in X$.
    Then also $x \in \pset{X}$ so that $x \ss X$.
    Hence since $x$ was arbitrary $X$ is transitive by definition. \qedsymbol
\end{solution}

\question{6.2.2}

\begin{solution}
    ($\to$) Suppose that $X$ is transitive and consider any $y \in \bigcup X$.
    Then there is an $x \in X$ such that $y \in x$.
    Since $X$ is transitive and $x \in X$ we have that $x \ss X$ so that $y \in X$ as well.
    Since $y$ was arbitrary this shows that $\bigcup X \ss X$.

    ($\leftarrow$) Now suppose that $\bigcup X \ss X$ and consider any $x \in X$.
    If $x = \es$ then clearly $x \ss X$.
    So suppose that $x \neq \es$ and consider any $y \in x$.
    Then since $x \in X$ it follows that $y \in \bigcup X$ so that also $y \in X$.
    So since $y$ was arbitrary it follows that $x \ss X$.
    Since $x$ was arbitrary by definition $X$ is transitive. \qedsymbol
\end{solution}

\question{6.2.3}

\begin{solution}
	(a) Let $X = \braces{\es, \braces{\es}, \braces{\braces{\es}}}$.
    Suppose $x \in X$.
    If $x = \es$ then obviously $x \ss X$.
    If $x = \braces{\es}$ then $x \ss X$ since $\es \in X$.
    If $x  = \braces{\braces{\es}}$ then $x \ss X$ since $\braces{\es} \in X$.
    Thus since the cases are exhaustive we've shown that $x \ss X$ so that $X$ is transitive by definition. \qedsymbol

    (b) Let $X = \braces{\es, \braces{\es}, \braces{\braces{\es}}, \braces{\es, \braces{\es}}}$.
    For $x \in X$ the three cases above have the same results and if $x = \braces{\es, \braces{\es}}$ then $x \ss X$ since $\es \in X$ and $\braces{\es} \in X$.
    Hence again $X$ is transitive by definition. \qedsymbol

    (c) Let $X = \braces{\es, \braces{\braces{\es}}}$.
    Then if $x = \braces{\braces{\es}}$ we have that $x$ is not a subset of $X$ since $\braces{\es} \in x$ but $\braces{\es} \notin X$.
    Hence $X$ is \emph{not} transitive. \qedsymbol
\end{solution}

\question{6.2.4}

\begin{solution}
	(a) Consider any $x \in X \cup Y$.
    If $x \in X$ then $x \ss X$ since $X$ is transitive.
    Since also $X \ss X \cup Y$ we clearly have that $x \ss X \ss X \cup Y$.
    We can make the same argument if it is the case that $x \in Y$.
    Hence since $x$ was arbitrary this shows that $X \cup Y$ is indeed transitive. \qedsymbol

    (b) Consider any $x \in X \cap Y$.
    Then $x \in X$ and $x \in Y$.
    Since $X$ and $Y$ are transitive this means that $x \ss X$ and $x \ss Y$.
    So consider any $y \in x$ then $y \in X$ and $y \in Y$ so that $y \in X \cap Y$.
    Hence since $y$ was arbitrary it follows that $x \ss X \cap Y$ so that $X \cap Y$ is transitive since $x$ was arbitrary. \qedsymbol

    (c) It was shown in Exercise~6.2.3 part (a) that $Y = \braces{\es, \braces{\es}, \braces{\braces{\es}}}$ is transitive.
    So let $X = \braces{\braces{\es}}$ so that clearly $X \in Y$.
    If then $x = \braces{\es}$ then $x \in X$ but $x$ is not a subset of $X$ since $\es \notin X$.
    Hence the original hypothesis is not true. \qedsymbol

    (d) Again $Y = \braces{\es, \braces{\es}, \braces{\braces{\es}}}$ is transitive so let $X = \braces{\braces{\es}, \braces{\braces{\es}}}$ so that clearly $X \ss Y$.
    Then if $x = \braces{\es}$ then $x \in X$ but $x$ is not a subset of $X$ because $\es \notin X$.
    Thus the original hypothesis is false. \qedsymbol

    (e) Consider any $x \in Y \cup S$.
    If $x \in Y$ then since $Y$ is transitive $x \ss Y$.
    Hence $x \ss Y \ss Y \cup S$.
    On the other hand if $x \in S$ then $x \in \pset{Y}$ since $S \ss \pset{Y}$
    Hence $x \ss Y \ss Y \cup S$.
    Since in all cases $x \ss Y \cup S$ and $x$ was arbitrary this shows that $Y \cup S$ is transitive by definition. \qedsymbol
\end{solution}

\question{6.2.5}

\begin{solution}
	Consider any $x \in \bigcup S$.
    Then there is an $X \in S$ where $x \in X$.
    Since $X$ is transitive it follows that $x \ss X$.
    So consider any $y \in x$ so that also $y \in X$.
    Thus also $y \in \bigcup S$ since $X \in S$.
    Since $y$ was arbitrary this shows that $x \ss \bigcup S$.
    Since $x$ was arbitrary this shows by definition that $\bigcup S$ is transitive. \qedsymbol
\end{solution}

\question{6.2.6}

\begin{solution}
	($\to$) Suppose that $n$ is a natural number and consider any nonempty subset $A$ of $n$.
    Since $A \ss n$ it follows that $|A| \leq |n| = n$ so that $A$ is finite.
    Thus $A$ is a finite set of natural numbers and so has a greatest element.
    This can be proven by a trivial inductive argument.

    ($\leftarrow$) We show this by contrapositive.
    Suppose that $\a$ is an ordinal such that $\a \notin \nats$.
    Then $\a \notin \w = \nats$ so that $\a \nless \w$, from which it follows that $\a \geq \w$
    Hence $\a = \w$ or $\a > \w$, in which case $\w \in \a$ so that $\w \ss \a$ since $\a$ is transitive.
    Thus in either case $\nats = \w \ss \a$.
    Clearly $\nats$ has no greatest element (since if $n$ were such a greatest element then $n+1 \in \nats$ but $n < n+1$).
    Thus there is a nonempty subset $A$ of $\a$ such that $A$ has no greatest element.
\end{solution}

\question{6.2.7}

\begin{solution}

    \begin{statement}{Lemma 6.2.7.1}
        If $\a$ and $\b$ are ordinals and $\a < \b$ then $\a+1 \leq \b$.
    \end{statement}

    \proof{
        To the contrary, suppose that $\a+1 > \b$.
        Then by the definition of $<$ we have that $\b \in \a+1 = \a \cup \braces{\a}$ and since $\b \neq \a$ it has to be that $\b \in \a$.
        But then $\b < \a$, which is a contradiction. \qedsymbol
    }

    \begin{statement}{Lemma 6.2.7.2}
        If $\a$ and $\b$ are ordinals and $\a < \b+1$ then $\a \leq \b$.
    \end{statement}

    \proof{
        Since $\a < \b+1$ we have that $\a \in \b+1 = \b \cup \braces{\b}$.
        Hence $\a \in \b$ or $\a = \b$.
        Thus $\a < \b$ or $\a = \b$, i.e. $\a \leq \b$. \qedsymbol
    }

    \mainprob
    
    Suppose that $X$ is a set of ordinals with no greatest element.
    Let $\b = \sup{X} = \bigcup X$.
    Then by the remarks following the proof of Theorem~6.2.6 $\b \notin X$ since $X$ has no greatest element.
    Now also suppose that $\b$ is a successor so that there is an ordinal $\a$ such that $\b = \a+1$.

    If $\a \in X$ then since $X$ has no greatest element there is a $\g \in X$ such that $\a < \g$.
    Then by Lemma~6.2.7.1 $\b = \a + 1 \leq \g$.
    It cannot be that $\g = \b$ since $\g \in X$ but $\b \notin X$ so it must be that $\b < \g$.
    But then since $\b$ is an upper bound of $X$ it follows that $\g$ is also.
    However, since $\g \in X$ this would make $\g$ the greatest element of $X$, which is a contradiction.

    On the other hand if $\a \notin X$ then consider any $\g \in X$.
    Then $\g < \b = \a+1$ so that by Lemma~6.2.7.2 $\g \leq \a$.
    Since $\g$ was arbitrary this shows that $\a$ is an upper bound of $X$.
    However, since $\a < \b$ this contradicts the definition of $\b$ as being the least upper bound of $X$, according to which $\a \geq \b$.

    Since all cases lead to a contradiction it cannot be that $\b = \sup{X}$ is a successor and therefore by definition is a limit ordinal. \qedsymbol
\end{solution}

\question{6.2.8}

\begin{solution}
	Suppose that $X$ is a set of ordinals.
    Then by Theorem~6.2.6d $X$ has a least element $\a$.
    We shall show that $\a = \bigcap X$, which simultaneously shows that $\bigcap X$ is an ordinal and the least element of $X$.

    Consider any $\b \in X$.
    Since $\a$ is the least element $\a \leq \b$ so that $\a = \b$ or $\a < \b$.
    Clearly $\a \ss \a = \b$ in the former case.
    In the latter case we have $\a \in \b$ so that $\a \ss \b$ as well since $\b$ is transitive (since it is an ordinal).
    Since $\b$ was arbitrary any $x \in \a$ is also in every $\b \in X$ so that $x \in \bigcap X$ so that $\a \ss \bigcap X$ since $x$ was arbitrary.

    Now consider any $x \in \bigcap X$.
    Then clearly $x \in \a$ since $\a \in X$ so that $\bigcap X \ss \a$ since $x$ was arbitrary.

    Thus we have shown that $\a = \bigcap X$ as desired. \qedsymbol
\end{solution}

\question{6.3.1}

\begin{solution}
	Define a property $\prop{R}(x,y)$ such that $\prop{R}(x,y)$ holds if and only if
    \begin{enumerate}
        \item $\prop{P}(x,y)$ holds, or
        \item $y = \es$ and there is not a $z$ such that $\prop{P}(x,z)$ holds.
    \end{enumerate}
    Clearly this property is such that for every $x$ there is a unique $y$ for which $\prop{R}(x,y)$ holds.

    Now consider any set $A$.
    Then by the Axiom Schema of Replacement there is a set $B$ such that, for every $x \in A$, there is a $y \in B$ for which $\prop{R}(x,y)$ holds.
    Consider any $x \in A$.
    Then by the above there is a $y \in B$ such that $\prop{R}(x,y)$ holds.
    Now suppose that $\prop{P}(x,z)$ holds for some $z$.
    Then option 2 above cannot by the case so that, $\prop{P}(x,y)$ holds (option 1) since $\prop{R}(x,y)$ does.
    Thus $\prop{P}(x,y)$ holds for some $y \in B$ as we were required to show. \qedsymbol
\end{solution}

\question{6.3.2}

\begin{solution}
	(a) Define the operation $\prop{G}(x,n)$ for set a $x$ and $n \in \nats$ by
    $$
    \prop{G}(x,n) = \braces{x} \,.
    $$
    Then by Theorem~6.3.6 there is a unique sequence $\angles{a_n \where n \in \nats}$ where
    \ali{
        a_0 &= \es \\
        a_{n+1} &= \prop{G}(a_n, n) = \braces{a_n}
    }
    for all $n \in \nats$.
    Clearly the range of $\angles{a_n}$ is the set we seek. \qedsymbol

    (b) Similarly define the operation $\prop{G}(x,n)$ for a set $x$ and $n \in \nats$ by
    $$
    \prop{G}(x,n) = \pset{x} \,,
    $$
    noting that this set exists by the Axiom of Power Set.
    Then by Theorem~6.3.6 there is a sequence $\angles{a_n \where n \in \nats}$ defined by
    \ali{
        a_0 &= \nats \\
        a_{n+1} &= \prop{G}(a_n, n) = \pset{a_n} \,,
    }
    noting that $a_0 = \nats$ exists by the Axiom of Infinity.
    Clearly then the range of $\angles{a_n}$ is the set we seek. \qedsymbol

    (c) Define the operation $\prop{G}(x,n)$ for a set $x$ and $n \in \nats$ by
    $$
    \prop{G}(x,n) = S(x) = x \cup \braces{x} \,,
    $$
    Then by Theorem~6.3.6 there is a sequence $\angles{a_n \where n \in \nats}$ defined by
    \ali{
        a_0 &= \w \\
        a_{n+1} &= \prop{G}(a_n, n) = S(a_n) = a_n+1 \,,
    }
    noting that $a_0 = \w = \nats$ exists by the Axiom of Infinity.
    Clearly then the range of $\angles{a_n}$ is $A = \braces{\w, \w+1, \w+2, \ldots}$.
    It then follows that  $\w + \w = \w \cup A$ is the set we seek. \qedsymbol
\end{solution}

\question{6.3.3}

\begin{solution}
    Define the operation $\prop{G}(x,n)$ for a set $x$ and $n \in \nats$ by
    $$
    \prop{G}(x,n) = \pset{x} \,,
    $$
    noting that this set exists by the Axiom of Power Set.
    Then by Theorem~6.3.6 there is a sequence $\angles{V_n \where n \in \nats}$ defined by
    \ali{
        V_0 &= \es \\
        V_{n+1} &= \prop{G}(V_n, n) = \pset{V_n} \,,
    }
    noting that $V_0 = \es$ exists by the Axiom of Existence.
    Then we let
    $$
    V_\w = \bigcup_{n \in \w} V_n \,,
    $$
    noting that $\w = \nats$.
    This set exists by the Axiom of Union. \qedsymbol
\end{solution}

\question{6.3.4}

\begin{solution}
	(a) First we show by induction that every $V_n$ is finite (for $n \in \nats$).
    For $n=0$ we have $V_n = V_0 = \es$, which is clearly finite.
    Now suppose that $V_n$ is finite then we have $V_{n+1} = \pset{V_n}$, which is finite by Theorem~4.2.8.

    Now consider any $x \in V_\w = \bigcup_{n \in \w} V_n$ so that there is an $n \in \w$ such that $x \in V_n$.
    We note that $n \neq 0$ since $V_0 = \es$ so it cannot be that $x \in V_0 = \es$.
    Hence $V_{n-1}$ is a set and moreover $V_n = \pset{V_{n-1}}$.
    So since $x \in V_n$ it follows that $x \in \pset{V_{n-1}}$ so that $x \ss V_{n-1}$.
    Thus it follows that $|x| \leq |V_{n-1}|$ so that clearly $x$ is finite since $V_{n-1}$ is (shown above). \qedsymbol

    (b) Consider any $x \in V_w$.
    Then by the same argument as in part (a) above it follows that $x \in V_n$ where $n \neq 0$.
    Hence again $V_{n-1}$ is a set and $V_n = \pset{V_{n-1}}$ so that $x \ss V_{n-1}$.
    Then for any $y \in x$ we have that $y \in V_{n-1}$, from which it follows that clearly $y \in \bigcup_{k \in \w} V_k = V_\w$.
    Hence since $y$ was arbitrary $x \ss V_\w$, and since $x$ was arbitrary this shows that $V_\w$ is transitive by definition. \qedsymbol

    (c) First we show by induction that each $V_n$ (where $n \in \w$) is transitive.
    For $n=0$ we have $V_n = V_0 = \es$, which is clearly vacuously transitive.
    Now suppose that $V_n$ is transitive and consider any $x \in V_{n+1} = \pset{V_n}$ so that $x \ss V_n$.
    Now consider any $y \in x$ so that also $y \in V_n$.
    But since $V_n$ is transitive $y \ss V_n$ so that $y \in \pset{V_n} = V_{n+1}$.
    Hence since $y$ was arbitrary this shows that $x \ss V_{n+1}$ and since $x$ was arbitrary this shows by definition that $V_{n+1}$ is transitive, thereby completing the inductive proof.

    Now we show that $V_\w$ is inductive.
    So  first note that $V_1 = \pset{V_0} = \pset{\es} = \braces{\es}$ so that $0 = \es \in V_1$.
    From this is clearly follows that $0 \in \bigcup_{n \in \w} V_n = V_\w$.

    Now suppose that $n \in V_\w = \bigcup_{k \in \w} V_k$ so that there is an $m \in \w$ such that $n \in V_m$.
    Since it was shown above that $V_m$ is transitive we have that $n \ss V_m$ as well.
    So consider any $x \in n+1 = n \cup \braces{n}$.
    If $x \in n$ then also $x \in V_m$ since $n \ss V_m$.
    On the other hand if $x \in \braces{n}$ then $x = n \in V_m$.
    Since $x$ was arbitrary this shows that $n+1 \ss V_m$ so that $n+1 \in \pset{V_m} = V_{m+1}$.
    From this it clearly follows that $n+1 \in \bigcup_{k \in \w} V_k = V_\w$.
    This shows that $V_\w$ is inductive by definition. \qedsymbol
\end{solution}

\question{6.3.5}

\begin{solution}
	(a) First we show that if $x \in V_n$ for some $n \in \w$ then $x \in V_m$ for all $m \geq n$.
    We show this by induction on $m$.
    So for $m=n$ clearly $x \in V_n = V_m$.
    Now suppose that $x \in V_m$.
    Then it was shown in Exercise~6.3.4 part (c) that $V_m$ is transitive so that $x \ss V_m$.
    Hence $x \in \pset{V_m} = V_{m+1}$, thereby completing the inductive proof.

    Now suppose that $x,y \in V_\w$.
    Then  there are $n,m \in \w$ such that $x \in V_n$ and $y \in V_m$.
    Without loss of generality we can assume that $n \leq m$ (since if this is not the case then we simply reverse the roles of $x$ and $y$).
    So since $m \geq n$ it follows from what was shown above that $x \in V_m$ as well.
    Hence we have that clearly $\braces{x,y} \ss V_m$ since both $x \in V_m$  and $y \in V_m$.
    Then $\braces{x,y} \in \pset{V_m} = V_{m+1}$ from which it clearly follows that $\braces{x,y} \in \bigcup_{k \in \w} V_k = V_\w$. \qedsymbol

    (b) Suppose that $X \in V_\w = \bigcup_{k \in \w} V_k$.
    Then there is an $n \in \w$ such that $X \in V_n$.
    It was shown in Exercise 6.3.4 part (c) that $V_n$ is transitive so that $X \ss V_n$.

    First we show that $\bigcup X \in V_\w$.
    So consider any $x \in \bigcup X$ so that there is a $Y \in X$ such that $x \in Y$.
    Then since $X \ss V_n$ we have that $Y \in V_n$.
    Since again $V_n$ is transitive we have that $Y \ss V_n$ so that $x \in V_n$ since $x \in Y$.
    Since $x$ was arbitrary it follows that $\bigcup X \ss V_n$ so that $\bigcup X \in \pset{V_n} = V_{n+1}$.
    From this it clearly follows that  $\bigcup X \in \bigcup_{k \in \w} V_k = V_\w$.

    Next we show that $\pset{X} \in V_\w$.
    So consider any $Y \in \pset{X}$ so that $Y \ss X$.
    Now consider any $y \in Y$ so that also $y \in X$.
    Since $X \ss V_n$ we have that $y \in V_n$.
    But since $y \in Y$ was arbitrary it follows that $Y \ss V_n$ so that $Y \in \pset{V_n} = V_{n+1}$.
    Then since $Y \in \pset{X}$ was arbitrary it follows that $\pset{X} \ss V_{n+1}$ so that $\pset{X} \in \pset{V_{n+1}} = V_{n+2}$.
    From this it clearly follows that  $\pset{X} \in \bigcup_{k \in \w} V_k = V_\w$. \qedsymbol

    (c) Note the issue with this part in the errata list.
    Since $A \in V_\w$ we have by Exercise~6.3.4 part (a) that $A$ is finite.
    Then by Theorem~2.2.5 it follows that $f[A]$ is finite.
    Also clearly $f[A]$ is a subset of $V_\w$ and hence is a finite subset.
    Therefore by part (d) below $f[A] \in V_\w$. \qedsymbol

    (d) Consider any finite $X \ss V_\w$.
    Suppose then that $|X| = n$ for some $n \in \nats$.
    Then for each $x_k \in X$, where $k \in n$, we have that $x_k \in V_\w = \bigcup_{m \in \w} V_m$ so that there is an $m_k \in \w$ where $x_k \in V_{m_k}$.
    Now let $m = \max_{k \in n} m_k$, which exists since $n$ is finite.
    Then, for any $k \in n$, by what was shown in Exercise~6.3.5 part (a) we have $x_k \in V_m$ since $x_k \in V_{m_k}$ and $m \geq m_k$.
    Hence it follows that $X \ss V_m$ so that $X \in \pset{V_m} = V_{m+1}$.
    Clearly then $X \in \bigcup_{k \in \w} V_k = V_\w$. \qedsymbol
\end{solution}

\question{6.4.1}

\begin{solution}
    \def\P{\prop{P}}
    \def\G{\prop{G}}
    \def\R{\prop{R}}
    \def\F{\prop{F}}
    Since in these Recursion Theorems the arguments of interest of the given operation $\G$ are typically functions, we assume that $\G$ still takes a single variable despite what the text says.
    Hence for each $x$ there is a unique $y$ such that $y = \G(x)$.
    It just happens that in this case the variables of interest are functions of two variables.

    For ordinals $\a$ and $\b$ we let $f_{\a,\b}$ be the isomorphism from the ordinal $(\b+1)\cdot(\a+1)$ to the lexicographic ordering of $(\a+1) \times (\b+1)$, which exists by Theorem~6.5.8.
    We also note that  according to Exercise~6.5.2 we have
    $$
    (\b+1)\cdot(\a+1) = (\b+1)\cdot\a + (\b+1)\cdot 1 = (\b+1)\cdot\a + (\b+1) = \squares{(\b+1)\cdot\a + \b} + 1
    $$
    so that $(\b+1)\cdot(\a+1)$ is a successor ordinal.
    Then for $\g < (\b+1)\cdot(\a+1)$ define
    $$
    X_{\a,\b,\g} : \braces{(\d,\e) \in (\a+1) \times (\b+1) \where (\d,\e) \prec f_{\a,\b}(\g)}
    $$

    Then for $\g < (\b+1) \cdot (\a+1)$ we say that $t$ is a computation of length $\g$ if $t$ is a transfinite sequence whose domain is $\g+1$ and such that $t(\d) = \G(t \circ (\inv{f_{\a,\b}} \rest X_{\a,\b,\d}))$ for all $\d \leq \g$.
    We note that since $(\b+1)\cdot(\a+1)$ is a successor that $\g+1 \leq (\b+1)\cdot(\a+1)$.

    Now we define the property $\P(x,y,z)$ such that $\P(x,y,z)$ holds if and only if
    \begin{enumerate}
        \item $x$ and $y$ are ordinal numbers and $z = t(\inv{f_{x,y}}(x,y))$ for some computation $t$ of length $(y+1)\cdot x + y$ (with respect to $\G$), or
        \item $x$ or $y$ is \emph{not} an ordinal and $z = \es$.
    \end{enumerate}

    We prove that $\P$ defines an operation.
    Hence we have to show that  for any $x$ and $y$ there is a unique $z$ such that $\P(x,y,z)$ holds.
    So consider any sets $\a$ and $\b$.
    If $\a$ or $\b$ is not an ordinal than clearly $\P(\a,\b,\es)$ holds and $\es$ is unique.
    So suppose that both $\a$ and $\b$ are ordinals.
    Then it suffices to show that there is a unique computation of length $(y+1)\cdot x + y$ (with respect to $\G$) since this will make $z = t(\inv{f_{\a,\b}}(\a,\b))$ unique.
    We show this via transfinite induction.

    So consider any ordinal $\g < (\b+1) \cdot (\a+1)$ so that $\g \leq (y+1)\cdot x + y$ and assume that for all $\d < \g$ that there is a unique computation of length $\d$ and we show that there exists a unique computation of length $\g$, which completes the proof that $\P$ defines an operation.

    \emph{Existence.} First define a property $\R(x,y)$ such that $\R(x,y)$ holds if and only if
    \begin{enumerate}
        \item $x$ is an ordinal where $x < \g$ and $y$ is a computation of length $x$ (with respect to $\G$), or
        \item $x$ is is an ordinal and $x \geq \g$ and $y = \es$, or
        \item $x$ is not an ordinal and $y = \es$ \,.
    \end{enumerate}
    Clearly by the induction hypothesis this property has a unique $y$ for every $x$.
    Hence we can apply the Axiom Schema of Replacement, according to which there is a set $T$ such that for every $\d \in \g$ (so that $\d < \g$) there is a $t$ in $T$ such that $\R(\d, t)$ holds.
    That is
    $$
    T = \braces{t \where t \text{ is the unique computation of length $\d$ for all $\d < \g$}}
    $$
    Now, $T$ is a system of transfinite sequences (which are functions) so define $\bar{t} = \bigcup T$ and let $\t = \bar{t} \cup \braces{(\g, \G(\bar{t} \circ (\inv{f_{\a,\b}} \rest X_{\a,\b,\g}))}$.

    \emph{Claim 1:} $\dom(\t) = \g+1$.
    So consider any $\e \in \dom(\t)$.
    Clearly if $\e = \g$ then $\e \in \g+1$.
    On the other hand if $\e \in \dom(\bar{t})$ then there is a $t \in T$ such that $\e \in \dom(t)$.
    But since $t$ is a computation of length $\d$ and $\d < \g$ it follows that $\e \leq \d < \g < \g+1$ so that $\e \in \g+1$.
    Hence since $\e$ was arbitrary $\dom(\t) \ss \g+1$.

    Now consider any $\e \in \g+1$ so that $\e \leq \g$.
    If $\e = \g$ then clearly by definition $\e \in \dom(\t)$.
    On the other hand if $\e \neq \g$ then $\e < \g$.
    So consider the $t \in T$ where $t$ is the unique computation of length $\e$ (which exists since $\e < \g$).
    Then clearly $\e \in \dom(t)$ so that $\e \in \dom(\bar{t})$.
    From this it follows that clearly $\e \in \dom(\t)$ so that $\g+1 \ss \dom(\t)$ since $\e$ was arbitrary.
    This proves the claim.
    
    \emph{Claim 2:} $\t$ is a function.
    Consider any $\e \in \dom(\t) = \g+1$ so that again $\e \leq \g$.
    If $\e = \g$ then clearly $\t(\e) = \t(\g) = \G(\bar{t} \circ (\inv{f_{\a,\b}} \rest X_{\a,\b,\g}))$ is unique since $\G$ is an operation.
    On the other hand if $\e < \g$ then $\t$ is a function so long as $\bar{t}$ is, and this is the case so long as $T$ is a compatible system of functions since $\bar{t} = \bigcup T$.
    We show this presently.

    So consider any arbitrary $t_1, t_2 \in T$ where $t_1$ is the computation of length $\e_1$ and $t_2$ is the computation of length $\e_2$.
    Without loss of generality we can assume that $\e_1 \leq \e_2$.
    We must show that $t_1(\d) = t_2(\d)$ for all $\d \leq \e_1$.
    This we show by transfinite induction.
    So suppose that $t_1(\k) = t_2(\k)$ for all $\k < \d \leq \e_1$.
    Then clearly $t_1 \rest \d = t_2 \rest \d$, from which it follows that $t_1 \circ (\inv{f_{\a,\b}} \rest X_{\a,\b,\d}) = t_2 \circ (\inv{f_{\a,\b}} \rest X_{\a,\b,\d})$ and since $\G$ is an operation we have $t_1(\d) = \G(t_1 \circ (\inv{f_{\a,\b}} \rest X_{\a,\b,\d})) = \G(t_2 \circ (\inv{f_{\a,\b}} \rest X_{\a,\b,\d})) = t_2(\d)$.
    This completes the proof of the claim.

    \emph{Claim 3:} $\t(\d) = \G(\t \circ (\inv{f_{\a,\b}} \rest X_{\a,\b,\d}))$ for all $\d \leq \g$.
    So consider any such $\d$.
    If $\d = \g$ then since $\t \rest \g = \bar{t}$ we clearly have $\t(\d) = \t(\g) = \G(\bar{t} \circ (\inv{f_{\a,\b}} \rest X_{\a,\b,\g})) = \G(\t \circ (\inv{f_{\a,\b}} \rest X_{\a,\b,\g})) = \G(\t \circ (\inv{f_{\a,\b}} \rest X_{\a,\b,\d}))$\,.
    On the other hand if $\d < \g$ then let $t \in T$ be the computation of length $\d$ (which exists since $\d < \g$).
    Then $\t(\d) = t(\d) = \G(t \circ (\inv{f_{\a,\b}} \rest X_{\a,\b,\d})) = \G(\t \circ (\inv{f_{\a,\b}} \rest X_{\a,\b,\d}))$ since $t$ is a computation (with respect to $\G$) and clearly $t \ss \t$.

    Claims 1 through 3 show that $\t$ is a computation of length $\g$ and hence that such a computation exists.

    \emph{Uniqueness.} Now let $\s$ be another computation of length $\g$.
    We show that $\s = \t$, which proves uniqueness.
    Since both $\s$ and $\t$ are functions with $\dom(\s) = \g+1 = \dom(\t)$ it suffices to show that $\s(\d) = \t(\d)$ for all $\d \leq \g$.
    We show this once again by using transfinite induction.
    So suppose that $\s(\e) = \t(\e)$ for all $\e < \d \leq \g$.
    It then follows that $\s \rest \d = \t \rest \d$ so that $\s \circ (\inv{f_{\a,\b}} \rest X_{\a,\b,\d}) = \t \circ (\inv{f_{\a,\b}} \rest X_{\a,\b,\d})$.
    Then since $\s$ and $\t$ are computations we have that $\s(\d) = \G(\s \circ (\inv{f_{\a,\b}} \rest X_{\a,\b,\d})) = \G(\t \circ (\inv{f_{\a,\b}} \rest X_{\a,\b,\d})) = \t(\d)$, thereby completing the uniqueness proof.

    This completes the proof that $\P$ defines an operation.

    So let $\F$ be the operation defined by $\P$.
    The last thing we need to show to complete the proof of the entire theorem is that $\F(\a, \b) = \G(\F \rest (\a \times \b))$ for all ordinals $\a$ and $\b$, noting that we are treating $\F$ as a function even though it is an operation.
    Thus, for any set $X$, $\F \rest X$ denotes the set $\braces{(x, \F(x)) \where x \in X}$, which forms a function with domain $X$.
    The range of this function is a set whose existence is guaranteed by the Axiom Schema of Replacement since $\F$ is an operation.

    So consider any ordinals $\a$ and $\b$ and the unique computation $t$ of length $(\b+1)\cdot\a + \b$.
    Then clearly for any $\g \leq (\b+1)\cdot\a + \b$ we have that $t_\g = t \rest (\g+1)$ is a computation of length $\g$.
    Since this computation is the unique computation of length $\g$, by the definition of $\F$ as it relates to $\P$ we have $\F(f(\g)) = t_\g(\g) = t(\g)$.
    Since $\g$ was arbitrary this shows that $\F \rest (\a \times \b) = t \rest (\b+1)\cdot\a + \b$.
    Then clearly we have $\F(\a, \b) = t((\b+1)\cdot\a + \b) = \G(t \circ (\inv{f_{\a,\b}} \rest X_{\a,\b,(\b+1)\cdot\a + \b})) = \G(t \rest (\b+1)\cdot\a + \b) = \G(\F \rest (\a \times \b))$ by what was just shown above. \qedsymbol

    NOTE: This is a very nasty exercise and there may be a simpler way to do this.
\end{solution}

\question{6.4.2}

\begin{solution}
    \def\G{\prop{G}}
    \def\F{\prop{F}}
    Note that there is no such operation that can exactly satisfy both conditions as they actually contradict each other.
    To  see this suppose there is such an operation $\F$.
    Then define the set $x = \es$ and $n = 0$
    Then by (a) we have that $\F(x,n+1) = \F(x, 1) = 0$.
    It then follows from (b) that there is a $y$ and $z$ such that $x = (x,y)$, but clearly this is not the case for $x=\es$.
    Hence a contradiction.

    To remedy this we simply add a condition to (b), which when restated becomes

    (b) $n > 0$ and $\F(x,n+1) = 0$ if and only if there exists $y$ and $z$ such that $x = (y,z)$ and $\F(y,n) = 0$.

	Now, define an operation $\G$ by $z = \G(x,y_u)$ if and only if either
    \begin{enumerate}
        \item $y_u$ is a function with parameter $u$, $\dom(y_u) = 1$, and $z = 0$, or
        \item $y_u$ is a function with parameter $u$, $\dom(y_u) = \a+1$ for some ordinal $\a$, there are $p$ and $q$ where $x = (p,q)$, $y_p(\a) = 0$, and $z = 0$ or
        \item None of the above hold and $z = 1$.
    \end{enumerate}

    Then by Theorem~6.4.9 there is an operation $\F$ such that $\F(x,\a) = \G(x, \F_x \rest \a)$ for all ordinals $\a$ and sets $x$.

    Then for any set $x$ we have that clearly $\F \rest 1$ is a function with domain $1$ (and parameter $x$) so that by definition
    $$
    \F(x,1) = \G(x, \F_x \rest 1) = 0 \,.
    $$
    This shows (a).

    To show (b) consider any ordinal $n$ and set $x$.

    ($\to$) Suppose that $n > 0$ and $\F(x,n+1) = 0$ so that clearly $(3)$ above cannot be the case.
    Also $(1)$ cannot be the case since $\F(x,n+1) = \G(x, \F_x \rest n+1)$ and $\dom(\F_x \rest n+1) = n + 1 > 1$.
    Hence (2) is the case so that there are $y$ and $z$ such that $x = (y,z)$ and $(\F_y \rest n+1)(n) = 0$.
    Hence it follows that $F(y,n) = 0$.

    ($\leftarrow$) Now suppose that there are $y$ and $z$ such that $x = (y,z)$ and $F(y,n) = 0$.
    Then $F(y,n) = G(y, \F_y \rest n) = 0$.

    So if $n=0$ then $\dom(\F_y \rest n) = \dom(\F_y \rest 0) = 0 \neq 1$ so (1) cannot be the case.
    Also since $0$ is not a successor ordinal (2) cannot be the case either (since $\dom(\F_y \rest 0) \neq \a+1$ for any ordinal $\a$).
    Hence (3) must be the case, but this implies that $F(y,n) = 1$, which is a contradiction.
    So we must have that $n \neq 0$ so $n > 0$.

    Since $\F(y, n) = 0$ clearly we have that $(\F_y \rest n+1)(n) = 0$.
    Since also $\dom(\F_x \rest n+1) = n+1$ we find that (2) holds for $\G(x, \F_x \rest n+1)$.
    From this it follows that $\F(x,n+1) = \G(x, \F_x \rest n+1) = 0$.

    This completes the proof. \qedsymbol
\end{solution}

\question{6.5.1}

\begin{solution}
    \begin{statement}{Lemma 6.5.1.1}
        $0 \cdot \a = 0$ for all ordinals $\a$.
    \end{statement}
    
    \proof{
        We show this by transfinite induction on $\a$.
        For $\a = 0$ we have $0 \cdot \a = 0 \cdot 0 = 0$ by Definition~6.5.6a.
        Now suppose that $0 \cdot \a = 0$ so that we have
        $$
        0 \cdot (\a + 1) = 0 \cdot \a + 0 = 0 + 0 = 0
        $$
        by the induction hypothesis, Definition~6.5.6b, and Definition~6.5.1a.
        Lastly, suppose that $\a \neq 0$ is a limit ordinal and that $0 \cdot \b = 0$ for all $\b < \a$.
        Then by Definition~6.5.6c we have that
        $$
        0 \cdot \a = \sup\braces{0 \cdot \b \where \b < \a} = \sup\braces{0 \where \b < \a} = \sup\braces{0} = 0
        $$
        by the induction hypothesis.
        This completes the proof. \qedsymbol
    }

    \begin{statement}{Lemma 6.5.1.2}
        Ordinal $\a$ is a limit ordinal if and only if $\b+1 < \a$ for every ordinal $\b < \a$.
    \end{statement}

    \proof{
        ($\to$) We show this by contrapositive.
        So suppose that there is a $\b < \a$ such that $\b+1 \geq \a$.
        Then by Lemma~6.2.7.1 we have also that $\b+1 \leq \a$, from which it follows that $\a = \b+1$ so that $\a$ is a successor ordinal and not a limit ordinal.

        ($\leftarrow$) Suppose that $\b+1 < \a$ for every $\b < \a$.
        Suppose also that $\a = \g + 1$ is a successor ordinal.
        Then clearly $\g < \g+1 = \a$ so that also $\g+1 < \a$, which is an immediate contradiction.
        Hence it must be that $\a$ is a limit ordinal.

        Note that the bi-conditional is vacuously true for $\a = 0$. \qedsymbol
    }

    \begin{statement}{Lemma 6.5.1.4}
        Suppose that $\a \neq 0$ is an ordinal and $\b \neq 0$ is a limit ordinal then $\a \cdot \b$ is a limit ordinal and $\a \cdot \b \neq 0$.
    \end{statement}

    \proof{
        Since $\a \neq 0$ we have that $\a \geq 1$.
        Now consider any $\g < \a \cdot \b$.
        We claim that $\g \leq \a \cdot \d$ for some $\d < \b$.
        Suppose to the contrary that $\g > \a \cdot \d$ for all $\d < \b$.
        Then $\g$ is an upper bound of the set $\braces{\a \cdot \d \where \d < \b}$.
        But from this it follows that
        $$
        \g \geq \sup\braces{\a \cdot \d \where \d < \b} = \a \cdot \b
        $$
        by Definition~6.5.6c, which contradicts the definition of $\g$.
        Hence the claim is true so that $\g \leq \a \cdot \d$ for some $\d < \b$.
        Then we have by Lemma~6.5.4a and the fact that $1 \leq \a$ that
        $$
        \g + 1 \leq \a \cdot \d + 1 \leq \a \cdot \d + \a = \a \cdot (\d + 1) < \a \cdot \b
        $$
        by Exercise~6.5.7a since $\d + 1 < \b$ since $\b$ is a limit ordinal.
        Note that we also used Definition~6.5.6b.
        Thus since $\g + 1 < \a \cdot \b$ and $\g$ was arbitrary it follows from Lemma~6.5.1.2 that $\a \cdot \b$ is a limit ordinal.

        We also have that $\a \neq 0$ and $0 < \b$ (since $0 \neq \b$) so that by Exercise~6.5.7a above and Definition~6.5.6a
        $$
        0 = \a \cdot 0 < \a  \cdot \b \,.
        $$
        Thus $\a \cdot \b \neq 0$. \qedsymbol
    }

    \mainprob
    
	We show this by transfinite induction on $\g$.

    First for $\g=0$ we have
    $$
    (\a \cdot \b) \cdot \g = (\a \cdot \b) \cdot 0 = 0 = \a \cdot 0 = \a \cdot (\b \cdot 0) = \a \cdot (\b \cdot \g) \,,
    $$
    where we have used Definition~6.5.6a repeatedly.
    Now suppose that $(\a \cdot \b) \cdot \g = \a \cdot (\b \cdot \g)$ for ordinal $\g$.
    Then
    \ali{
        (\a \cdot \b) \cdot (\g+1) &= (\a \cdot \b) \cdot \g + \a \cdot \b & \text{(by Definition~6.5.6b)} \\
        &= \a \cdot (\b \cdot \g) + \a \cdot \b & \text{(by the induction hypothesis)} \\
        &= \a \cdot \squares{\b \cdot \g + \b} & \text{(by the distributive law, see Exercise~6.5.2)} \\
        &= \a \cdot \squares{\b \cdot (\g+1)} \,. & \text{(by Definition~6.5.6b)}
    }
    Now suppose that $\g \neq 0$ is a limit ordinal and $(\a \cdot \b) \cdot \d = \a \cdot (\b \cdot \d)$ for all $\d < \g$.
    First if $\b = 0$ then
    $$
    (\a \cdot \b) \cdot \g = (\a \cdot 0) \cdot \g = 0 \cdot \g = 0 = \a \cdot 0 = \a \cdot (0 \cdot \g) = \a \cdot (\b \cdot \g)
    $$
    where we have used Definition~6.5.6a and Lemma~6.5.1.1.
    So assume that $\b \neq 0$.
    Then we have $(\a \cdot \b) \cdot \g = \sup\braces{(\a \cdot \b) \cdot \d \where \d < \g} = \sup\braces{\a \cdot (\b \cdot \d) \where \d < \g}$ by Definition~6.5.6c and the induction hypothesis.
    Then, since $\b \neq 0$, by Lemma~6.5.1.4 we have that $\b \cdot \g$ is a limit ordinal and $\b \cdot \g \neq 0$.
    From this and Definition~6.5.6c we have that
    $$
    (\a \cdot \b) \cdot \g = \sup\braces{\a \cdot (\b \cdot \d) \where \d < \g} = \sup\braces{\a \cdot \d \where \d < \b \cdot \g} = \a \cdot (\b \cdot \g)
    $$
    as desired.
    This completes the  inductive proof. \qedsymbol
\end{solution}

\question{6.5.2}

\begin{solution}
	We show this by transfinite induction on $\g$.
    So  for $\g = 0$ we have
    \ali{
        \a \cdot (\b + \g) &= \a \cdot (\b + 0) = \a \cdot \b & \text{(by Definition~6.5.1a)} \\
        &= a \cdot \b + 0 & \text{(by Definition~6.5.1a)} \\
        &= \a \cdot \b + \a \cdot 0 & \text{(by Definition~6.5.6a)} \\
        &= \a \cdot \b + \a \cdot \g \,.
    }
    Now suppose that $\a \cdot (\b + \g) = \a \cdot \b + \a \cdot \g$ for ordinal $\g$.
    We then have
    \ali{
        \a \cdot \squares{\b + (\g + 1)} &= \a \cdot \squares{(\b + \g) + 1} & \text{(by Definition~6.5.1b)} \\
        &= \a \cdot (\b + \g) + \a & \text{(by Definition~6.5.6b)} \\
        &= (\a \cdot \b + \a \cdot \g) + \a & \text{(by the induction hypothesis)} \\
        &= \a \cdot \b + (\a \cdot \g + \a) & \text{(by the associativity of addition, Lemma~6.5.4c)} \\
        &= \a \cdot \b + \a \cdot (\g + 1) \,. & \text{(by Definition~6.5.6b)}
    }
    Lastly, suppose that $\g \neq 0$ is a limit ordinal and that $\a \cdot (\b + \d) = \a \cdot \b + \a \cdot \d$ for all $\d < \g$.
    If $\a = 0$ then we have
    $$
    \a \cdot (\b + \g) = 0 \cdot (\b + \g) = 0 = 0 + 0 = 0 \cdot \b + 0 \cdot \g = \a \cdot \b + \a \cdot \g \,,
    $$
    where we have used Lemma~6.5.1.1 above.
    So suppose that $\a \neq 0$ so that $\a \geq 1$.
    Now, if $\x < \b  + \g$ then $\x \leq \b + \d$ for some $\d < \g$.
    Then by Lemma~6.5.4 we have that $\x + 1 \leq (\b + \d) + 1 = \b + (\d+1) < \b + \g$ since $\g$ is a limit ordinal.
    Hence $\b + \g$ is a limit ordinal so that by Definition~6.5.6c we have $\a \cdot (\b + \g) = \sup\braces{\a \cdot \x \where \x < \b + \g}$.
    But then by Definition~6.5.1c we have that $\b + \g = \sup\braces{\b + \d \where \d < \g}$.
    It then follows that
    $$
    \a \cdot (\b + \g) = \sup\braces{\a \cdot \x \where \x < \b + \g} = \sup\braces{\a \cdot (\b + \d) \where \d < \g}
    = \sup\braces{\a \cdot \b + \a \cdot \d \where \d < \g}
    $$
    by the induction hypothesis.
    Then by Definition~6.5.6c we have that $\sup\braces{\a \cdot \d \where \d < \g} = \a \cdot \g$ since $\g$ is a limit ordinal.
    It then follows from Lemma~6.5.1.4 that $\a \cdot \g$ is also a limit ordinal and $\a \cdot \g \neq 0$ since $\a \neq 0$.
    From  this and Definition~6.5.1c we have that
    $$
    \a \cdot (\b + \g) = \sup\braces{\a \cdot \b + \a \cdot \d \where \d < \g} = \sup\braces{\a \cdot \b + \d \where \d < \a \cdot \g}
    = \a \cdot \b + \a \cdot \g
    $$
    as desired.
    This completes the inductive proof. \qedsymbol
\end{solution}

\question{6.5.3}

\begin{solution}
	(a) By Lemma~6.5.4c we have
    $$
    (\w + 1) + \w = \w + (1 + \w) = \w + \w = \w \cdot 2 \,.
    $$

    (b) We simply have
    $$
    \w + \w^2 = \w \cdot 1 + \w \cdot \w = \w \cdot (1 + \w) = \w \cdot \w = \w^2
    $$

    (c) First we show that $(\w+1) \cdot n = \w \cdot n + 1$ for all $n \in \w$ where $n > 0$.
    We show this by standard (as opposed to transfinite) induction on $n$.
    For $n=1$ we have
    $$
    (\w + 1) \cdot n = (\w+1) \cdot 1 = \w + 1 = \w \cdot 1 + 1 = \w \cdot n + 1\,.
    $$
    Now suppose that $(\w + 1) \cdot n = \w \cdot n + 1$ so that we have
    \ali{
        (\w+1) \cdot (n+1) &= (\w + 1) \cdot n + (\w + 1) & \text{(by Definition~6.5.6b)} \\
        &= (\w\cdot n + 1) + (\w + 1) & \text{(by the induction hypothesis)} \\
        &= \w \cdot n + (1 + \w) + 1 & \text{(by the associativity of addition)} \\
        &= \w \cdot n + \w + 1 \\
        &= \w \cdot (n+1) + 1 \,. & \text{(by Definition~6.5.6b)}
    }
    This completes the inductive proof.

    Next we show that $(\w + 1) \cdot \w = \w^2$.
    Since $\w$ is a limit ordinal we have by Definition~6.5.6c
    $$
    (\w + 1) \cdot \w = \sup\braces{(\w + 1) \cdot n \where n \in \w} = \sup\braces{\w \cdot n + 1 \where n \in \w}
    $$
    by what was shown inductively above.
    Now, since we have
    $$
    \w \cdot n + 1 < (\w \cdot n + 1) + \w = \w \cdot n + (1 + \w) = \w \cdot n + \w = \w \cdot (n+1)
    $$
    it is clear that we have
    $$
    (\w + 1) \cdot \w = \sup\braces{\w \cdot n + 1 \where n \in \w} = \sup\braces{\w \cdot n \where n \in \w} = \w \cdot \w = \w^2
    $$
    by Definition~6.5.6c.

    Hence for the main problem we have
    $$
    (\w+1) \cdot \w^2 = (\w+1) \cdot (\w \cdot \w) = \squares{(\w+1) \cdot \w} \cdot w = \w^2 \cdot \w = \w^{2+1} = \w^3
    $$
    by Definition~6.5.9b. \qedsymbol
\end{solution}

\question{6.5.4}

\begin{solution}
    \begin{statement}{Lemma~6.5.4.1}
        If $\a$ and $\b$ are ordinals and $\b \neq 0$ is a limit ordinal then $\a + \b$ is a limit ordinal.
    \end{statement}

    \proof{
        First we note that since $\b \neq 0$ we have $1 \leq \b$.
        Consider any $\g < \a + \b$.
        If $\g < \a$ then $\g + 1 \leq \a < \a + 1 \leq \a + \b$.
        On the other hand if $\g \geq \a$ then by Lemma~6.5.5 there is an ordinal $\d$ such that $\a + \d = \g$.
        Then since $\a + \d = \g < \a + \b$ it follows from Lemma~6.5.4a that $\d < \b$, and so by Lemma~6.5.1.2 we have that $\d+1 < \b$ since $\b$ is a limit ordinal.
        Therefore we have $\g + 1 = (\a + \d) + 1 = \a + (\d + 1) < \a + \b$ by Lemma~6.5.4 parts c and a.
        Hence in all cases $\g + 1 < \a + \b$ so that $\a + \b$ is a limit ordinal by Lemma~6.5.1.2 since $\g$ was arbitrary. \qedsymbol
    }

    \begin{statement}{Lemma~6.5.4.2}
        If $\a$ is a limit ordinal and $\b$ is another ordinal such that $\b < \a$ then $\b + n < \a$ for any natural number $n$.
    \end{statement}

    \proof{
        We show this by normal (not transfinite) induction on $n$.
        For $n = 0$ we clearly have $\b + n = \b + 0 = \b < \a$.
        So suppose that $\b + n < \a$ so that we have
        $$
        \b + (n + 1) = (\b + n) + 1 < \a
        $$
        by Definition~6.5.1b.
        The inequality follows from Lemma~6.5.1.2 and the induction hypothesis since $\a$ is a limit ordinal.
        This completes the inductive proof. \qedsymbol
    }

    \mainprob

	\emph{Existence.}
    First for any ordinal $\a$ let
    $$
    B = \braces{\g \in \a+1 \where \text{$\g$ is a limit ordinal}}
    $$
    and define $\b = \sup B$.

    First we show that $\b$ is a limit ordinal.
    To this end consider any $\d < \b$.
    Then since $\b$ is the least upper bound of $B$ it follows that $\d$ is not an upper bound of $B$ so that there is a $\g \in B$ such that $\d < \g$.
    Since $\g \in B$ it is a limit ordinal so that also $\d+1 < \g$ by Lemma~6.5.1.2.
    Then since $\b$ is an upper bound of $B$ we have that $\d+1 < \g \leq \b$ so that $\b$ is a limit ordinal by Lemma~6.5.1.2 since $\d$ was arbitrary.

    Now, since each $\g \in B$ is in $\a+1$ we have that $\g < \a+1$ so that $\g \leq \a$.
    Hence $\a$ is an upper bound of $B$, and since $\b$ is the \emph{least} upper bound of $B$ it follows that $\b \leq \a$.
    Because of this there is a unique ordinal $\x$ such that $\b + \x = \a$ by Lemma~6.5.5.

    We claim that $\x < \w$ so that $\x$ is a natural number.
    To the contrary, suppose that $\x \geq \w$.
    It then follows from Lemma~6.5.4 that  $\b + \w \leq \b + \x = \a$.
    Also, $\b + \w$ is a limit ordinal by Lemma~6.5.4.1 above since $\w$ is so that $\b+\w \in B$ since also $\b + \w \in \a+1$ (since $\b+\w \leq \a$).
    Then $\b + \w \leq \b$ since $\b$ is an upper bound of $B$, but this is a contradiction since clearly $\b = \b + 0 < \b + \w$ by Lemma~6.5.4a since $0 < \w$.
    Hence it must be that $\x < \w$.

    Thus $\a = \b + \x$ for a limit ordinal $\b$ and natural number $\x$, thereby proving existence.

    \emph{Uniqueness.}
    Suppose that $\a = \b_1 + n_1 = \b_2 + n_2$ where $\b_1$ and $\b_2$ are limit ordinals and $n_1$ and $n_2$ natural numbers.
    First suppose that $\b_1 \neq \b_2$ so that without loss of generality we can assume that $\b_1 < \b_2$.
    It then follows from Lemma~6.5.4.2 that $\b_1 + n_1 < \b_2$ as well since $\b_2$ is a limit ordinal and $n_1$ is a natural number.
    But then we have $\b_1 + n_1 < \b_2 = \b_2 + 0 \leq \b_2 + n_2$, which contradicts the fact that $\b_1 + n_1 = \a = \b_2 + n_2$.
    So it must be that in fact $\b_1 = \b_2$.
    But then it follows from Lemma~6.5.4b that $n_1 = n_2$ also, which shows uniqueness. \qedsymbol
\end{solution}

\question{6.5.5}

\begin{solution}
    \begin{statement}{Lemma~6.5.5.1}
        If $\a$ and $\b$ are ordinals and $n$ a natural number then $\a < \b$ if and only if $\a + n < \b + n$.
     \end{statement}

    \proof{
        ($\to$) We show this by induction on $n$.
        So for $n=0$ and any ordinals $\a$ and $\b$ where $\a < \b$ we clearly have
        $$
        \a + n = \a + 0 = \a < \b = \b + 0 = \b + n \,.
        $$
        Now suppose that $\a + n < \b + n$ for any ordinals $\a$ and $\b$ where $\a < \b$.
        Suppose that $\a$ and $\b$ are such ordinals so that by Lemma~6.2.7.1 we have $\a + 1 \leq \b < \b + 1$.
        Hence $\a+1$ and $\b+1$ are ordinals such that $\a+1 < \b+1$.
        It then follows from the induction hypothesis that
        $$
        \a + (n+1) = \a + (1+n) = (\a + 1) + n < (\b + 1) + n = \b + (1 + n) = \b + (n+1) \,,
        $$
        noting that clearly natural numbers commute with respect to addition.
        This completes the proof by induction.

        ($\leftarrow$) Suppose that $\a + n < \b + n$ for ordinals $\a$ and $\b$ and natural number $n$.
        It cannot be that $\b < \a$ for then it would follow that $\b + n < \a + n$ by what was just shown.
        Nor can it be that $\a = \b$ since then clearly $\a + n = \b + n$.
        Hence by the linearity of the ordinal ordering it follows that $\a < \b$. \qedsymbol
    }

    \begin{statement}{Corollary~6.5.5.2}
        If $\a$ and $\b$ are ordinals and $n$ a natural number then $\a = \b$ if and only if $\a + n = \b + n$.
    \end{statement}

    \proof{
        This follows directly from Lemma~6.5.5.1 in the same way as the proof of Lemma~6.5.4b. \qedsymbol
    }

    \begin{statement}{Lemma~6.5.5.3}
        If $\a \geq \w$ is an ordinal and $n$ a natural number then $n + \a = \a$.
    \end{statement}

    \proof{
        For any natural number $n$ we show this by transfinite induction on $\a$.
        So if $\a = \w$ then as explained in the text we have $n + \a = n + \w = \w = \a$.
        Now suppose that $n + \a = \a$ so that we have $n + (\a + 1) = (n + \a) + 1 = \a + 1$.
        Lastly, suppose that $\a >\w$ is a limit ordinal and that $n + \g = \g$ for all $\g < \a$.
        We then have by Definition~6.5.1c that
        $$
        n + \a = \sup\braces{n + \g \where \g < \a} = \sup\braces{\g \where \g < \a} = \a
        $$
        by the induction hypothesis and comments in the text after Theorem~6.2.10.
        This completes the inductive proof. \qedsymbol
    }

    \mainprob
    
    In what follows suppose generally that $\a = \g + n$ and $\b = \d + m$ where $\g$ and $\d$ are limit ordinals and $n$ and $m$ are natural numbers.
    Note that $\a$ and $\b$ can be expressed in this way uniquely by Exercise~6.5.4.

    Case: $\b = 0$.
    If $\a = 0$ then clearly $\x = 0$ is the only solution since for any other $\x \neq 0$ we have $\x + \a = \x + 0 = \x \neq 0 = \b$.
    On the other hand if $\a \neq 0$ then there is no solution since for any $\x$ we have by Lemma~6.5.4 that $\x + \a > \x + 0 = \x \geq 0 = \b$ so that $\x + \a \neq \b$.

    Case: $\b$ is a successor.
    From this it follows that clearly $m \neq 0$ since otherwise $\b = \d + m = \d + 0 = \d$ would be a limit ordinal.

    Now, if $\a = 0$ then clearly $\x = \b$ is the only solution so that $\x + \a = \x + 0 = \x = \b$.

    If $\a$ is a successor then similarly $n \neq 0$.
    Suppose that $\g = 0$ so that since $\a \leq \b$ we have $0 + n = n \leq \d + m = \b$.
    Then if $n \leq m$ then $\x = \d + (m - n)$ is a solution since then
    $$
    \x + \a = [\d + (m-n)] + n = \d + [(m-n) + n] = \d + m = \b \,.
    $$
    Moreover clearly this is the only solution since if $\x \neq \d + (m-n)$ then
    $$
    \x + \a = \x + n \neq [\d + (m-n)] + n = \d + m = \b
    $$
    by Corollary~6.5.5.2.
    On the other hand if $n > m$ then for any ordinal $\x = \e + k$ where $\e$ is a limit ordinal and $k$ is a natural number, we have that $k + n > 0 + n = n > m$ so that clearly $k + n \neq m$.
    From this it follows that
    $$
    \x + \a = (\e + k) + n = \e + (k + n) \neq \d + m = \b
    $$
    since both the limit ordinal and the natural number part of ordinals must be equal for the overall ordinals to be equal.
    Hence this case has no solutions.
    Now suppose that $\g \neq 0$.
    Then if $n \neq m$ then there are no solutions since for any ordinal $\x$ we have
    $$
    \x + \a = \x + (\g + n) = (\x + \g) + n \neq \d + m
    $$
    since $n \neq m$ and $\x + \g$ is a limit ordinal by Lemma~6.5.4.1.
    On the other hand if $n = m$ then since $\g + n = \a \leq \b = \d + m = \d + n$ it follows from Lemma~6.5.5.1 and Corollary~6.5.5.2 that $\g \leq \d$, and since $\g \neq 0$ we have that $0 < \g \leq \d$.

    Then we have that $\x$ is a solution if and only if $\x + \g = \d$.
    For, supposing that $\x + \g = \d$, we have
    $$
    \x + \a = \x + (\g + n) = (\x + \g) + n = \d + n = \d + m = \b
    $$
    and if it is not the case then
    $$
    \x + \a = \x + (\g + n) = (\x + \g) + n \neq \d + n = \d + m = \b
    $$
    by Corollary~6.5.5.2.
    Since $\g$ and $\d$ are both nonzero limit ordinals it then follows from the final case below (where both $\a$ and $\b$ are nonzero limit ordinals) there are either zero or infinite such $\x$.

    If $\a$ is a nonzero limit ordinal then clearly are no solutions since for any ordinal $\x$ Lemma~6.5.4.1 tells us that $\x + \a$ is a limit ordinal whereas $\b$ is a successor so that it has to be that $\x + \a \neq \b$.

    Case: $\b$ is a nonzero limit ordinal.

    If $\a = 0$ then clearly $\x = \b$ is the only solution since $\x + \a = \x + 0 = \x$.

    If $\a$ is a successor ordinal then $n \neq 0$ and for any ordinal $\x$ we have that $\x + \a = \x + (\g + n) = (\x + \g) + n$, which is a successor ordinal as well since $\x + \g$ is a limit ordinal again by Lemma~6.5.4.1.
    Hence there are no solutions since $\b$ is a limit ordinal and $\x$ was arbitrary.

    Lastly, suppose that $\a$ is also a nonzero limit ordinal.
    Suppose that there at least one solution $\x$ such that $\x + \a = \b$.
    Now consider any natural number $k$ so that we have
    $$
    (\x + k) + \a = \x + (k + \a) = \x + \a = \b
    $$
    by Lemma~6.5.5.3 since $\a$ is a nonzero limit ordinal and therefore clearly $\a \geq \w$.
    Hence $\x + k$ is also a solution and so there are an infinite number of solutions since $k$ was arbitrary.
    Thus there are either zero solutions or an infinite number of solutions (since a nonzero number of solutions implies an infinite number of solutions).
    This completes the case structure.

    These results are summarized in the following table:
    \begin{center}
    \begin{tabular}{|c|c|c|c|c|}
        \hline
        $\b$ & $\a$ & Other & Other & Solutions \\
        \hline
        \multirow{2}{*}{0} & 0 & & & 1 ($\x = 0$) \\
        \cline{2-5}
        & $>0$ & & &  0 \\
        \hline
        \multirow{6}{*}{Successor} & 0 & & & 1 ($\x = \b$) \\
        \cline{2-5}
        & \multirow{4}{*}{Successor} & \multirow{2}{*}{$\g = 0$} & $n \leq m$ & 1 ($\x = \d + m - n$) \\
        \cline{4-5}
        & & & $n > m$ & 0 \\
        \cline{3-5}        
        & & \multirow{2}{*}{$\g > 0$} & $n \neq m$ & 0 \\
        \cline{4-5}
        & & & $n = m$ & 0 or $\infty$ \\
        \cline{2-5}
        & Limit $>0$ & & & 0 \\
        \hline
        \multirow{3}{*}{Limit $>0$} & 0 & & & 1 ($\x =\b$) \\
        \cline{2-5}
        & Successor & & & 0 \\
        \cline{2-5}
        & Limit $>0$ & & & 0 or $\infty$ \\
        \hline
    \end{tabular}
    \end{center}

    Since this case structure is exhaustive the result follows. \qedsymbol
\end{solution}

\question{6.5.6}

\begin{solution}
    \begin{statement}{Definition 6.5.6.1}
    	We call an ordinal $\a$ an additive ordinal if it has the property that $\b + \a = \a$ for all $\b < \a$.
    \end{statement}

    \begin{statement}{Lemma 6.5.6.2}
        If $\a$ is a limit ordinal and $\b$ is any ordinal then $\a \cdot \b$ is a limit ordinal.
    \end{statement}

    \proof{
        We show this by transfinite induction on $\b$.

        So for $\b = 0$ we clearly have $\a \cdot \b = \a \cdot 0 = 0$, which is a limit ordinal.

        Now suppose that $\a \cdot \b$ is a limit ordinal.
        Then we have $\a \cdot (\b+1) = \a \cdot \b + \a$ by Definition~6.5.6b.
        If $\a = 0$ then $\a \cdot (\b+1) = \a \cdot \b + \a = \a \cdot \b + 0 = \a \cdot \b$, which is a limit ordinal by the induction hypothesis.
        On the other hand if $\a \neq 0$ then by Lemma~6.5.4.1 $\a \cdot (\b+1) = \a \cdot \b + \a$ is a limit ordinal since $\a$ is.

        Lastly suppose that $\b$ is a limit ordinal and that $\a \cdot \g$ is a limit ordinal for all $\g < \b$.
        Let $A = \braces{\a \cdot \g \where \g < \b}$ so that by Definition~6.5.6b we have $\a \cdot \b = \sup{A}$.
        Consider any $\d < \a \cdot \b$ so that $\d$ is not an upper bound of $A$.
        Hence there is a $\g < \b$ such that $\d < \a \cdot \g$.
        Then by the induction hypothesis $\a \cdot \g$ is a limit ordinal so that $\d+1 < \a \cdot \g$ by Lemma~6.5.1.2.
        But then we have $\d + 1 < \a \cdot \g \leq \sup{A} = \a \cdot \b$.
        Thus by Lemma~6.5.1.2 this shows that $\a \cdot \b$ is also a limit ordinal, which completes the inductive proof. \qedsymbol
    }

    \begin{statement}{Lemma 6.5.6.3}
        An ordinal  $\a = \w \cdot n$ for a natural number $n$ if and only if $\a$ is a limit ordinal and $\a < \w^2$.
    \end{statement}

    \proof{
        ($\to$) Suppose that $\a = \w \cdot n$ for natural number $n$.
        Then by Lemma~6.5.6.2 $\a$ is a limit ordinal since $\w$ is.
        Also clearly by Exercise~6.5.7a $\a = \w \cdot n < \w \cdot (n+1) \leq \sup\braces{\w \cdot k \where k < \w} = \w \cdot \w = \w^2$.

        ($\leftarrow$) Now suppose that $\a$ is a limit ordinal and $\a < \w^2$.
        We have $a < \w^2 = \w \cdot \w = \sup\braces{\w \cdot n \where n < \w}$ so that $\a$ is not an upper bound of $\braces{\w \cdot n \where n < \w}$.
        Hence there is a $k < \w$ such that $\a < \w \cdot k$.
        It therefore follows that the set $A = \braces{n \in \w \where \a \leq \w \cdot n}$ is nonempty.
        Since $A$ is nonempty set of natural numbers (which are well-ordered) it follows that $A$ has a least element $n$.
        If $n=0$ it follows that $\a = 0 = \w \cdot 0$ since $\a \leq \w \cdot n = \w \cdot 0 = 0$ and $0$ is the only ordinal for which this is true.
        Then if $n > 0$ we have that $n-1$ is a natural number and moreover it follows that $\w \cdot (n-1) < \a$ since otherwise $n-1$ would have been the least element of $A$.

        Thus we have that $\w \cdot (n-1) < \a \leq \w \cdot n$.
        But since $\w \cdot n = \w \cdot\squares{(n-1) + 1} = \w \cdot (n-1) + \w$ clearly any ordinal $\g$ where $\w \cdot (n-1) + 0 = \w \cdot (n-1) < \g < \w \cdot n = \w \cdot (n-1) + \w$ must have the form $\w \cdot (n-1) + m$ for a natural number $m > 0$.
        From this it clearly follows that $\g$ is a successor ordinal.
        Since $\a$ is a limit ordinal it can thus not be such a $\g$ so that it cannot be that $\a < \w \cdot n$.
        But since we have established that $\a \leq \w \cdot n$ (since $n \in A$) it has to be that $\a = \w \cdot n$. \qedsymbol
    }

    \begin{statement}{Lemma 6.5.6.4}
        An ordinal $\a = \w \cdot n + k$ for natural numbers $n$ and $k$ if and only if $\a < \w^2$.
    \end{statement}

    \proof{
        ($\to$) Suppose that $\a = \w \cdot n + k$ for natural numbers $n$ and $k$.
        We then have
        \ali{
            \a &= \w \cdot n + k = (\w \cdot n + k) + 0 \\
            &< (\w \cdot n + k) + \w & \text{(by Lemma~6.5.4a since $0 < \w$)} \\
            &= \w \cdot n + (k + \w) & \text{(by the associative property)} \\
            &= \w \cdot n + \w & \text{(by Lemma~6.5.5.3)} \\
            &= \w \cdot (n+1) & \text{(by Definition~6.5.6b)} \\
            &\leq \sup\braces{\w \cdot m \where m < \w} \\
            &= \w \cdot \w & \text{(Definition~6.5.6c)} \\
            &= \w^2 & \text{(by Example~6.5.10a)}
        }
        as desired.

        ($\leftarrow$) Now suppose that $\a < \w^2$.
        By Exercise~6.5.4 we have that $\a = \b + k$ for a unique limit ordinal $\b$ and natural number $k$.
        We also have by Lemma~6.5.4 that $\b + 0 \leq \b + k = \a < \w^2$ since obviously $0 \leq k$.
        Hence $\b$ is a limit ordinal such that $\b < \w^2$ so that by Lemma~6.5.6.3 there is a natural number $n$ such that $\b = \w \cdot n$, thereby proving the result since this means that $\a = \b + k = \w \cdot n + k$. \qedsymbol
    }
    
    \mainprob

    We claim that $\w^2$ is the first additive ordinal after $\w$.

    To see this we first show that $\w^2$ is a limit ordinal.
    So consider any $\a < \w^2$ so that by Lemma~6.5.6.4 there are natural numbers $n$ and $k$ such that $\a = \w \cdot n + k$.
    We then have $\a + 1 = (\w \cdot n + k) + 1 = \w \cdot n + (k+1) < \w^2$ again by Lemma~6.5.6.4 since $k+1$ is a natural number.
    Hence $\w^2$ is a limit ordinal by Lemma~6.5.1.2.

    Next we show that $\w^2$ is an additive ordinal.
    So again consider $\a < \w^2$ so that by Lemma~6.5.6.4 there are natural numbers $n$ and $k$ such that $\a = \w \cdot n + k$.
    We then have
    $$
    \a + \w^2 = (\w \cdot n + k) + \w^2 = \w \cdot n + (k + \w^2) = \w \cdot n + \w^2 = \w \cdot n + \w \cdot \w
    = \w \cdot (n + \w) = \w \cdot \w = \w^2
    $$
    since $k + \w^2 = \w^2$ and $n + \w = \w$ by Lemma~6.5.5.3.

    Lastly we show that if $\w < \a < \w^2$ then $\a$ is not an additive ordinal.
    Clearly by Lemma~6.5.6.4 there are natural numbers $n$ and $k$ such that $\a = \w \cdot n + k$.
    Now let $\b = \w$ so that clearly $\b < \a$.
    We then have that
    $$
    \b + \a = \w + \w \cdot n + k = \w \cdot 1 + \w \cdot n + k = \w \cdot (1 + n) + k = \w \cdot (n+1) + k \,.
    $$
    We also clearly have
    \ali{
        n &< n+1 \\
        \w \cdot n &< \w \cdot (n+1) & \text{(by Exercise~6.5.7a)} \\
        \w \cdot n + k &<  \w \cdot (n+1) + k & \text{(by Lemma~6.5.5.1}) \\
        \a &< \b + \a
    }
    so that $\b + \a \neq \a$.
    Since $\b < \a$ this shows that $\a$ is not an additive ordinal. \qedsymbol
\end{solution}

\question{6.5.7}

\begin{solution}
	(a) ($\to$) We show this by transfinite induction on $\a_2$.
        So suppose and that  $\a_1 < \d$ implies that $\b \cdot \a_1 < \b \cdot \d$ for all $\d < \a_2$ and that $\a_1 < \a_2$.
        If $\a_2$ is a successor ordinal then $\a_2 = \d + 1$ for some ordinal $\d$ where $\a_1 \leq \d$ since $\a_1 < \a_2$.
        Then we have
        \ali{
            \b \cdot \a_1 &\leq \b \cdot \d & \text{(by the induction hypothesis if $\a_1 < \d$ and trivially if $\a_1 = \d$)} \\
            &< \b \cdot \d + \b & \text{(by Lemma~6.5.4a since $0 < \b$)} \\
            &= \b \cdot (\d + 1) & \text{(by Definition~6.5.6b)} \\
            &= \b \cdot \a_2 \,.
        }
        On the other hand if $\a_2$ is a limit ordinal then $\a_1 + 1 < \a_2$ since $\a_1 < \a_2$.
        Then we have that
        \ali{
            \b \cdot \a_1 &< \b \cdot (\a_1 + 1) & \text{(by the induction hypothesis since $\a_1 < \a_1 +1 < \a_2$)} \\
            &\leq \sup_{\d < \a_2} \b \cdot \d & \text{(since the supremum is an upper bound)} \\
            &= \b \cdot \a_2 \,. & \text{(by Definition~6.5.6c)}
        }
        This completes the inductive proof.

        ($\leftarrow$) For this we assume that $\b \cdot \a_1 < \b \cdot \a_2$.
        If it were the case that $\a_1 > \a_2$ than it would follow by the implication already shown that $\b \cdot \a_1 > \b \cdot \a_2$, which is a contradiction.
        Similarly if $\a_1 = \a_2$ then $\b \cdot \a_1 = \b \cdot \a_2$, another contradiction.
        Hence by the linearity of the order $<$ it must be that $\a_1 < \a_2$ as desired. \qedsymbol

    (b) This follows from part (a) in the same way as the proof of Lemma~6.5.4b but is repeated for completeness.

    ($\to$) We prove this part by contrapositive, so suppose that $\a_1 \neq \a_2$.
    Then either $\a_1 < \a_2$ or $\a_1 > \a_2$.
    In the former case then part (a) implies that $\b \cdot \a_1 < \b \cdot \a_2$ so that $\b \cdot \a_1 \neq \b \cdot \a_2$.
    In the latter case part (a) similarly implies that $\b \cdot \a_1 > \b \cdot \a_2$ so that again $\b \cdot \a_1 \neq \b \cdot \a_2$.

    ($\leftarrow$) If $\a_1 = \a_2$ then we trivially have $\b \cdot \a_1 = \b \cdot \a_2$. \qedsymbol
\end{solution}

\question{6.5.8}

\begin{solution}
	(a) We show this by transfinite induction on $\g$.
    For $\g = 0$ we clearly have
    $$
    \a + \g = \a + 0 = \a < \b = \b + 0 = \b + \g
    $$
    so that $\a + \g \leq \b + \g$ is true.
    Now suppose that $\a + \g \leq \b + \g$ so that we have
    $$
    \a + (\g + 1) = (\a + \g) + 1 \leq (\b + \g) + 1 = \b + (\g + 1)
    $$
    by Lemma~6.5.5.1 and Corollary~6.5.5.2 since $\a + \g \leq \b + \g$ (induction hypothesis) and $1$ is a natural number.

    Lastly, suppose that $\g$ is a nonzero limit ordinal and that $\a + \d \leq \b + \d$ for all $\d < \g$.
    Let $A = \braces{\a + \d \where \d < \g}$ and tentatively suppose that $\a + \g > \b + \g$.
    Then $\b + \g < \a + \g = \sup{A}$ so that $\b + \g$ is not an upper bound of $A$.
    Hence there is a $\d < \g$ such that $\b + \g < \a + \d$.
    We then have that
    $$
    \b + \d \leq \sup\braces{\b + \e \where \e < \g} = \b + \g < \a + \d \,,
    $$
    but this contradicts the induction hypothesis since $\d < \g$.
    So it has to be that $\a + \g \leq \b + \g$ as desired.
    This completes the inductive proof.

    Furthermore we give an example that shows that the $\leq$ cannot be replaced with $<$ in the conclusion.
    Let $\a = 1$, $\b = 2$, and $\g = \w$.
    Then clearly $\a = 1 < 2 = \b$ but we also have
    $$
    \a + \g = 1 + \w = \w = 2 + \w = \b + \g
    $$
    so that clearly $\a + \g < \b + \g$ is not true since they are equal.

    Note also that clearly if $\a = \b$ then $\a + \g = \b + \g$ so that $\a + \g \leq \b + \g$ is still true.
    Hence the conclusion is also true in the slightly more general case of $\a \leq \b$. \qedsymbol

    (b) We also show this by transfinite induction on $\g$.
    For $\g = 0$ we clearly have
    $$
    \a \cdot \g = \a \cdot 0 = 0 = \b \cdot 0 = \b \cdot \g
    $$
    so that $\a \cdot \g \leq \b \cdot \g$ is true.
    Now suppose that $\a \cdot \g \leq \b \cdot \g$ so that we have
    \ali{
        \a \cdot (\g + 1) &= \a \cdot \g + \a & \text{(by Definition~6.5.6b)} \\
        &< \a \cdot \g + \b & \text{(by Lemma~6.5.4 since $\a < \b$)} \\
        &\leq \b \cdot \g + \b & \text{(by part (a) and the induction hypothesis)} \\
        &= \b \cdot (\g + 1) & \text{(by Definition~6.5.6b again)} \,.
    }

    Lastly, suppose that $\g$ is a nonzero limit ordinal and that $\a \cdot \d < \b \cdot \d$ for all $\d < \g$.
    The argument is analogous to that in part (a).
    Let $A = \braces{\a \cdot \d \where \d < \g}$ and tentatively suppose that $\a \cdot \g > \b \cdot \g$.
    Then $\b \cdot \g < \a \cdot \g = \sup{A}$ so that $\b \cdot \g$ is not an upper bound of $A$.
    Hence there is a $\d < \g$ such that $\b \cdot \g < \a \cdot \d$.
    We then have that
    $$
    \b \cdot \d \leq \sup\braces{\b \cdot \e \where \e < \g} = \b \cdot \g < \a \cdot \d \,,
    $$
    but this contradicts the induction hypothesis since $\d < \g$.
    So it has to be that $\a \cdot \g \leq \b \cdot \g$ as desired.
    This completes the inductive proof.

    A case in which $\a < \b$ but $\a \cdot \g = \b \cdot \g$ is clearly provided when $\g = 0$ so that $\leq$ in the conclusion cannot be replaced with $<$. Another example is if $\a$ and $\b$ are natural numbers (such that $\a < \b$) and $\g = \w$ so that
    $$
    \a \cdot \g = \a \cdot \w = \w = \b \cdot \w = \b \cdot \g
    $$
    by Lemma~6.5.5.3.

    Similarly here if $\a = \b$ then $\a \cdot \g = \b \cdot \g$ so that $\a \cdot \g \leq \b \cdot \g$ is still true.
    Hence the conclusion is also true in the slightly more general case of $\a \leq \b$. \qedsymbol
\end{solution}

\question{6.5.9}

\begin{solution}
    \begin{statement}{Lemma 6.5.9.1}
        If $n > 0$ is a natural number then $n \cdot \w = \w$.
    \end{statement}

    \proof{
        By Definition~6.5.6 we have that $n \cdot \w = \sup\braces{n \cdot k \where k < \w}$ but clearly $n \cdot k$ is a natural number for any $k < \w$ so that $n \cdot \w = \sup\braces{n \cdot k \where k < \w} = \sup\braces{k \where k < \w} = \w$. \qedsymbol
    }

    \mainprob

	(a) Let $\a = 1$, $\b = 2$, and $\g = \w$ so that
    $$
    \a + \g = 1 + \w = \w = 2 + \w = \b + \g
    $$
    by Lemma~6.5.5.3 but $\a = 1 < 2 = \b$ so that $\a \neq \b$.

    (b) Again let $\a = 1$, $\b = 2$, and $\g = \w$ so that $\g = \w > 0$ and
    $$
    \a \cdot \g = 1 \cdot \w = \w = 2 \cdot \w = \b \cdot \g
    $$
    by Lemma~6.5.9.1.
    Clearly though $\a = 1 < 2 = \b$ so that $\a \neq \b$.

    (c) Here let $\a = w$, $\b = 1$, and $\g = 2$.
    Then we have
    $$
    (\b + \g) \cdot \a = (1 + 2) \cdot \w = 3 \cdot \w = \w
    $$
    by Lemma~6.5.9.1, whereas
    $$
    \b \cdot \a + \g \cdot \a = 1 \cdot \w + 2 \cdot \w = \w + \w = \w \cdot 2 \,,
    $$
    where we have used Lemma~6.5.9.1 here as well as Example~6.5.7b.
    That $(\b + \g) \cdot \a = \w = \w \cdot 1 \neq \w \cdot 2 = \b \cdot \a + \g \cdot \a$ follows from Exercise~6.5.7b.
\end{solution}

\question{6.5.10}

\begin{solution}
    \begin{statement}{Lemma 6.5.10.1}
        If $\a$ is limit ordinal and $\b$ any other ordinal then the product $\a \cdot \b$ is a limit ordinal.
    \end{statement}

    \proof{
        First note that if $\a = 0$ then clearly by Lemma~6.5.1.1 we have $\a \cdot \b = 0 \cdot \b = 0$, which is clearly a limit ordinal.
        We therefore assume henceforth that $\a$ is a nonzero limit ordinal.
        We proceed by transfinite induction on $\b$.
        For $\b = 0$ we clearly have that $\a \cdot \b = \a \cdot 0 = 0$, which is a limit ordinal by definition.
        Now suppose that $\a \cdot \b$ is a limit ordinal.
        Then $\a \cdot (\b+1) = \a \cdot \b + \a$ by Definition~6.5.6b.
        Then since $\a \neq 0$ the sum $\a \cdot \b + \a = \a \cdot (\b+1)$ is a limit ordinal by Lemma~6.5.4.1.

        Lastly suppose that $\b \neq 0$ is a limit ordinal and that $\a \cdot \d$ is a limit ordinal for every $\d < \b$.
        Consider then any $\g < \a \cdot \b$ so that since by Definition~6.5.6c $\a \cdot \b = \sup\braces{\a \cdot \d \where \d < \b}$ it follows that $\g$ is not an upper bound of $\braces{\a \cdot \d \where \d < \b}$.
        Thus there is a $\d < \b$ such that $\g < \a \cdot \d$.
        Then since $\a \cdot \d$ is a limit ordinal by the inductive hypothesis we have that $\g + 1 < \a \cdot \d$ by Lemma~6.5.1.2.
        Also since $\d < \b$ we have that $\a \cdot \d < \a \cdot \b$ by Exercise~6.5.7a since $\a \neq 0$.
        Hence we have $\d + 1 < \a \cdot \d < \a \cdot \b$, which shows that $\a \cdot \b$ is a limit ordinal by Lemma~6.5.1.2.
        This completes the inductive proof. \qedsymbol
    }

    \begin{statement}{Lemma 6.5.10.2}
        If $\a$ is an ordinal then $\a < \a + \w$ and $\a + \w$ is the next limit ordinal after $\a$, i.e. every ordinal $\b$ such that $\a < \b < \a + \w$ is a successor ordinal.
    \end{statement}

    \proof{
        Consider any ordinal $\a$.
        First we note that clearly $\a = \a + 0 < \a + \w$ by Lemma~6.5.6a since $0 < \w$.
        It also clearly follows from  Lemma~6.5.4.1 that $\a + \w$ is a limit ordinal.
        
        Now suppose that $\b$ is any ordinal such that $\a < \b < \a + \w$.
        Since $\a < \b$ there is a unique ordinal $\g$ such that $\a + \g = \b$ by Lemma~6.5.5.
        Now, since $\a + 0 = \a < \b = \a + \g$ it follows again from Lemma~6.5.6a that $0 < \g$.
        Similarly we have $\a + \g = \b < \a + \w$ so that by the same lemma $\g < \w$.
        Hence $0 < \g < \w$ so that $\g$ is a nonzero natural number.
        In particular $\g \geq 1$ so that $n = \g - 1$ is also a natural number and $\g = n + 1$.
        We then have that $\b = \a + \g = \a + (n + 1) = (\a + n) + 1$ so that clearly $\b$ is a successor ordinal. \qedsymbol
    }

    \mainprob

	($\to$) Suppose to the contrary that not every limit ordinal is equal to $\w \cdot \b$ for some $\b$.
    Then let $\a$ be a limit ordinal such that $\a \neq \w \cdot \g$ for every ordinal $\g$.
    Now let $\b$ be the set of ordinals $\d$ such that $\w \cdot \d < \a$, noting that it could potentially be the empty set.
    We claim that $\b$ is an ordinal number and that moreover it is a limit ordinal.

    Clearly if $\b = \es = 0$ then it is a limit ordinal, so assume that $\b \neq \es$.
    Then consider any $\d \in \b$ so that $\w \cdot \d < \a$.
    Also consider any $x \in \d$ so that $x$ is also an ordinal number by Lemma~6.2.8 and moreover that $x < \d$.
    It then follows from Exercise~6.5.7a that $\w \cdot x < \w \cdot \d < \a$.
    Hence we have that $x \in \b$ so that $\d \ss \b$ since $x$ was arbitrary.
    Since $\d$ was also arbitrary this shows that $\b$ is transitive.
    Also since $\b$ is nonempty set of ordinals it follows from Theorem~6.2.6d that $\b$ is well-ordered.
    Thus by definition $\b$ is an ordinal number.

    Now consider any $\d < \b$ so that $\d \in \b$ so that $\w \cdot \d < \a$.
    By Lemma~6.5.10.2 we have that $\w \cdot \d + \w = \w \cdot (\d + 1)$ is the next limit ordinal after $\w \cdot \d$ (i.e. there are no limit ordinals between them) so it has to be that $\w \cdot (\d + 1) \leq \a$ since otherwise $\a$ would be a limit ordinal between $\w \cdot \d$ and $\w \cdot (\d + 1)$.
    But since $\a \neq \w \cdot \g$ for all ordinals $\g$ it cannot be that $\a = \w \cdot (\d + 1)$.
    Thus it must be that $\w \cdot (\d + 1) < \a$ so that $\d + 1 \in \b$ so that $\d + 1 < \b$.
    This shows that $\b$ is a limit ordinal by Lemma~6.5.1.2.

    Now we claim that $\w \cdot \b = \a$, which is of course is a contradiction that proves the desired result.
    To see this let $A = \braces{\w \cdot \d \where \d < \b}$ so that by Definition~6.5.6c we have that $\w \cdot \b = \sup{A}$ since $\b$ is a limit ordinal.
    Now since $\d < \b$ means that $\d \in \b$ so that $\w \cdot \d < \a$ by the definition of $\b$, clearly $\a$ is an upper bound of $A$ so that $\w \cdot \b = \sup{A} \leq \a$.
    However, if it were the case that $\w \cdot \b < \a$ then we would have by definition that $\b \in \b$ which contradicts Lemma~6.2.7 since we have shown that $\b$ is an ordinal.
    Thus the only possibility is that $\w \cdot \b = \a$, which gives rise to the contradiction.

    Moreover, we can show that $\b$ is unique.
    To see this, consider $\b_1$ and $\b_2$ where $\a = \w \cdot \b_1$ and $\a = \w \cdot \b_2$.
    Then clearly $\w \cdot \b_1 = \w \cdot \b_2$ so that $\b_1 = \b_2$ by Exercise~6.5.7b since clearly $\w \neq 0$.

    ($\leftarrow$) Suppose that $\a = \w \cdot \b$ for some ordinal $\b$.
    Then the result that $\a$ is a limit ordinal follows immediately from Lemma~6.5.10.1 above since $\w$ is a limit ordinal. \qedsymbol
\end{solution}

\def\ex{6.5.11}
\setcounter{itm}{0}
\question{\ex}

\begin{solution}
    \def\Dx{\Delta x}

% This ended up not being needed but kept it in case it is ever needed since I went through the trouble of proving it
\iffalse
    \begin{statement}{Lemma~\ex.\itm{lem:emb:gel}}
        An ordinal $\a$ has a greatest element if and only if it is a successor ordinal.
    \end{statement}

    \proof{
        ($\to$) Suppose that $\a$ has a greatest element $\b$.
        We claim that $\a = \b \cup \braces{\b} = \b+1$.
        So first consider any $x \in \a$ so that $x \leq \b$ since $\b$ is the greatest element.
        If $x = \b$ then clearly $x \in \b \cup \braces{\b}$ so assume that $x < \b$.
        Then $x \in \b$ so that also clearly $x \in \b \cup \braces{\b}$.
        Hence in either case $x \in \b \cup \braces{\b}$ so that $\a \ss \b \cup \braces{\b}$ since $x$ was arbitrary.
        Now consider any $y \in \b \cup \braces{\b}$.
        If $y = \b$ then clearly $y = \b \in \a$ since $\b$ is the greatest element of $\a$.
        On the other hand if $y \in \b$ then since $\a$ is transitive (since it is an ordinal) and $\b \in \a$ it follows that $y \in \a$ as well.
        Thus in either case $y \in \a$ so that $\b \cup \braces{\b} \ss \a$ since $y$ was arbitrary.
        This shows that $\a = \b \cup \braces{\b} = \b+1$ and hence is a successor ordinal.

        ($\leftarrow$) Suppose that $\a$ is a successor ordinal, in  particular that $\a = \b+1 = \b \cup \braces{\b}$ for an ordinal $\b$.
        Consider any $x \in \a$.
        Then either $x \in \b$ so that $x < \b$ or $x = \b$.
        Hence in either case $x \leq \b$.
        Since $x$ was arbitrary this shows that $\b$ is the greatest element of $\a$. \qedsymbol
    }
\fi

    \begin{statement}{Lemma~\ex.\itm{lem:emb:isord}}
        If $\b$ is an initial segment of an ordinal $\a$ then $\b$ is also an ordinal.
    \end{statement}

    \proof{
        By Lemma~6.1.2 there is an $a \in \a$ such that $\b = \braces{x \in \a \where x < a}$.
        Also, since $a \in \a$ and $\a$ is an ordinal, $a$ is an ordinal as well by Lemma~6.2.8.
        Finally, by the comments after Theorem~6.2.10 $\b$ is simply the ordinal $a$ since $\b = \braces{x \in \a \where x < a}$ and each $x$ in that set is an ordinal (by Lemma~6.2.8 since each $x \in \a$ and $\a$ is an ordinal).

        Since $\b$ is an initial segment of $\a$, by definition $\b \pss \a$.
        From this it clearly follows that $\b$ is well-ordered since $\a$ is (since for any $B \ss \b$ it is also true that $B \ss \a$ and so $B$ has a least element since $\a$ is well-ordered).
        Thus we need only show that $\b$ is transitive to complete the proof that it is an ordinal.
        So consider any $X \in \b$ and any $x \in X$. \qedsymbol
    }

    Next we need to build up a little theory.
    
    \begin{statement}{Definition~\ex.\itm{def:emb:emb}}
        For an ordinal $\a$ we call a set $E_\a$ an \emph{embedding} of $\a$ if it has the following properties:
        \begin{enumerate}
            \item $E_\a$ is a subset of $Q$.
            \item $E_\a$ is order isomorphic to $\a$ under the usual ordering of the rationals.
            \item For some $a$ and $b$ in $\rats$, $a \leq x < b$ for every $x \in E_\a$. We denote this by saying $a \leq E_\a < b$ or that $E_\a$ is an embedding \emph{in} $[a,b)$.
            \end{enumerate}
            We call $U_\a$ a \emph{unit embedding} of $\a$ if it is an embedding of $\a$ in $[0,1)$.
    \end{statement}

    \begin{statement}{Theorem~\ex.\itm{thrm:emb:dx}}
        If $E_\a$ is an embedding of $\a$ in $[a, b)$ then for each $x \in E_\a$ there is a $\Dx \in \prats$ such that $x + \Dx \leq b$ and if $y \in \rats$ and $x < y < x+\Dx$ then $y$ is not in $E_\a$.
        That is, $x$ is not a limit point from the right.
    \end{statement}

    \proof{
        Consider any $x \in E_\a$, noting that $x < b$.

        If $x$ is the greatest element of $E_\a$ then let $\Dx = b - x$, noting that $\Dx > 0$ since $b > x$.
        We also have
        $$
        x + \Dx = x + b - x = b \leq b \,.
        $$
        Then consider any $y \in \rats$ where $x < y < x + \Dx$ so that clearly $y \notin E_\a$ since otherwise $x$ would not be the greatest element of $E_\a$.

        On the other hand if $x$ is not the greatest element of $E_\a$ then let $f$ be the isomorphism between $\a$ and $E_\a$ and let $\b = \inv{f}(x)$.
        It follows that $\b$ is not the greatest element of $\a$ so that $\b + 1 \in \a$ as well.
        Then let $\Dx = f(\b+1) - x$, noting that $\Dx > 0$ since $f(\b+1) > f(\b) = x$ since $f$ is an isomorphism.
        We also have
        $$
        x + \Dx = x + f(\b+1) - x = f(\b+1) < b
        $$
        since $f(\b+1) \in E_\a$.
        Now consider any $y \in \rats$ where $f(\b) = x < y < x + \Dx = f(\b+1)$.
        If it were the case that $y \in E_\a$ then we would have that $\b < \inv{f}(y) < \b + 1$ since $f$ is an isomorphism, which is impossible since $\b$ is an ordinal.
        So it must be that $y \notin E_\a$ as desired. \qedsymbol
    }

    \begin{statement}{Corollary~\ex.\itm{cor:emb:dxi}}
        If $E_\a$ is an embedding and $x$ and $y$ are in $E_\a$ where $x < y$ then $x + \Dx \leq y$, where $\Dx \in \prats$ is that guaranteed by Theorem~\ex.\ref{thrm:emb:dx}.
    \end{statement}

    \proof{
        If it were the case that $x + \Dx > y$ then we have that $x < y < x + \Dx$, which is in direct contradiction to Theorem~\ex.\ref{thrm:emb:dx} since $y \in E_\a$.
    }

    For an embedding $E_\a$ consider $p \in \prats$ and $q \in \rats$.
    We define a set denoted by $pE_\a + q$ to be the set $\braces{px + q \where x \in E_\a}$.

    \begin{statement}{Theorem~\ex.\itm{thrm:emb:scale}}
        If $E_\a$ is an embedding in $[a, b)$ then $F_\a = pE_\a + q$ is also an embedding of $\a$ for any $p \in \prats$ and $q \in \rats$.
        Moreover $pa + q \leq F_\a < pb + q$.
    \end{statement}

    \proof{
        So first consider any $y \in F_\a$ so that $y = px + q$ for some $x \in E_\a$.
        Since $E_\a$ is an embedding $x \in \rats$ so that clearly $y = px + q \in \rats$ as well since $p,q \in \rats$.
        Hence since $y$ was arbitrary we have that $F_\a \ss \rats$ so that (1) is satisfied.

        Now consider the mapping $f : E_\a \to F_\a$ defined by $f(x) = px + q$ for $x \in E_\a$.
        Clearly $F_\a = \braces{f(x) \where x \in E_\a}$ so that $f$ is onto.
        Consider then $x,y \in E_\a$ where $x < y$ so that we have
        \ali{
            x &< y \\
            px &< py & \text{(since $p > 0$)} \\
            px + q &< py + q \\
            f(x) &< f(y) \,.
        }
        Hence $f$ is an isomorphism.
        Thus $F_\a$ is isomorphic to $E_\a$ so that clearly it is also isomorphic to $\a$ since $E_\a$ is, thereby showing (2).

        Lastly for any $y \in F_\a$ we have $y = px + q$ for some $x \in E_\a$.
        We then have
        \ali{
            a &\leq x < b \\
            pa &\leq px < pb & \text{(since $p > 0$)} \\
            pa + q &\leq px + q < pb + q \\
            pa + q &\leq  y < pb + q \,.
        }
        This shows both (3) and the last statement. \qedsymbol
    }

    For an embedding $E_\a$ in $[a, b)$ and a unit embedding $U_\b$ for ordinals $\a$ and $\b$ we define the product
    $$
    E_\a \cdot U_\b = \bigcup_{x \in E_\a} \parens{\Dx \cdot U_\b + x} \,,
    $$
    where $\Dx \in \prats$ is that guaranteed to exist by Theorem~\ex.\ref{thrm:emb:dx}.

    \begin{statement}{Theorem \ex.\itm{thrm:emb:prod}}
        For an embedding $E_\a$ in $[a, b)$ and a unit embedding $U_\b$ the product $E_\a \cdot U_\b$ is an embedding of $\b \cdot \a$ in $[a, b)$.
    \end{statement}

    \proof{
        First consider any $y \in E_\a \cdot U_\b$ so that there is an $x \in E_\a$ such that $y \in \Dx \cdot U_\b + x$.
        Since $y \in \Dx \cdot U_\b + x$ is an embedding by Theorem~\ex.\ref{thrm:emb:scale} it follows that $y \in \rats$, which shows (1) since $y$ was arbitrary.

        Now since $E_\a$ is an embedding of $\a$ there is an isomorphism $f: \a \to E_\a$.
        Similarly there is an isomorphism $g: \b \to U_\b$ since $U_\b$ is an embedding of $\b$.
        Consider $\a \times \b$ with lexicographic ordering $\prec$.
        Then define a mapping $h: \a \times \b \to E_\a \cdot U_\b$ by
        $$
        h(\d, \e) = \Delta f(\d) \cdot g(\e) + f(\d)
        $$
        for $(\d, \e) \in \a \times \b$, noting that $\Delta f(\d)$ is that guaranteed by Theorem~\ex.\ref{thrm:emb:dx} since $f(\d) \in E_\a$.
        
        First we claim that $h$ is surjective.
        So consider any $z \in E_\a \cdot U_\b$ so that there is an $x \in E_\a$ such that $z \in \Dx \cdot U_\b + x$.
        By definition then there is a $y \in U_\b$ such that $z = \Dx \cdot y + x$.
        Now let $\d = \inv{f}(x)$ and $\e = \inv{g}(y)$ so that $x = f(\d)$ and $y = g(\e)$, which can be done since $f$ and $g$ are bijections.
        We then have that
        $$
        h(\d, \e) = \Delta f(\d) \cdot g(\e) + f(\d) = \Delta x \cdot y + x = z \,,
        $$
        which shows that $h$ is surjective since $z$ was arbitrary.

        We also claim that $h$ is an isomorphism and therefore also injective.
        So consider any $(\d_1, \e_1)$ and $(\d_2, \e_2)$ in $\a \times \b$ where $(\d_1, \e_1) \prec (\d_2, \e_2)$.
        By the definition of lexicographic ordering we have the following:

        Case: $\d_1 < \d_2$.
        Then since $f$ is an isomorphism $f(\d_1) < f(\d_2)$, and also by Corollary~\ex.\ref{cor:emb:dxi} it follows that $f(\d_1) < f(\d_1) + \Delta f(\d_1) \leq f(\d_2)$.
        We also have that
        $$
        \Delta f(\d_1) \cdot g(\e_1) + f(\d_1) < \Delta f(\d_1) + f(\d_1)
        $$
        by Theorem~\ex.\ref{thrm:emb:scale} since $g(\e_1) \in U_\b$ and $U_\b$ is a unit embedding.
        Also since $g(\e_2) \geq 0$ (since $g(\e_2) \in U_\b$) and $\Delta f(\d_2) > 0$ (by Theorem~\ex.\ref{thrm:emb:dx}) that
        $$
        f(\d_2) \leq \Delta f(\d_2) \cdot g(\e_2) + f(\d_2) \,.
        $$
        Combining all this results in
        $$
        h(\d_1, \e_1) = \Delta f(\d_1) \cdot g(\e_1) + f(\d_1) < \Delta f(\d_1) + f(\d_1) \leq f(\d_2) = h(\d_2, \e_2) \,.
        $$

        Case: $\d_1 = \d_2$ and $\e_1 < \e_2$.
        Then obviously $f(\d_1) = f(\d_2)$ so that $\Delta f(\d_1) = \Delta f(\d_2)$ but also $g(\e_1) < g(\e_2)$ since $g$ is an isomorphism.
        We then have
        \ali{
            g(\e_1) &< g(\e_2) \\
            \Delta f(\d_1) \cdot g(\e_1) &< \Delta f(\d_1) \cdot g(\e_2) & \text{(since $\Delta f(\d_1) > 0$)} \\
            \Delta f(\d_1) \cdot g(\e_1) + f(\d_1) &< \Delta f(\d_1) \cdot g(\e_2) + f(\d_1) \\
            \Delta f(\d_1) \cdot g(\e_1) + f(\d_1) &< \Delta f(\d_2) \cdot g(\e_2) + f(\d_2) & \text{(since $\Delta f(\d_1) = \Delta f(\d_2)$ and $f(\d_1) = f(\d_2)$)} \\
            h(\d_1, \e_1) &< h(\d_2, \e_2) \,.
        }
        
        Thus in all cases $h(\d_1, \e_1) < h(\d_2, \e_2)$, which shows that $h$ is an isomorphism since $(\d_1, \e_1)$ and $(\d_2, \e_2)$ were arbitrary.
        Hence $E_\a \cdot U_\b$ is isomorphic to the lexicographic ordering of $\a \times \b$ and therefore also to $\b \cdot \a$ by Theorem~6.5.8.
        This shows part (2) of the embedding definition.

        Lastly consider any $z \in E_\a \cdot U_\b$ so that there is an $x \in E_\a$ such that $z \in \Dx \cdot U_\b + x$.
        Then since $U_\b < 1$ we have that $z < \Dx + x \leq b$ by Theorems~\ex.\ref{thrm:emb:scale} and \ex.\ref{thrm:emb:dx}.
        Also since $0 \leq U_\b$ it follows from Theorem~\ex.\ref{thrm:emb:scale} that $a \leq x \leq z$ since $a \leq E_\a$.
        Since $z$ was arbitrary this shows that $a \leq E_\a \cdot U_\b < b$, which shows (3).
        This completes the proof. \qedsymbol
    }

    \begin{statement}{Theorem~\ex.\itm{thrm:emb:sup}}
    Suppose that $\a$ is a limit ordinal and that $\braces{\a_n}$ is a sequence ($n \in \w$) of nonzero ordinals in $\a$.
    Also suppose that $E_\w$ is an embedding of $\w$ in $[a,b)$ and $U_{\a_n}$ is a unit embedding of $\a_n$ for each $n \in \w$.
    Suppose further that $\a = \sup_{n \in \w} \a_n$ and that $\a_n + \a_{n+1} = \a_{n+1}$ for every $n \in \w$, noting that clearly $n+1 \in \w$ as well.
    Lastly, suppose that $f$ is the isomorphism from $\w$ to $E_\w$.
    Let
    $$
    A_n = \Delta f(n) \cdot U_{\a_n} + f(n)
    $$
    for $n \in \w$.
    Then the set
    $$
    E_\a = \bigcup_{n \in \w} A_n
    $$
    is an embedding of $\a$ in $[a,b)$.
    \end{statement}

    \proof{
        First consider any $n$ and $m$ in $\w$ where $n \neq m$.
        We can assume without loss of generality $n < m$.
        Then $f(n) < f(m)$ since $f$ is an isomorphism and, moreover, it follows from Corollary~\ex.\ref{cor:emb:dxi} that $\Delta f(n) + f(n) \leq f(m)$.
        Hence by Theorem~\ex.\ref{thrm:emb:scale} we have that
        $$
        A_n = \Delta f(n) \cdot U_{\a_n} + f(n) < \Delta f(n) + f(n) \leq f(m) \leq \Delta f(m) \cdot U_{\a_m} + f(m) = A_m
        $$
        since $U_{\a_n}$ and $U_{\a_m}$ are unit embeddings.
        Hence all the $A_n$ are disjoint and moreover $A_0 < A_1 < A_2 < \ldots$.
        Also clearly by Theorem~\ex.\ref{thrm:emb:scale} each $A_n$ is isomorphic to $\a_n$ since $U_{\a_n}$ is.

        Now for $p,q \in \rats$ let $[p,q) = \braces{x \in \rats \where q \leq x < q}$.
        We then claim that $E_\a \cap [a, f(n+1))$ for any $n \in \w$ is isomorphic to $\a_n$, which we shall show by induction on $n$.
        For $n=0$ we clearly have that $a \leq A_0 \leq \Delta f(0) + f(0) \leq f(1) \leq A_m$ for any $m \geq 1$.
        From this it follows that $E_\a \cap [a, f(n+1)) = E_\a \cap [a, f(1)) = A_0$, which is isomorphic to $\a_0 = \a_n$ by what was shown above.

        Now suppose that $E_\a \cap [a, f(n+1))$ is isomorphic to $\a_n$.
        We clearly have $E_\a \cap [a, f(n+2)) = \squares{E_\a \cap [a, f(n+1))} \cup \squares{E_\a \cap [f(n+1), f(n+2))}$ and that $E_\a \cap [f(n+1), f(n+2)) = A_{n+1}$, which is isomorphic to $\a_{n+1}$ by what was shown above.
        Then $E_\a \cap [a, f(n+2))$ is the sum of $E_\a \cap [a, f(n+1))$ and $E_\a \cap [f(n+1), f(n+2)) = A_{n+1}$ so that by Theorem~6.5.3, the induction hypothesis, and the given property of $\braces{\a_n}$ it is isomorphic to $\a_n + \a_{n+1} = \a_{n+1}$.
        This completes the inductive proof.

        We also show that $E_\a \cap [a, f(n+1))$ is an initial segment of $E_\a$ for any $n \in \w$.
        So consider any such $n$, any $x \in E_\a \cap [a, f(n+1))$, and any $y \in E_\a$ where $y < x$.
        Since $a \leq E_\a$ clearly $a \leq y$.
        We also have $y < x < f(n+1)$ (since $x \in [a, f(n+1))$).
        Hence $y \in [a, f(n+1))$ so that also $y \in E_\a \cap [a, f(n+1))$, which shows that $E_\a \cap [a, f(n+1))$ is an initial segment of $E_\a$ by definition.

        Now we claim that $E_\a$ is a well-ordered set.
        So consider any nonempty subset $B$ of $E_\a$.
        Then there is some $x \in B$ and since $x \in E_\a$ there is an $n \in \w$ such that $x \in A_n$.
        Then clearly $x \in B \cap [a, f(n+1)]$ so that $B \cap [a, f(n+1)]$ is a nonempty subset of $E_\a \cap [a, f(n+1))$.
        It was shown above that $E_\a \cap [a, f(n+1))$ is isomorphic to $\a_n$ so that it is a well-ordered set.
        Hence $B \cap [a, f(n+1)]$ has a least element $y$.
        We claim that this is the least element of $B$, so consider any $z \in B$.
        If $z < f(n+1)$ then clearly $z \in B \cap [a, f(n+1))$ so that obviously $y \leq z$ since $y$ is the least element of $B \cap [a, f(n+1))$.
        On the other hand if $z \geq f(n+1)$ then $y < f(n+1) \leq z$ (since $y \in [a, f(n+1))$) so that again $y \leq z$.
        Since $z$ was arbitrary this shows that $y$ is in fact the least element of $B$.
        Since $B$ was an arbitrary subset of $E_\a$ this shows that $E_\a$ is well-ordered.

        Since $E_\a$ is a well-ordered set it is isomorphic to some ordinal $\g$ by Theorem~6.3.1.
        We then claim that $\g = \a$.
        Letting $C = \braces{\a_n \where n \in \w}$, we show this by showing that $\g$ is the least upper bound of $C$, which shows that $\g = \a$ by the least upper bound property (see the Theorems document) since $\a = \sup{C}$ by definition.
        So first consider any $\a_n \in C$.
        It was shown above that $\a_n$ is isomorphic to $E_\a \cap [a, f(n+1))$, and this was shown to be an initial segment of $E_\a$, which itself is isomorphic to $\g$.
        Thus it follows that that $\a_n < \g$ so that $\a_n \leq \g$ is true.
        Since $\a_n$ was arbitrary this shows that $\g$ is an upper bound of $C$.

        Now consider any ordinal $\d < \g$ so that $\d \in \g$.
        Let $g$ be the isomorphism from $\g$ to $E_\a$ since it has been shown that they are isomorphic.
        Then since $g(\d) \in E_\a$ so that there is an $n \in \w$ such that $g(\d) \in A_n$.
        Then also $g(\d) \in E_\a \cap [a, f(n+1))$.
        Since $E_\a \cap [a, f(n+1))$ is an initial segment of $E_\a$ (just shown above) it follows that $\inv{g}[E_\a \cap [a, f(n+1))]$ is an initial segment of $\g$.
        Since $\g$ is an ordinal it follows from Lemma~\ex.\ref{lem:emb:isord} that $\inv{g}[E_\a \cap [a, f(n+1))]$ is an ordinal.
        Since $\inv{g} \rest E_\a \cap [a, f(n+1))$ is an isomorphism (since $g$ is) and $E_\a \cap [a, f(n+1))$ is isomorphic to $\a_n$ (shown above), it follows that $\inv{g}[E_\a \cap [a, f(n+1))]$ is in fact $\a_n$!
        Then, since $g(\d) \in E_\a \cap [a, f(n+1))$ we have that $\d \in \inv{g}[E_\a \cap [a, f(n+1))] = \a_n$ so that $\d < \a_n$.
        Hence $\d$ is not an upper bound of $C$.
        Since $\d$ was arbitrary this shows that $\g$ is in fact the least upper bound of $C$ so that $\g = \a$.

        Parts (1) and (3) of the definition of an embedding are trivial to show by the same arguments as those used in the proof of Theorem~\ex.\ref{thrm:emb:prod}.
        Hence $E_\a$ is an embedding of $\a$ in $[a,b)$. \qedsymbol
    }
    
    \mainprob

    (a) Define $f: \w \to \rats$ by
    $$
    f(n) = 1 - \frac{1}{n+1} = \frac{n}{n+1}
    $$
    for $n \in \w$.
    Then let $U_\w = \braces{f(n) \where n \in \w}$.
    We claim that $U_\w$ is a unit embedding of $\w$.
    Clearly $U_\w \ss \rats$ so that (1) is satisfied.

    To show (2) let $g: \w \to U_\a$ be defined by $g(n) = f(n) = 1 - 1/(n+1)$, which is clearly onto based on the definition of $U_\a$.
    We also claim that $g$ is an isomorphism.
    To this end consider any $n,m \in \w$ where $n < m$.
    We then have
    \ali{
        n &< m \\
        n + 1 &< m + 1 \\
        \frac{n+1}{m+1} &< 1 & \text{(since $m+1 \geq 1 > 0$ since $m \geq 0$)} \\
        \frac{1}{m+1} &< \frac{1}{n+1} & \text{(since $n+1 \geq 1 > 0$ since $n \geq 0$)} \\
        -\frac{1}{m+1} &> -\frac{1}{n+1} \\
        1-\frac{1}{m+1} &> 1-\frac{1}{n+1} \\
        g(m) &> g(n) \\
        g(n) &< g(m) \,.
    }
    Hence $g$ is an isomorphism so that $U_\w$ is indeed isomorphic to $\w$, which shows (2).

    Lastly consider any $x \in U_\w$ so that $x = f(n) = 1 - 1/(n+1)$ for some $n \in \w$.
    Then we have
    \ali{
        n &\geq 0 > -1 \\
        n+1 &\geq 1 > 0 \\
        1 &\geq \frac{1}{n+1} > 0 & \text{(since $n+1 \geq 1 > 0$)} \\
        -1 &\leq -\frac{1}{n+1} < 0 \\
        0 &\leq 1 - \frac{1}{n+1} < 1 \\
        0 &\leq x < 1 \,.
    }
    Since $x$ was arbitrary this shows that $0 \leq U_\w < 1$ so that (3) holds and $U_\w$ is a unit embedding.

    Now let $E_1 = \braces{1}$.
    Clearly this is an embedding of the ordinal 1.
    Moreover we have $0 \leq U_\w < 1 \leq E_1 < 2$ so that $U_\w$ and $E_1$ are disjoint.
    Then clearly $U_\w \cup E_1$ is the sum of $U_\w$ and $E_1$ so that it is isomorphic to $\w + 1$ by Theorem~6.5.3 and thus an embedding of $\w + 1$ since it is trivial to show that $0 \leq U_\w \cup E_1 < 2$.

    (b) Now consider the same $U_\w$ from part (a) and let $E_\w = 1 \cdot U_\w + 1$ so that by Theorem~\ex.\ref{thrm:emb:scale} this is another embedding of $\w$ and $1 \leq E_\w < 2$.
    Hence we have that $0 \leq U_\w < 1 \leq E_\w < 2$ so that $U_\w$ and $E_\w$ are disjoint.
    Also clearly $E_{\w \cdot 2} = U_\w \cup E_\w$ is the sum of $U_\w$ and $E_\w$ so that it is isomorphic to $\w + \w = \w \cdot 2$ by Theorem~6.5.3.
    Hence since also $0 \leq E_{\w \cdot 2} < 2$ (it is trivial to show) it is an embedding of $\w \cdot 2$ as desired.

    (c) Considering the same $E_{\w \cdot 2}$ from part (b) let $F_\w = 1 \cdot U_\w + 2$ so that it is yet another an embedding of $\w$ and $2 \leq F_\w < 3$ by Theorem~\ex.2.
    Then by the same arguments as in part (b) it follows that $E_{\w \cdot 2} \cup F_\w$ is an embedding of $\w \cdot 2 + \w = \w \cdot 2 + \w \cdot 1 = \w \cdot (2 + 1) = \w \cdot 3$ as desired.

    \def\Uww{U_{\w^\w}}
    (d) We first aim to construct a unit embedding of $\w^n$ for any $n \in \w$.
    We do this recursively.
    For $n=0$ clearly $U_{\w^0} = \braces{0}$ is a unit embedding of $\w^0 = 1$.
    We have also already constructed $U_\w$ as a unit embedding of $\w$ in part (a).
    Now suppose we have constructed $U_{\w^n}$, a unit embedding of $\w^n$.
    Then we let $U_{\w^{n+1}} = U_\w \cdot U_{\w^n}$ so that $U_{\w^{n+1}}$ is a unit embedding of $\w^n \cdot \w = \w^{n+1}$ by Theorem~\ex.\ref{thrm:emb:prod}.

    Clearly then $\braces{\w^n}$ is a sequence of ordinals in $\w^\w$ and by definition $\w^\w = \sup_{n \in \w} \w^n$.
    We also have in hand the embedding $U_\w$ and $U_{\w^n}$ for each $n \in \w$ as just constructed recursively.
    Now consider any $n \in \w$ so that we have we have
    $$
    \w^n + \w^{n+1} = \w^n \cdot 1 + \w^n \cdot \w = \w^n \cdot (1 + \w) = \w^n \cdot \w = \w^{n+1} \,.
    $$
    We can therefore apply Theorem~\ex.\ref{thrm:emb:sup} to construct a unit embedding $U_{\w^\w}$ of $\w^\w$.

    (e) Here we use the operation of tetration, which we define recursively for all ordinals.
    So for all ordinals $\b$ we define:
    \begin{enumerate}
        \item $\tet{0}{\b} = 1$
        \item $\tet{\a+1}{\b} = \b^{\tet{\a}{\b}}$ for all $\a$
        \item $\tet{\a}{\b} = \sup\braces{\tet{\g}{\b} \where \g < \a}$ for all limit $\a \neq 0$ \,.
    \end{enumerate}
    Thus we have $\tet{1}{\w} = \w$, $\tet{2}{\w} = \w^\w$, $\tet{3}{\w} = \w^{\w^\w}$, etc.
    We then have that $\e = \tet{\w}{\w} = \sup_{n < \w} \tet{n}{\w}$ by definition.

    Despite working and thinking about this problem for weeks I have been unable to come up with a way to embed $\e$.
    To be sure we can easily construct embeddings of ordinals larger than $\w^\w$, for example we can embed $\w^\w \cdot \w^\w = \w^{\w + \w} = \w^{\w \cdot 2}$ by simply applying Theorem~\ex.\ref{thrm:emb:prod} to our just-constructed embedding of $\w^\w$.
    However, I could think of no way to get to $\e$ easily using this approach.
    Ideally what we would have is a way to construct an embedding of $\w^\a$ for any ordinal $\a$ for which we already have an embedding.
    This would easily lead to an embedding of $\e$ using Theorem~\ex.\ref{thrm:emb:sup} after constructing the embeddings for $\tet{n}{\w}$ recursively.
    However, I was unable to think of how to construct this and eventually had to admit defeat and move on.
\end{solution}

\question{6.5.12}

\begin{solution}
	We have
    \ali{
        \parens{\w \cdot 2}^2 &= \parens{\w \cdot 2} \cdot \parens{\w \cdot 2} & \text{(by Example~6.5.10a)} \\
        &= \squares{\w \cdot \parens{2 \cdot \w}} \cdot 2 & \text{(by the associativity of multiplication)} \\
        &= \parens{\w \cdot \w} \cdot 2 \\
        &= \w^2 \cdot 2
    }
    whereas
    \ali{
        \w^2 \cdot 2^2 &= \w^2 \cdot 4
    }
    so that the two are clearly not equal by Exercise~6.5.7b since $2 \neq 4$ and $\w \neq 0$.
\end{solution}

\def\ex{6.5.13}
\setcounter{itm}{0}
\question{\ex}

\begin{solution}
    \begin{statement}{Lemma~\ex.\itm{lem:exp:sups}}
        Suppose that $\a$ is an ordinal, $\b \neq 0$ is a limit ordinal, and $\braces{\g_\d}$ for $\d < \a + \b$ is a non-decreasing transfinite sequence of ordinals.
        Then
        $$
        \sup_{\d < \a + \b} \g_\d = \sup_{\d < \b} \g_{\a + \d} \,.
        $$
    \end{statement}

    \proof{
        First we note that $\a + \b$ is a nonzero limit ordinal by Lemma~6.5.4.1.
        Let $A = \braces{\g_\d \where \d < \a + \b}$ and
        $$
        \g = \sup_{\d < \b} \g_{\a+\d} \,.
        $$
        We show that $\g$ is the least upper bound of $A$, which shows the result by the least upper bound property (see the Theorems document).
        First consider any $\g_\d$ in $A$ so that $\d < \a + \b$.

        If $\d < \a$ then $\g_\d \leq \g_\a = \g_{\a + 0} \leq \g$ since the sequence is non-decreasing and $0 < \b$.
        On the other hand if $\d \geq \a$ then by Lemma~6.5.5 there is a unique ordinal $\e$ such that $\a + \e = \d$.
        Then, since $\d = \a + \e < \a + \b$ it follows from Lemma~6.5.4a that $\e < \b$.
        Hence by the definition of $\g$ we have that $\g_\d = \g_{\a + \e} \leq \g$.
        Thus in all cases $\g_\d \leq \g$, which shows that $\g$ is an upper bound of $A$ since $\g_\d$ was arbitrary.

        Now consider any $\e < \g$.
        Then by the definition of $\g$ it follows that there is a $\d < \b$ such that $\e < \g_{\a+\d}$ since $\e$ is not an upper bound $\braces{\g_{\a+\z} \where \z < \b}$.
        Thus again by Lemma~6.5.4a we have that $\a + \d < \a + \b$ so that $\g_{\a+\d} \in A$.
        Hence since $\e < \g_{\a + \d}$ it cannot be that $\e$ is an upper bound of $A$.
        Since $\e < \g$ was arbitrary this completes the proof that $\g = \sup A$ as desired. \qedsymbol
    }

    \begin{statement}{Lemma~\ex.\itm{lem:exp:zero}}
        If $\a$ is an ordinal such that $\a > 0$ then $0^\a = 0$.
        Otherwise if $\a = 0$ then $0^\a = 0^0 = 1$.
    \end{statement}

    \proof{
        First if $\a = 0$ then clearly $0^\a = 0^0 = 1$ by Definition~6.5.9a.
        Then if $\a > 0$ we show the result by transfinite induction on $\a$.
        First if $\a = 1$ then we have
        $$
        0^\a = 0^1 = 0^{0+1} = 0^0 \cdot 0 = 1 \cdot 0 = 0 \,,
        $$
        where we have used Definitions~6.5.9b and 6.5.6a.
        Now suppose that $0^\a = 0$ so that we have
        $$
        0^{\a+1} = 0^\a \cdot 0 = 0 \,,
        $$
        where we again have used the same two definitions.
        Lastly suppose that $\a$ is a nonzero limit ordinal and that $0^\b = 0$ for all $\b < \a$.
        Then we have by Definition~6.5.9c that
        $$
        0^\a = \sup_{\b < \a} 0^\b = \sup_{\b < \a} 0 = 0 \,,
        $$
        where we have used the induction hypothesis.
    }

    \begin{statement}{Lemma~\ex.\itm{lem:exp:one}}
        If $\a$ is an ordinal then $1^\a = 1$.
    \end{statement}

    \proof{
        We show this by transfinite induction on $\a$.
        First for $\a = 0$ we have $1^\a = 1^0 = 1$ by Definition~6.5.9a.
        Now assume that $1^\a = 1$ so that $1^{\a+1} = 1^\a \cdot 1 = 1^\a = 1$ by Definition~6.5.9b and the induction hypothesis.

        Lastly suppose that $\a$ is a nonzero limit ordinal and that $1^\b = 1$ for all $\b < \a$.
        We then clearly have by Definition~6.5.9c that
        $$
        1^\a = \sup_{\b < \a} 1^\b = \sup_{\b < \a} 1 = 1 \,.
        $$
        This completes the inductive proof. \qedsymbol
    }

    \begin{statement}{Lemma~\ex.\itm{lem:exp:gt}}
        If $\a$, $\b$, and $\g$ are ordinals where $\a > 0$ and $\b \leq \g$ then $\a^\b \leq \a^\g$.
    \end{statement}

    \proof{
        Clearly if $\b = \g$ then $\a^\b = \a^\g$ so that the conclusion holds.
        So assume that $\b < \g$.

        Case: $\a = 1$.
        Then by Lemma~\ex.\ref{lem:exp:one} we have
        $$
        \a^\b = 1^\b = 1 = 1^\g = \a^\g \,.
        $$

        Case: $\a > 1$.
        Then it follows from Exercise~6.5.14b that $\a^\b < \a^\g$.

        Hence in either case $\a^\b \leq \a^\g$ is true. \qedsymbol
    }

	(a)
    We show this by transfinite induction on $\g$.
    First for $\g = 0$ we have
    \ali{
        \a^{\b + \g} &= \a^{\b + 0} \\
        &= \a^\b & \text{(by Definition~6.5.1a)} \\
        &= \a^\b \cdot 1 & \text{(by Example~6.5.7a )} \\
        &= \a^\b \cdot \a^0 & \text{(by Definition~6.5.9a)} \\
        &= \a^\b \cdot \a^\g \,.
    }
    Now assume that $\a^{\b + \g} = \a^\b \cdot \a^\g$ so that we have
    \ali{
        \a^{\b + (\g + 1)} &= \a^{(\b + \g) + 1} & \text{(by Definition~6.5.1b)} \\
        &= \a^{\b+\g} \cdot \a & \text{(by Definition~6.5.9b)} \\
        &= \parens{\a^\b \cdot \a^\g} \cdot \a & \text{(by the induction hypothesis)} \\
        &= \a^\b \cdot \parens{\a^\g \cdot \a} & \text{(by the associativity of multiplication)} \\
        &= \a^\b \cdot \a^{\g+1} \,. & \text{(by Definition~6.5.9b)}
    }
    Lastly suppose that $\g$ is a nonzero limit ordinal and that $\a^{\b+\d} = \a^\b \cdot \a^\d$ for all $\d < \g$.
    First if $\a = 0$ then $\a^{\b+\g} = 0^{\b+_\g} = 0$ by Lemma~\ex.\ref{lem:exp:zero} since $\b+\g > 0$ since $\g > 0$.
    Hence we have
    $$
    \a^{\b+ \g} = 0 = 0^\b \cdot 0 = 0^\b \cdot 0^\g = \a^\b \cdot \a^\g \,,
    $$
    where we have used Definition~6.5.6a and the fact that $0^\g = 0$ by Lemma~\ex.\ref{lem:exp:zero} since $\g > 0$.

    On the other hand if $\a > 0$ then we have by Definition~6.5.9c that
    $$
    \a^{\b+\g} = \sup_{\d < \b+\g} \a^\d \,.
    $$
    It then follows from Lemma~\ex.\ref{lem:exp:gt} that $\braces{\a^\d}$ for $\d < \b + \g$ is a non-decreasing sequence since $\a > 0$.
    Thus we can apply Lemma~\ex.\ref{lem:exp:sups} so that
    $$
    \a^{\b+\g} = \sup_{\d < \b+\g} \a^\d = \sup_{\d < \g} \a^{\b + \d} = \sup_{\d < \g} \parens{\a^\b \cdot \a^\d}
    $$
    by the induction hypothesis.
    It then follows from statement (6.6.1) in the text that
    $$
    \a^{\b+\g} = \sup_{\d < \g} \parens{\a^\b \cdot \a^\d} = \a^\b \cdot \sup_{\d < \g} \a^\d = \a^\b \cdot \a^\g
    $$
    by Definition~6.5.9c.
    This completes the transfinite induction. \qedsymbol

    (b) We show this by transfinite induction on $\g$ as well.
    First for $\g = 0$ we have
    \ali{
        \parens{\a^\b}^\g &= \parens{\a^\b}^0 = 1 & \text{(by Definition~6.5.9a)} \\
        &= \a^0 & \text{(again by Definition~6.5.9a)} \\
        &= \a^{\b \cdot 0} = \a^{\b \cdot \g} \,. & \text{(by Definition~6.5.6a)}
    }
    Next suppose that $\parens{\a^\b}^\g = \a^{\b \cdot \g}$ so that we have
    \ali{
        \parens{\a^\b}^{\g+1} &= \parens{\a^\b}^\g \cdot \a^\b & \text{(by Definition~6.5.9b)} \\
        &= \a^{\b \cdot \g} \cdot \a^\b & \text{(by the induction hypothesis)} \\
        &= \a^{\b \cdot \g + \b} & \text{(by part a)} \\
        &= \a^{\b \cdot \parens{\g + 1}} \,. & \text{(by Definition~6.5.6b)}
    }
    Lastly, suppose that $\g$ is a nonzero limit ordinal and that $\parens{\a^\b}^\d = \a^{\b \cdot \d}$ for all $\d < \g$.

    Case: $\a = 0$.
    Then if $\b = 0$ we have
    \ali{
        \parens{\a^\b}^\g &= \parens{0^0}^\g \\
        &= 1^\g & \text{(by Definition~6.5.9a)} \\
        &= 1 & \text{(by Lemma~\ex.\ref{lem:exp:one})} \\
        &= 0^0 & \text{(by Definition~6.5.9a)} \\
        &= 0^{0 \cdot \g} & \text{(by Lemma~6.5.1.1)} \\
        &= \a^{\b \cdot \g} \,.
    }
    On the other hand if $\b > 0$ then first we note that $0 = \b \cdot 0 < \b \cdot \g$ by Definition~6.5.6a and Exercise~6.5.7a since both $\b \neq 0$ and $0 < \g$.
    We then have
    \ali{
        \parens{\a^\b}^\g &= \parens{0^\b}^\g \\
        &= 0^\g & \text{(by Lemma~\ex.\ref{lem:exp:zero} since $\b > 0$)} \\
        &= 0 & \text{(by Lemma~\ex.\ref{lem:exp:zero} again since $\g > 0$)} \\
        &= 0^{\b \cdot \g} & \text{(by Lemma~\ex.\ref{lem:exp:zero} yet again since $\b \cdot \g > 0$ as shown above)} \\
        &= \a^{\b \cdot \g} \,.
    }
    Case: $\a > 0$.
    Then we have
    \ali{
        \parens{\a^\b}^\g &= \sup_{\d < \g} \parens{\a^\b}^\d & \text{(by Definition~6.5.9c)} \\
        &= \sup_{\d < \g} \a^{\b \cdot \d} & \text{(by the induction hypothesis)} \\
        &= \a^{\sup_{\d < \g} \b \cdot \d} & \text{(by statement (6.6.1) in the text since $\a > 0$)} \\
        &= \a^{\b \cdot \g} & \text{(by Definition~6.5.6c)}
    }
    Since these cases are exhaustive this completes the inductive proof. \qedsymbol
\end{solution}

\def\ex{6.5.14}
\setcounter{itm}{0}
\question{\ex}

\begin{solution}
    \begin{statement}{Lemma~\ex.\itm{lem:expin:sups}}
        Suppose that $\g$ is a nonzero limit ordinal and $\braces{\a_\n}$ and $\braces{\b_\n}$ for $\n < \g$ are two transfinite sequences.
        Also suppose that $\a_\n \leq \b_\n$ for every $\n < \g$.
        Then
        $$
        \sup_{\n < \g} \a_\n \leq \sup_{\n < \g} \b_\n \,.
        $$
    \end{statement}

    \proof{
        First let $A = \braces{\a_\n \where \n < \g}$ and $B = \braces{\b_\n \where \n < \g}$ be the ranges of the sequences so that we must show that $\sup{A} \leq \sup{B}$.
        Now consider any $\a \in A$ so that $\a = \a_\n$ for some $\n < \g$.
        We then have that
        $$
        \a = \a_\n \leq \b_\n \leq \sup{B}
        $$
        so that $\sup{B}$ is an upper bound of $A$ since $\a$ was arbitrary.
        It then follows from the least upper bound property that $\sup{A} \leq \sup{B}$ as desired. \qedsymbol
    }

    \mainprob

    (a) We show this by transfinite induction on $\g$.
    So first suppose that $\a \leq \b$.
    Then for $\g = 0$ we clearly have
    $$
    \a^\g = \a^0 = 1 = \b^0 = \b^\g \,,
    $$
    where we have used Definition~6.5.9a twice.
    Then $\a^\g \leq \b^\g$ clearly holds.

    Now suppose that $\a^\g \leq \b^\g$ so that we have
    \ali{
        \a^{\g+1} &= \a^\g \cdot \a & \text{(by Definition~6.5.9b}) \\
        &\leq \b^\g \cdot \a & \text{(follows from Exercise~6.5.8b and the induction hypothesis)} \\
        &\leq \b^\g \cdot \b & \text{(follows from Exercise~6.5.7 since $\a \leq \b$)} \\
        &= \b^{\g + 1} \,. & \text{(by Definition~6.5.9b)}
    }
    Lastly suppose that $\g$ is a nonzero limit ordinal and that $\a^\d \leq \b^\d$ for all $\d < \g$.
    We then have
    \ali{
        \a^\g &= \sup_{\d < \g} a^\d & \text{(by Definition~6.5.9c)} \\
        &\leq \sup_{\d < \g} \b^\d & \text{(by Lemma~\ex.\ref{lem:expin:sups} and the induction hypothesis)} \\
        &= \b^\g\,, & \text{(by Definition~6.5.9c again)}
    }
    which completes the transfinite induction. \qedsymbol

    (b) Suppose that $\a > 1$.
    We then show the result by transfinite induction on $\g$ similarly to the proof of Exercise~6.5.7a.
    Suppose that $\b < \d$ implies that $\a^\b < \a^\d$ for all $\d < \g$ and that $\b < \g$.

    Case: $\g$ is a successor ordinal.
    Then $\g = \d + 1$ for an ordinal $\d$ and $\b \leq \d$ since $\b < \g = \d + 1$.
    We then have
    \ali{
        \a^\b &\leq \a^\d & \text{(by the induction hypothesis if $\b < \d$ and trivially if $\b = \d$)} \\
        &= \a^\d \cdot 1 \\
        &< \a^\d \cdot \a & \text{(by Exercise~6.5.7a since $1 < \a$ and $\a^\d \neq 0$)} \\
        &= \a^{\d+1} & \text{(by Definition~6.5.9b)} \\
        &= \a^\g \,.
    }
    Case: $\g$ is a limit ordinal.
    Here we have that $\b+1 < \g$ as well since $\b < \g$.
    We then have
    \ali{
        \a^\b &< \a^{\b+1} & \text{(by the induction hypothesis since $\b < \b+1 < \g$)} \\
        &\leq \sup_{\d < \g} \a^\d & \text{(since the supremum is an upper bound)} \\
        &= \a^\g \,. & \text{(by Definition~6.5.9c)}
    }
    Thus in either case $\a^\b < \a^\g$, thereby completing the transfinite induction. \qedsymbol
\end{solution}

\def\ex{6.5.15}
\setcounter{itm}{0}
\question{\ex}

\begin{solution}
    \begin{statement}{Lemma~\ex.\itm{lem:props:sups}}
        Suppose that $\a$ and $\g$ are a nonzero limit ordinals, that $\braces{\a_\d}$ is a transfinite sequence indexed by $\g$, and that
        $$
        \a = \sup_{\d < \g} \a_\d \,.
        $$
        Also suppose that ${\b_\d}$ is another non-decreasing transfinite sequence indexed by $\a$.
        Then
        $$
        \sup_{\d < \a} \b_\d = \sup_{\d < \g} \b_{\a_\d} \,.
        $$
    \end{statement}
    
    \proof{
        First, let $A = \braces{\a_\d \where \d < \g}$, $B = \braces{\b_\d \where \d < \a}$, and $C = \braces{\b_{\a_\d} \where \d < \g}$ so that $\a = \sup{A}$ and we must show that $\sup{B} = \sup{C}$.
        We show this by showing that $\sup{C}$ has the least upper bound property of $B$.

        So first consider any $\b_\d \in B$ so that $\d < \a$.
        It then follows that  $\d$ is not an upper bound of $A$ so that there is a $\x < \g$ such that $\d < \a_\x$.
        From this we have that $\b_\d \leq \b_{\a_\x} \leq \sup{C}$ since $\braces{\b_\d}$ is a non-decreasing sequence and $\b_{\a_\x} \in C$.
        Since $\b_\d$ was arbitrary this shows that $\sup{C}$ is an upper bound of $B$.

        Now consider any $\d < \sup{C}$.
        Then $\d$ is not an upper bound of $C$ so that there is a $\x < \g$ such that $\d < \b_{\a_\x}$.
        And since $\x < \g$ we have that $\a_\x \in A$ so that $\a_\x < \sup{A} = \a$ since $\a$ is a limit ordinal.
        Thus $\b_{\a_\x} \in B$.
        Since we have that $\d < \b_{\a_\x}$ this show that $\d$ is not an upper bound of $B$.
        Hence since $\d$ was arbitrary this concludes the proof that $\sup{B} = \sup{C}$ by the least upper bound property. \qedsymbol
    }

    \mainprob

    (a) We claim that the least such ordinal here is $\x = \w^2$.
    That this has the desired property (i.e. that $\w + \x = \x$) was shown in Exercise~6.5.3b.

    Now we show that any ordinal $\a < \w^2$ does not have this property.
    So consider any such $\a$ so that by Lemma~6.5.6.4 there are natural numbers $n$ and $k$ such that $\a = \w \cdot n + k$.
    Thus we have
    $$
    \w + \a = \w + \w \cdot n + k = \w \cdot 1 + \w \cdot n + k = \w \cdot (1 + n) + k = w \cdot (n + 1) + k
    $$
    We then have that
    \ali{
        \a &= \w \cdot n + k = \parens{\w \cdot n + k} + 0 \\
        &< \parens{\w \cdot n + k} + \w & \text{(by Lemma~6.5.4a since $0 < \w$)} \\
        &= \w \cdot n + (k + \w) & \text{(associativity of addition)} \\
        &= \w \cdot n + \w = \w \cdot n + \w \cdot 1 \\
        &= \w \cdot (n + 1) & \text{(distributive law)} \\
        &= \w \cdot (n + 1) + 0 \\
        &\leq \w \cdot (n + 1) + k & \text{(by Lemma~6.5.4 since $0 \leq k$)} \\
        &= \w + \a
    }
    so that clearly $\w + \a \neq \a$.
    Since $\a < \w^2$ was arbitrary this shows our result. \qedsymbol

    (b) We claim that $\x = \w^\w$ is the first nonzero ordinal to have this property.

    First we have
    \ali{
        \w \cdot \x = \w \cdot \w^\w = \w^1 \cdot \w^\w = \w^{1 + \w} = \w^\w = \x \,
    }
    where we have used Exercise~6.5.13a.
    Thus $\x = \w^2$ has the desired property.

    Now consider any $0 < \a < \w^\w$.
    Since by Definition~6.5.9c we have that $\w^\w = \sup_{n < \w} \w^n$ it follows from the least upper bound property that there is an $n < \w$ such that $\a < \w^n$ since $\a$ is not an upper bound of $\braces{\w^n \where n < \w}$.
    From this is follows that the set $A = \braces{k \in \w \where \a < \w^k}$ is not empty.
    Since this is a set of natural numbers (which is well-ordered) it has a least element $m$.
    Note also that it has to be that $m > 0$ since were it the case that $m=0$ then we would have $\a < \w^m = \w^0 = 1$, which implies that $\a = 0$, which contradicts our initial supposition that $0 < \a$.
    Thus $m \geq 1$ so that $m-1$ is still a natural number.

    Since $m$ is the least element of $A$ it follows that $\a \geq \w^{m-1}$ since otherwise $m-1$ would be the least element of $A$.
    Hence we have
    $$
    \w^{m-1} \leq \a < \w^m \,.
    $$
    It then follows from Exercise~6.5.8b that
    \ali{
        \w \cdot \w^{m-1} &\leq \w \cdot \a \\
        \w^1 \cdot \w^{m-1} &\leq \w \cdot \a \\
        \w^{1 + m - 1} &\leq \w \cdot \a & \text{(by Exercise~6.5.13a)} \\
        \w^m &\leq \w \cdot \a \,.
    }
    Putting this together, we have that
    $$
    \a < \w^m \leq \w \cdot \a
    $$
    so that clearly $\w \cdot \a \neq \a$.
    Since $1 < \a < \w^\w$ was arbitrary, this shows that $\w^\w$ is the least such ordinal \qedsymbol

    (c) We claim here that $\e = \tet{\w}{\w}$ is the least such ordinal, where we use the notation for tetration introduced in Exercise~6.5.11e.

    To show that $\e$ has the required property we must first show that the sequence defined by $\braces{\tet{n}{\w}}$ for $n < \w$ is non-decreasing even though this is somewhat obvious.
    We show this by induction.
    For $n=0$ we have
    $$
    \tet{n}{\w} = \tet{0}{\w} = 1 < \w = \w^1 = \w^{\tet{0}{\w}} = \tet{1}{\w} = \tet{n+1}{\w}
    $$
    Now suppose that $\tet{n}{\w} < \tet{n+1}{\w}$ so that we have
    \ali{
        \tet{n+1}{\w} &= \w^{\tet{n}{\w}} \\
        &< \w^{\tet{n+1}{\w}} & \text{(by Exercise~6.5.14b and the induction hypothesis)} \\
        &= \tet{n+2}{\w} \,,
    }
    which completes the induction.
    
    Returning to the main problem, we then have
    \ali{
        \w^\e &= \sup_{\a < \e} \w^\a & \text{(by Definition~6.5.9c since $\e$ is clearly a limit ordinal)} \\
        &= \sup_{n < \w} \w^{\tet{n}{\w}} & \text{(by Lemma~\ex.\ref{lem:props:sups} since $\braces{\tet{n}{\w}}$ is non-decreasing)} \\
        &= \sup_{n < \w} \tet{n+1}{\w} = \tet{\w}{\w} = \e
    }
    so that $\e$ does have the desired property.

    Now we must show that it is the least such ordinal that has this property.
    So consider any $\a < \e$ then since $\e = \sup_{n < \w} \tet{n}{\w}$ it follows that there is an $n < \w$ such that $\a < \tet{n}{\w}$ by the least upper bound property.
    It then follows that the set $A = \braces{k \in \w \where \a < \tet{k}{\w}}$ is not empty.
    Since this is a set of natural numbers it follows that it has a least element $m$.

    Case: $m=0$.
    Then $\a < \tet{m}{\w} = \tet{0}{\w} = 1$ so that it must be that $\a = 0$.
    But then we have
    $$
    \w^\a = \w^0 = 1 \neq 0 = \a
    $$
    so that $\a$ does not have the property.

    Case: $m > 0$.
    Then $m \geq 1$ so that $m-1$ is still a natural number.
    It then follows that $\a \geq \tet{m-1}{\w}$ since otherwise $m-1$ would be the least element of $A$.
    Thus we have
    $$
    \tet{m-1}{\w} \leq \a < \tet{m}{\w} \,.
    $$
    It then follows from Exercise~6.5.14b that
    \ali{
        \w^{\tet{m-1}{\w}} &\leq \w^\a \\
        \tet{m}{\w} &\leq \w^\a \,.
    }
    Thus we have
    $$
    \a < \tet{m}{\w} \leq \w^{\a}
    $$
    so that clearly $\a$ does not have the property since $\w^\a \neq \a$.
    Since $\a$ was arbitrary and the cases exhaustive this shows that $\e$ is indeed the least such ordinal. \qedsymbol
\end{solution}

\def\ex{6.5.16}
\setcounter{itm}{0}
\question{\ex}

\begin{solution}
    First we define summation notation for ordinals.
    Since ordinal addition is \emph{not} commutative we define summation notation such that each additional term is pre-added to the previous terms, i.e. added on the left.
    So, for example, we have
    $$
    \sum_{n = 1}^5 \a_n = \a_5 + \a_4 + \a_3 + \a_2 + \a_1 \,.
    $$
    We also adopt the convention that
    $$
    \sum_{k=n}^m \alpha_k = 0
    $$
    any time $n > m$.

    First we note that if $\b = 0 = \es$ then the only function from $\b$ to $\a$ (it  could even be here that $\a = 0 = \es$) is the vacuous function $\es$.
    Hence $S(\b, \a) = \braces{\es}$, which is clearly vacuously isomorphic to (in fact identical to) $\a^\b = \a^0 = 1 = \braces{0} = \braces{\es}$.
    Thus in the following we assume that $\b \neq 0$, which implies that $\a \neq 0$ as well since functions from $\b$ to $\a$ cannot exist when $\b \neq 0 = \es$ but $\a = 0 = \es$.

    Also if $\a = 1 = \braces{0}$ (and still $\b \neq 0$) then there is clearly only a single function $f: \b \to \a$, namely that where $f(\x) = 0$ for all $\x \in \b$.
    Hence $S(\b, \a) = \braces{f}$, which is clearly trivially isomorphic to $\a^\b = 1^\b = 1 = \braces{0}$.
    Thus in what follows we shall assume the more interesting case when $\a > 1$ (and $\b > 0$).

    Now we define a function $h : S(\b, \a) \to \a^\b$.
    For any $f \in S(\b, \a)$ since $s(f)$ is a finite set of ordinals it follows that it is isomorphic to some natural number $n$.
    Thus its elements can be expressed as a  strictly increasing sequence $\braces{s_k}$ for $k < n$ where each $s_k \in \b$ since $s(f) \ss \b$.
    We now set
    $$
    h(f) = \sum_{k=0}^{n-1} \a^{s_k} \cdot f(s_k) \,,
    $$
    noting that it could be the case that $n=0$ so that $h(f) = 0$, consistent with our convention.

    First we show that $h(f) \in \a^\b$ so that the range of $h$ is in fact a subset of $\a^\b$.
    We begin by showing by induction on $m$ that
    $$
    \sum_{k=0}^m \a^{s_k} \cdot f(s_k) < \a^{s_m + 1}  \,.
    $$
    First for $m=0$ we have
    \ali{
        \sum_{k=0}^m \a^{s_k} \cdot f(s_k) &= \sum_{k=0}^0 \a^{s_k} \cdot f(s_k) \\
        &= \a^{s_0} \cdot f(s_0) \\
        &< \a^{s_0} \cdot \a & \text{(by Exercise~6.5.7a since $\a \neq 0$ and $f(s_0) < \a$)} \\
        &= \a^{s_0 + 1} = \a^{s_m + 1} & \text{(by Definition~6.5.9b)}
    }
    Now suppose that
    $$
    \sum_{k=0}^m \a^{s_k} \cdot f(s_k) < \a^{s_m + 1} \,.
    $$
    It follows from Exercise~6.5.14b that
    $$
    \a^{s_m + 1} \leq \a^{s_{m+1}}
    $$
    since $\braces{s_k}$ is a strictly increasing sequence so that $s_m + 1 \leq s_{m+1}$.
    We then have
    \ali{
        \sum_{k=0}^{m+1} \a^{s_k} \cdot f(s_k) &= \a^{s_{m+1}} \cdot f(s_{m+1}) + \sum_{k=0}^m \a^{s_k} \cdot f(s_k) \\
        &< \a^{s_{m+1}} \cdot f(s_{m+1}) + \a^{s_m + 1} & \text{(by the induction hypothesis and Lemma~6.5.4a)} \\
        &\leq \a^{s_{m+1}} \cdot f(s_{m+1}) + \a^{s_{m+1}} & \text{(follows from Lemma~6.5.4 and the above)} \\
        &= \a^{s_{m+1}} \cdot \squares{f(s_{m+1}) + 1} & \text{(by Definition~6.5.6b)} \\
        &= \a^{s_{m+1}} \cdot \a & \text{(by Exercise~6.5.7 since $f(s_{m+1}) + 1 \leq \a$)} \\
        &= \a^{s_{m+1} + 1} \,, & \text{(by Definition~6.5.9b)}
    }
    which completes the induction.
    Hence we have
    $$
    h(f) = \sum_{k=0}^{n-1} \a^{s_k} \cdot f(s_k) < \a^{s_{n-1} + 1} \leq \a^\b
    $$
    by Exercise~6.5.14b since $s_{n-1} + 1 \leq \b$ since $s_{n-1} < \b$.
    Clearly then $h(f) \in \a^\b$ by definition of ordinal ordering.

    Now we show that $h$ is an increasing function and therefore also injective.
    So consider any $f$ and $g$ in $S(\b, \a)$ such that $f \prec g$.
    Then by the definition of $\prec$ there is a $\x_0 < \b$ such that $f(\x_0) < g(\x_0)$ and $f(\x) = g(\x)$ for all $\x_0 < \x < \b$.
    Now let $S = s(f) \cup  s(g)$, which is clearly a finite set of ordinals since $s(f)$ and $s(g)$ are.
    Hence it is also isomorphic to a natural number $n$ so that it can be expressed as a strictly increasing sequence $\braces{s_k}$ for $k < n$.
    Moreover
    \ali{
        h(f) &= \sum_{k=0}^{n-1} \a^{s_k} \cdot f(s_k) &
        h(g) &= \sum_{k=0}^{n-1} \a^{s_k} \cdot g(s_k)
    }
    since for each term where $s_k \notin s(f)$ we have that $f(s_k) = 0$ so that the term contributes nothing to the sum by Definition~6.5.6a (and similarly for $g$).
    Also since $f(\x_0) < g(\x_0)$ is has to be that $0 < g(\x_0)$ so that $\x_0 \in s(g)$ and hence $\x_0 \in S$.
    From this it follows that there is an $m < n$ such that $s_m = \x_0$.

    We then have
    \ali{
        \a^{s_m} \cdot f(s_m) &+ \sum_{k=0}^{m-1} \a^{s_k} \cdot f(s_k) \\
        &< \a^{s_m} \cdot f(s_m) + \a^{s_{m-1} + 1} & \text{(by Lemma~6.5.4a and the above)} \\
        &\leq \a^{s_m} \cdot f(s_m) + \a^{s_m} & \text{(since $\braces{s_k}$ is an increasing sequence)} \\
        &= \a^{s_m} \cdot \squares{f(s_m) + 1} & \text{(by Definition 6.5.6b)} \\
        &\leq \a^{s_m} \cdot g(s_m) & \text{(since $f(s_m) < g(s_m)$)} \\
        &\leq \a^{s_m} \cdot g(s_m) + \sum_{k=0}^{m-1} \a^{s_k} \cdot g(s_k) \,. & \text{(by Lemma~6.5.4)}
    }
    Thus it follows yet again from Lemma~6.5.4a that
    \ali{
        h(f) &= \sum_{k=0}^{n-1} \a^{s_k} \cdot f(s_k) \\
        &= \sum_{k=m+1}^{n-1} \a^{s_k} \cdot f(s_k) + \a^{s_m} \cdot f(s_m) + \sum_{k=0}^{m-1} \a^{s_k} \cdot f(s_k) \\
        &< \sum_{k=m+1}^{n-1} \a^{s_k} \cdot g(s_k) + \a^{s_m} \cdot g(s_m) + \sum_{k=0}^{m-1} \a^{s_k} \cdot g(s_k) \\
        &= \sum_{k=0}^{n-1} \a^{s_k} \cdot g(s_k) = h(g) \,,
    }
    which shows that $h$ is indeed increasing.

    Lastly we show that $h$ is a surjective function, which then completes the proof that $h$ is an isomorphism so that $(S(\b,\a), \prec)$ is isomorphic to $(\a^\b, <)$ as desired.
    So consider any $\g \in \a^\b$ so that by definition $\g < \a^\b$.
    
    We construct a function $f: \b \to \a$ recursively by first initializing $f(\x) = 0$ for all $\x \in \b$.
    We then initialize $\r = \g$ and perform the following iteratively:

    If $\r < \a$ then we set $f(0) = \r$, noting that by definition $\r \in \a$ and $0 \in \b$ since $\b > 0$.
    We then terminate the iteration, noting that it could be that $f(0) = \r = 0$. 

    If $\r \geq \a$ then $1 < \a \leq \r$ so that by Lemma~6.6.2 there is a greatest ordinal $\s$ such that $\a^\s \leq \r$.
    Then clearly we have $\a^\s \leq \r \leq \g < \a^\b$ so that it follows from Exercise~6.514b that $\s < \b$ since $\a > 1$ (the exercise only asserts implication in one direction but the other direction follows immediately similarly to the proof of Lemma~6.54a).
    Then since clearly $\a^\s \neq 0$ it follows from Theorem~6.6.3 that there are unique ordinals $\t$ and $\r'$ such that
    $$
    \a^\s \cdot \t + \r' = \r
    $$
    and $\r' < \a^\s$.
    We claim the following about these:
    \begin{enumerate}
        \item $\t < \a$.
        To the contrary, assume that $\t \geq \a$ so that we have
        $$
        \r = \a^\s \cdot \t + \r' \geq \a^\s \cdot \t + 0 = \a^\s \cdot \t \geq \a^\s \cdot \a = \a^{\s + 1}
        $$
        by Lemma~6.5.4, Exercise~6.5.7 since $\a^\s \neq 0$, and Definition~6.5.9b.
        However, this contradicts the definition of $\s$, i.e. that it is greatest ordinal $\x$ such that $\r \geq \a^\x$.
        Hence it must be that $\t < \a$.

        \item $0 < \t$.
        Were it the case that $\t = 0$ then we would have
        $$
        \r = \a^\s \cdot \t + \r' = \a^\s \cdot 0 + \r' = 0 + \r' = \r' \,.
        $$
        However, we then would have both $\r \geq \a^\s$ and $\r = \r' < \a^\s$, which is clearly a contradiction.
        So it must be that $\t \neq 0$ so $\t > 0$.

        \item $\r' < \r$.
        We have
        $$
        \r' < \a^\s = \a^\s \cdot 1 \leq \a^\s \cdot \t = \a^\s \cdot \t + 0 \leq \a^\s \cdot \t + \r' = \r
        $$
        by Exercise~6.5.7 since $\a^\s \neq 0$ and Lemma~6.5.4 since $0 \leq \r'$, noting that we also just showed that $1 \leq \t$ since $0 < \t$.
    \end{enumerate}

    We therefore set $f(\s) = \t$, noting that we have shown that $\s < \b$ and $\t < \a$ (so that $\s \in \b$ and $\t \in \a$).
    We then repeat the above, setting $\r'$ as the new $\r$, noting that $\r' < \r \leq \g$ so that $\r' \leq \g$ is still true.

    Now it has to be that this construction terminates after a finite number of iterations since each $\r$ in the iterations forms a strictly decreasing sequence of ordinals.
    Hence this sequence has to be finite since otherwise the range of this sequence would be a set of ordinals with no least element.
    It thus follows that $f(\x) \neq 0$ for a finite number of $\x \in \b$ so that $s(f)$ is finite and $f \in S(\b,\a)$.
    The fact that $h(f) = \g$ follows immediately from the construction of $f$ so that $h$ is surjective since $\g$ was arbitrary. \qedsymbol

    In conclusion we note that $h$ maps $f \in S(\b,\a)$ to its corresponding ordinal $\g \in \a^\b$ by expanding $\g$ in base-$\a$ with digits in the range of $f$.
    Running through the above procedures for finite $\a$ and $\b$ shows that this is the usual base expansion in base-$\a$.
    However, this is much more general since $\a$ or $\b$ (or both) might be transfinite.
    The interesting conclusion here is that any ordinal can be expressed with a finite number of (potentially transfinite) digits with respect to any other ordinal (potentially transfinite) as a base.
    It would seem that Theorem~6.6.4 (the normal form) is a special case of this in which the base is $\w$.
\end{solution}

\question{6.6.1}

\begin{solution}
	This was shown in Exercise~6.5.15c.
\end{solution}

\question{6.6.2}

\begin{solution}
	We show the first 5 terms of the Goodstein sequences for $m = 28$:
    \ali{
        m_0 &= m = 28 = 2^{2^2} + 2^{2 + 1} + 2^2 \\
        m_1 &= 3^{3^3} + 3^{3 + 1} + 3^3 - 1 = 3^{3^3} + 3^{3 + 1} + 3^2 \cdot 2 + 3 \cdot 2 + 2 \approx 7.626 \times 10^{12} \\
        m_2 &= 4^{4^4} + 4^{4 + 1} + 4^2 \cdot 2 + 4 \cdot 2 + 1 \approx 1.341 \times 10^{154} \\
        m_3 &= 5^{5^5} + 5^{5 + 1} + 5^2 \cdot 2 + 5 \cdot 2 \approx 1.911 \times 10^{2182} \\
        m_4 &= 6^{6^6} + 6^{6 + 1} + 6^2 \cdot 2 + 6 \cdot 2 - 1 = 6^{6^6} + 6^{6 + 1} + 6^2 \cdot 2 + 6 + 5 \approx 2.659 \times 10^{36305} \,.
    }
\end{solution}
